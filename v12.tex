\section{Vorlesung 12}
\begin{align*}
\int \frac{f'(x)}{f(x)}dx &= ln |f(x)| + c \quad c \in \mathbb{R} \\
( ln | f(x)| )'   &= 
\begin{cases} 
(ln f(x))' , f(x)> 0 \\
(ln f(x))' , f(x)< 0 \\ 
\end{cases}\\
&=
\begin{cases}
\frac{1}{f(x)}f'(x),f(x) >0 \\
\frac{1}{-f(x)}-f'(x),f(x) < 0\\
\end{cases} \\
&= 
\begin{cases}
\dfrac{f'(x)}{f(x)}
\end{cases}
\end{align*}
\begin{example}
\begin{align*}
\int{\frac{ 2x +2 +1 }{ x^2 + 2x +5} dx }
&= \int{\frac{ 2x + 2  }{ x^2 + 2x +5} dx } + 
\int{\frac{1}{ x^2 + 2x +5} dx }\\
 &= \underbrace{ln(x^2 + 2x + 5)}_{\text{keine reelle Nullstell}}  + \quad ? 
\end{align*}
\subsection{Kettenregel $\rightsquigarrow$ Integration durch \textbf{Substitution}}
\end{example}
\begin{align*}
(f(g(x)))' - f'(g(x))g'(x) \rightsquigarrow \\
\int f'(g(x)) \times g'(x)dx = f((g(x)) + c \quad c \in \mathbb{R} 
\end{align*}
Die Ableitung der zu Substituierende Funktion $g(x)$ steht als \textbf{Faktor} im Integration
$$ \int{ f(g(x)) \times g(x) dx } $$
man vereinfache $g(x)$ durch : $z : = g(x)$
\begin{gather*}
\frac{dz}{dx} = g'(x) \Rightarrow dz = g'(x)dx \Rightarrow \\
\int{f(g(x)) \times g'(x)dx} = \int f(z)dz  = f(z) + c = f(g(x)) + c 
\end{gather*}
\begin{example}
Sub : $z = sinx$
\begin{align*}
\int {e^{sinx} cos} dx = \int e^z dz\\
\frac{dz}{dx} = cos x \Rightarrow dz = cos dx\\
 e^z + c = e^{sin x } + c, \quad c \in \mathbb{R}
\end{align*}
\textbf{probe}
$$ (e^{sinx} + c)' = (e^{sinx})' = c' = e^{sinx} cosx + 0 $$
\end{example}
\begin{example}
Sub : $z = lnx $
\begin{gather*}
\int \dfrac{dx}{ x (1+(lnx)^2)}\\
\frac{dz}{dx} = \frac{1}{x} \Rightarrow dz = \frac{1}{x} dx
\\
= \int \frac{dz}{ 1 + z^2} = arctan(z) + c = arctan(ln x )+ c \quad c \in \mathbb{R}
\end{gather*}
\textbf{Regel:}
$$ \int f(ax+b)dx $$
Sub : $z = ax+b $ , $ \frac{dz}{dx} = a $
\begin{align*}
&= \frac{1}{a} \int a f(ax+b) dx \\
&= \frac{1}{a} \int f(z)dz = \frac{1}{a} F(z) + c \\
&=\frac{1}{a} F(ax +b) + c \quad c \in \mathbb{R}
\end{align*}
\end{example}
\subsection{Produktregel $\rightsquigarrow$  \textbf{Partielle Integration}}
\begin{gather*}
(u(x) \times v(x))' = u'(x) v(x) + u(x) \times v'(x)\\
(uv)' = u'v + uv' \Rightarrow\\
\int(uv)'dx = \int u'v dx + \int uv' dx\\
\end{gather*}
\textbf{d.h}
\[ \int u'v dx = u v - \int u v' dx \]
\[ \int uv' dx = u v - \int u' v dx \]
\begin{example}
\begin{align*}
\int \underbrace{x}_{u} \underbrace{cosx}_{v'} dx 
\end{align*}
\textbf{Sub:}\\
$u  := x \Rightarrow u' = 1$\\
$v' := cos x \Rightarrow v = sinx $
\begin{align*}
&= x sinx - \int 1 \times sinx dx \\
&= x sinx - (-cos x) + c \quad c \in \mathbb{R}\\
&= x sinx + cos x + c
\end{align*}
\end{example}

\begin{example}
\begin{align*}
\int \underbrace{sinx}_{u} \underbrace{cosx}_{v'} dx 
\end{align*}
\textbf{Sub:}\\
$u  := sinx \Rightarrow u' = cosx$\\
$v' := cos x \Rightarrow v = sinx $
\begin{align*}
= sinx \times sinx - \int cos x \times sinx dx 
\end{align*}
Diese Partielle Integration führt auf das Ausgangs integral zurück
\begin{gather*}
2 \int \dots \dots = (sinx)^2 + \tilde{c} , \quad c \in \mathbb{R} | : 2\\
\Rightarrow \int sinx \times cos x dx = \frac{1}{2} (sin x )^2 + c , \quad c \in \mathbb{R} \frac{\tilde{c}}{2} = c
\end{gather*} 
\end{example}
\begin{example}
\begin{align*}
\frac{2x +1}{(x-1)(x+2)}dx &= \int \frac{1}{x-1}dx + \int \frac{1}{x+2}dx\\
\frac{2x+1}{(x-1)(x+2)} &= \frac{A}{x-1}+ \frac{B}{x-2} | \text{ gesucht A und B  }\\
2x+1 &= \underbrace{ A(x+2) + B(x-2)}_{(A+B)x+(2A-B)}  
\end{align*}
\subsection{Koeffizienten Regel}
? = A + B LGs A = 1 \\
1 = 2 A - B lösen b = 1
\end{example}
