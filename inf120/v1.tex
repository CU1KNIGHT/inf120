%%
%% Author: mahmoud
%% 19/04/12
%%
\chapter{Folge und Reihen}
\section{Vorlesung 1}
\subsection{Folge}
\numberwithin{equation}{theorem}
\begin{definition}[Folgen]
    Ein folge ist eine Abbildung

    \[ f: \mathbb{N} \rightarrow \underbrace{\mathbf{M}}_{Menge} : \mathrm{n} \mapsto \underbrace{x_n}_{folgenglied} \]

\end{definition}
\begin{remark}

    \begin{align}	\mathbf{M} &= \mathbb{R} \quad \text{reelewert Folge} \notag \\
    \mathbf{M}&= \mathbb{C} \quad	\text{komplexwertig Folge}   \notag \\
    \mathbf{M}&= \mathbb{R}^n \quad \text{vertical Folge} \notag
    \end{align}




\end{remark}
\begin{description}

    \item[Bezeichnung]

    \quad $(x_n)$ \space \text{mit} \space $ \left( x_n   \right)$=$ \frac{n}{n+1} $
    \\ \\ Aufzählung der folglieder: 0 , $\frac{1}{2}$ ,$\frac{2}{3}$ , $\frac{3}{4}$ , \dots

\end{description}
\begin{remark}
    zuwerten wird $\mathbb{N}$ durch $\mathbb{N}$ {0,1 \dots} erstellt.


\end{remark}
\begin{tikzpicture}
    \begin{axis}[
    ymin = -1,
    ymax = 1,
    xmin = -1,
    xmax = 5,
    axis x line=center,
    axis y line=center]
    \addplot[samples at={1.5,...,4.5},only marks,mark size=1] { x/(x+1)};
    \end{axis}
\end{tikzpicture}
\begin{example}
    \[\]
    \begin{enumerate}

        \item Konstante Folge $(x_n)$ mit \quad $x_n = a \in \mathbf{M },a \dots$ \\
        \[ x_n = a \in \mathbf{M} \]
        \item Harmonische Folge $(x_n)$  mit $x_n$ =  $\frac{1}{n+1}$ \quad$ n \geq 1$
        \item Geometrische folge $(x_n)$ mit $x_n = q^n \:, \: q \in \mathbb{R}, \dots $
        \item Fibonaccifolge $(x_n)$ mit
        \[ x_n =\frac{1}{\sqrt{5}} \Big(  \big( \frac{1+ \sqrt{5}}{2} \big)^n - \big( \frac{1- \sqrt{5}}{2}\big)^n   \Big)    \]

        \item Fibonacci folgen $(x_n)$
        \begin{align}
            X_0 &=0 \notag \\X_1 &= 1 \notag \\
            X_{n+1} &= x_n+X_{n-1} \quad (n>0)  \notag
        \end{align}

        \item conway  folge
        \[ 1, 11 ,21 , 1211, 111217, 312211 \dots \]

        \item folge aller Primzahlen: \[ 2, 3 ,5 ,7 ,11, 13 , \dots \]

    \end{enumerate}
\end{example}

\section{Rechnen mit Folgen }
\begin{align*}
    ( M  = \mathbb{R} \quad & oder \quad M = \mathbb{C} ) \\
    (x_n)+(y_n) &:= (x_n+y_n)\\
    K(x_n)&:=(Kx_n)\in \mathbb{R} \quad  oder \quad \in  \mathbb{C}
\end{align*}
\begin{remark}

    Die Folge bildet ein Vektorraum.
\end{remark}
\newpage

\begin{definition}[Beschränktheit]$ \newline$
        \begin{enumerate}

        \item Eine reellwertige Funktion ist in der Mathematik eine Funktion, deren Funktionswerte reelle Zahlen sind.

        \item Eine reellwertige heißt beschränkt wenn gilt

        \[	\exists r \in \mathbb{R}_+ , \forall r \in \mathbb{N}: \underbrace{|x_n|}_{\mathclap{\text{Betrag einer reellen oder komplexer Zahl}}} \leqq r   \]

    \end{enumerate}
\end{definition}

\begin{example}
    \[(x_n)\quad mit \quad x_n = (-1)^n \times \frac{1}{n} \]
    \[-1 ,\quad \frac{1}{2}, \quad \frac{-1}{3}, \quad \frac{1}{4} ,\quad \frac{-1}{5},\dots \]
\end{example}



\begin{tikzpicture}
    \begin{axis}[
    ymin = -2,
    ymax = 2,
    xmin = -3,
    xmax = 3,
    axis x line=center,
    axis y line=center]
        \addplot[dashed, name path =A] coordinates {(-3,1) (3,1)};
        \addplot[samples at={-3,...,3},only marks,mark size=1]{(-1)^x*(1/x)};
    \end{axis}
\end{tikzpicture}
\begin{remark}

    $(x_n)$ ist beschränkt mit $r = 1$ denn $|(-1)^n \frac{1}{n}|=|\frac{1}{n}| \leqq 1 \hookleftarrow r $

\end{remark}

\newpage
\begin{example}
    \[  (x_n) \text{ mit  } x_n = (-1)^n \quad \frac{1}{n}+1 \quad \text{bechränkt r = 3/2}\]

    \[ -3/2 \quad \leq x_n \leq 3/2 \quad \forall n \in \mathbb{N} \]
    %
    \begin{tikzpicture}
        \begin{axis}[
        ymin = -2,
        ymax = 2,
        xmin = -4,
        xmax = 4,
        axis x line=center,
        axis y line=center]
            \addplot[dashed, name path =A] coordinates {(-4,3/2) (4,3/2)};
            \addplot[dashed,samples at={0,...,3},only marks,mark size=1][ domain=(0):(2)] {log10(3*x + 1)};
        \end{axis}
    \end{tikzpicture}
\end{example}

\begin{example}{Standard:}\\

{Die folge}
$ \bigg(\big(1 + \frac{1}{n} \big)^n \bigg)^\infty_{n=1}$
{ist beschränkt durch 3}\\

Zu zeigen: \quad $ -3 \leq x_n \leq 3 \quad \text{für alle} \quad n \in \mathbb{N} $


\[ { (a+b)^n = \sum_{k=0}^{n} \binom{n}{k} a^k . b^{n .k} = \sum_{k=0}^{n} \binom{n}{k} a^{n.k} b^k }  \]
\[  \binom{n}{k} = \frac{n!}{k!(n-k!)} =  \frac{n(n-1) -(n-k-1))}{k!} \]
\[  \sum_{K=1}^{n} \frac{1}{k} = 1+ \frac{1}{2} + \frac{1}{2.3} + \frac{1}{2.3.4} + \dots \]
\end{example}


\section{geometrische Summen Formel (Tafelwerk)}
\begin{definition}[Monoton]

    Die Folge $(x_n)$ heißt monoton $\Big\{ \text{wachsend fallend} \Big\}$
    \[ wenn \quad gilt: \forall n \in \mathbb{N}:
    \left\{
    \begin{array}{ll}
        x_n  & \leq x_n +1 \\
        x_n  & \geq X_{n+1}
    \end{array}
    \right. \]
    \text{man spricht von Streng monotonie}
    $ wenn \leqq durch > und \geqq durch < \dots  $
\end{definition}
\begin{remark}
    \[ x_n \leq X_{n+1} \ \Leftrightarrow  x_n - X_{n+1} \leq 0 \quad \Leftrightarrow  \frac{x_n}{X_{n+1}} \leq 1 \]
\end{remark}

\begin{example}


    \[	(x_n) \text{ mit }\quad X_0  := 1 \quad, X_{n+1}   := \sqrt{x_n +6} \]
    \text{ist Streng monoton wachsend Beweis mit Vollständiger Induktion}
    \paragraph{Standard Bsp:}
    $ \big( \big(1+ \frac{1}{n} \big)^n \big) $ ist streng monoton wachsend
\end{example}
\begin{remark}

    \[
        \begin{tabular}{|c| c c |}
            \hline
            monoton & ja & nein  \\
            \hline
            \rule{0pt}{3ex}
            Beschränkkeit & $(\frac{1}{n})$ & $(-1)^n$ \\
            nein & (n) & $(-1)^n$ \\
            \hline
        \end{tabular}
    \]

\end{remark}
\begin{definition}[Konvergenz,Divergenz]
    $(x_n)$ heißt $\boldsymbol{Konvergenz}$ wenn $(x_n)$ ein grenzwert hat.\\
    $(x_n)$ heißt $\boldsymbol{Divergenz}$ wenn sie keinen grenzwert hat.
\end{definition}
\begin{definition}[grenzwert]
 $ a \in$ $\mathbb{R}$ heißt grenzwert von $(x_n)$, wenn gilt:
\[ \underbrace{\forall \epsilon > 0 }_{beliebes \; klein} \quad \underbrace{\exists \mathbf{N} \in \mathbb{N}}_{beliebes \;  klein \; \underbrace{\Rightarrow |x_n -a|< \varepsilon}_{a- \varepsilon \leq x_n \leq a+\varepsilon } } , \forall n \in \mathbb{N} : m \geq \mathbb{N}  \]
\[ \text{Sei  } \varepsilon > 0 ; \varepsilon \text{  fest} \]
\[ \text{alle folglieder$ x_n$ mit n } \geq \mathbb{N} \curvearrowright \]

\begin{tikzpicture}[scale=1.1],

\begin{axis}[
height=8cm,
width=10cm,
ymax=2,
ymin=-2,
xmin=1,
xmax=5.5,
axis y line=left,
axis x line=bottom,
yticklabels={,, $ a - \varepsilon$, a , $a + \varepsilon$ },xtick={1,...,10},
xticklabels={,,N,n},
]

    \addplot[dashed, name path =A] coordinates {(0,1) (5.5,1)};


    \addplot[thick, samples=50, smooth,domain=0:6,magenta, name path=V] coordinates {(3,-2)(3,3)};
    \addplot[dashed, name path =B] coordinates {(0,-1) (5.5,-1)};
    \addplot[gray, pattern=north west lines] fill between[of=A and B, soft clip={domain=3:6}];
    \addplot[samples at={0,...,5},only marks,mark size=1] { 3/x -1/3};
    \addplot[samples at={0,...,5},only marks,mark size=1] { ln(x)-2 };
   



\end{axis}
\end{tikzpicture}%

\end{definition}
