% Mathematische Methoden für Informatiker: Institut %Algebra Algebra
% TU Dresden SS19
% Prof Dr Ulrike Baumann

% Setup document
\documentclass[a4paper,12pt]{report}

\usepackage{inputenc,fontenc}
\usepackage{longtable,tabularx,tabulary,array,booktabs,multicol}
\usepackage{graphicx,caption,color,xcolor}
\usepackage{chngcntr}
\usepackage{verbatim}
\usepackage{url}
\usepackage{amsmath,amsthm,amssymb,amsfonts}
\usepackage{amsthm}
\usepackage[nottoc]{tocbibind}
\usepackage{mathtools}
\usepackage{mathabx}
\usepackage{thmtools}
\usepackage[a4paper,margin=3cm]{geometry}
\usepackage{pgfplots}
\usepackage[english,german]{babel}
\usepackage{hyperref}
\usepackage{MnSymbol} %sch\"{o}nere Mathe-fonts
\usepackage{eucal}    %sch\"{o}nere Skript-Buchstaben
\usepackage{dsfont}   %sch\"{o}nere "Doppelstrich"-Fonts
\usepackage{tikz}
\usepackage{forest}



% Fontauswahl

\pgfplotsset{compat=1.12}

\usepgfplotslibrary{fillbetween}
\usetikzlibrary{patterns}

\newcommand{\mathscr}{\mathcal}
%\usepackage{picinpar} %Einbinden von Bildern
%\usepackage{picins}   %Einbinden von Bildern

% Define oftenly used commands

\newcommand{\nameE}{Mohamed}
\newcommand{\vornameE}{Abdelshafi}
\newcommand{\emailE}{m.abdelshafi@mail.de}
\newcommand{\nameS}{Kiki}
\newcommand{\vornameS}{Mahmoud}
\newcommand{\emailS}{mahmoud.kiki@tu-dresden.de}



%% Created defnintions 
\def\checkmark{\tikz\fill[scale=0.4](0,.35) -- (.25,0) -- (1,.7) -- (.25,.15) -- cycle;} 




%% Theorem environments etc.
%% See also: ftp://ftp.ams.org/ams/doc/amscls/amsthdoc.pdf
%%
%created theorem
\swapnumbers
\theoremstyle{plain} %Text ist Kursiv
\newtheoremstyle{break}% name
{}%         Space above, empty = `usual value'
{}%         Space below
{\itshape}% Body font
{}%         Indent amount (empty = no indent, \parindent = para indent)
{\bfseries}% Thm head font
{.}%        Punctuation after thm head
{\newline}% Space after thm head: \newline = linebreak
{}%         Thm head spec
\newtheorem{theorem}{Satz}[chapter]
\newtheorem{lemma}[theorem]{Lemma}
\newtheorem{proposition}[theorem]{Proposition}
\newtheorem{corollary}[theorem]{Korollar}
%\theoremstyle{definition} %Text ist \"upright"
\theoremstyle{break}
\newtheorem{remark}[theorem]{Bemerkung}
\newtheorem{definition}[theorem]{Definition}
\newtheorem{example}[theorem]{Beispiel}
\newtheorem{beweis}[theorem]{Beweis}
\newtheorem{schreibweise}[theorem]{Schreibweise}
\newtheorem{rechnen}[theorem]{Berechnung}



\hypersetup{
	colorlinks=true,
	linkcolor=blue,
	filecolor=magenta,
	urlcolor=cyan}


{\author{ \vornameE \nameE \vornameS \nameS}}
{\title{Mathematische Methoden für Informatiker}}







\newcommand{\institut}{Institut Ihres Betreuers / Ihrer Betreuerin}

\newcommand{\thema}{Mathematische Methoden für Informatiker}

\newcommand{\datum}{\today}%Format tt.\ mm.\ jjjj

%shortcut :Mahmoud
\newcommand{\sumIn}{\sum_{K=1}^{n}}
\newcommand{\sumIin}{\sum_{K=1}^{\infty}}
\newcommand{\sumOin}{\sum_{K=0}^{\infty}}
\newcommand{\sumOn}{\sum_{K=0}^{n}}
\newcommand{\fracKetwo}{\frac{1}{k^2}} %1/(k^2)
\newcommand{\fracKone}{\frac{1}{k}}   %1/k
\newcommand{\limNin}{\lim\limits_{n \rightarrow \infty}}
\newcommand{\limKin}{\lim\limits_{k \rightarrow \infty}}


%created defintions 
\newcommand*\circled[1]{\tikz[baseline=(char.base)]{
    \node[shape=circle,draw,inner sep=2pt] (char) {#1};}}

\begin{document}
	\selectlanguage{german}
	%\selectlanguage{english} %Entferne "%", wenn Sprache Englisch ist



	%% Titelseite

	\thispagestyle{empty}


    \begin{center}
    {\Large
    Technische Universit\"{a}t Dresden\  \ \textbullet\ \ Fakult\"{a}t Informatik
    }

        \vfil

        {\bfseries\Huge\thema}

        \vfil

        {\LARGE
Mitschrift zur Vorlesung Sommer Semester 2019  \\[\bigskipamount]

        \bfseries{\itshape Bachelor of Science  \textup{(}B.Sc.\textup{)}}\\[\bigskipamount]
        }% Ende Large

        \vfil\vfil\vfil
        Dozent: Prof. Dr. Ulrike Baumann \\
        vorgelegt von\\
        \item "..." \\
        \item \textsc{\vornameE\ \nameE } \\ \texttt{\emailE} \\  \item
        \textsc{\vornameS\ \nameS \qquad } \\ \texttt{\emailS}  \\
        \item ... \\
        Tag der Einreichung: \datum\\[\bigskipamount]

    \end{center}


    \cleardoublepage



    \tableofcontents

    \thispagestyle{empty}

    %\makeatletter
    %\begin{titlepage}

    %	The title is \@title
    %	It was written by \@author\space on \@date

    %\end{titlepage}
    \setcounter{page}{0}
    \chapter*{Einleitung}
    Wir schreiben hier die vorlesungen von INF-120-1( Mathematische Methoden für Informatiker) mit.
    wenn Ihr Fragen habt oder Fehlern gefunden Sie können gerne uns eine E-mail schreiben oder Sie können einfach bei github eine  \href{https://github.com/CU1KNIGHT/inf120/issues}{Issue (link)} erstellen.
    wir freuen uns wenn Sie mit uns mitschreiben möchten, oder helfen mit der Fehlerbehebung.
    \newline \newline
    \vornameE \ \nameE \newline
    \vornameS  \ \nameS
    
    
%%
%% Author: mahmoud
%% 19/04/12
%%

\chapter{Folge und Reihen}
\section{Folgen}
\numberwithin{equation}{theorem}
\begin{definition}[Folgen]
    Ein folge ist eine Abbildung

    \[ f: \mathbb{N} \rightarrow \underbrace{\mathbf{M}}_{Menge} : \mathrm{n} \mapsto \underbrace{X_n}_{folgenglied} \]

\end{definition}
\begin{remark}

    \begin{align}	\mathbf{M} &= \mathbb{R} \quad \text{reelewert Folge} \notag \\
    \mathbf{M}&= \mathbb{C} \quad	\text{komplexwertig Folge}   \notag \\
    \mathbf{M}&= \mathbb{R}^n \quad \text{vertical Folge} \notag
    \end{align}




\end{remark}
\begin{description}

    \item[Bezeichnung]

    \quad $(X_n)$ \space \text{mit} \space $ \left( X_n   \right)$=$ \frac{n}{n+1} $
    \\ \\ Aufzählung der folglieder: 0 , $\frac{1}{2}$ ,$\frac{2}{3}$ , $\frac{3}{4}$ , \dots

\end{description}
\begin{remark}
    zuwerten wird $\mathbb{N}$ durch $\mathbb{N}$ {0,1 \dots} erstellt.


\end{remark}
\begin{tikzpicture}
    \begin{axis}[
    ymin = -1,
    ymax = 1,
    xmin = -1,
    xmax = 5,
    axis x line=center,
    axis y line=center]
    \addplot[samples at={1.5,...,4.5},only marks,mark size=1] { x/(x+1)};
    \end{axis}
\end{tikzpicture}
\begin{example}
    \[\]
    \begin{enumerate}

        \item Konstante Folge $(X_n)$ mit \quad $X_n = a \in \mathbf{M },a \dots$ \\
        \[ X_n = a \in \mathbf{M} \]
        \item Harmonische Folge $(X_n)$  mit $X_n$ =  $\frac{1}{n+1}$ \quad$ n \geq 1$
        \item Geometrische folge $(X_n)$ mit $X_n = q^n \:, \: q \in \mathbb{R}, \dots $
        \item Fibonaccifolge $(X_n)$ mit
        \[ X_n =\frac{1}{\sqrt{5}} \Big(  \big( \frac{1+ \sqrt{5}}{2} \big)^n - \big( \frac{1- \sqrt{5}}{2}\big)^n   \Big)    \]

        \item Fibonacci folgen $(X_n)$
        \begin{align}
            X_0 &=0 \notag \\X_1 &= 1 \notag \\
            X_n+1 &= X_n+X_{n-1} \quad (n>0)  \notag
        \end{align}

        \item conway  folge
        \[ 1, 11 ,21 , 1211, 111217, 312211 \dots \]

        \item folge aller Primzahlen: \[ 2, 3 ,5 ,7 ,11, 13 , \dots \]

    \end{enumerate}
\end{example}

\section{Rechnen mit Folgen }
\begin{align*}
    ( M  = \mathbb{R} \quad & oder \quad M = \mathbb{C} ) \\
    (X_n)+(y_n) &:= (X_n+y_n)\\
    K(X_n)&:=(KX_n)\in \mathbb{R} \quad  oder \quad \in  \mathbb{C}
\end{align*}
\begin{remark}

    Die Folge bildet ein Vektorraum.
\end{remark}
\newpage

\begin{definition}$ \newline$
        \begin{enumerate}

        \item Eine reellwertige Funktion ist in der Mathematik eine Funktion, deren Funktionswerte reelle Zahlen sind.

        \item Eine reellwertige heißt beschränkt wenn gilt

        \[	\exists r \in \mathbb{R}_+ , \forall r \in \mathbb{N}: \underbrace{|X_n|}_{\mathclap{\text{Betrag einer reellen oder komplexer Zahl}}} \leqq r   \]

    \end{enumerate}
\end{definition}

\begin{example}
    \[(X_n)\quad mit \quad X_n = (-1)^n \times \frac{1}{n} \]
    \[-1 ,\quad \frac{1}{2}, \quad \frac{-1}{3}, \quad \frac{1}{4} ,\quad \frac{-1}{5},\dots \]
\end{example}



\begin{tikzpicture}
    \begin{axis}[
    ymin = -2,
    ymax = 2,
    xmin = -3,
    xmax = 3,
    axis x line=center,
    axis y line=center]
        \addplot[dashed, name path =A] coordinates {(-3,1) (3,1)};
        \addplot[samples at={-3,...,3},only marks,mark size=1]{(-1)^x*(1/x)};
    \end{axis}
\end{tikzpicture}
\begin{remark}

    $(X_n)$ ist beschränkt mit $r = 1$ denn $|(-1)^n \frac{1}{n}|=|\frac{1}{n}| \leqq 1 \hookleftarrow r $

\end{remark}
\begin{example}
    \[  (X_n) \text{ mit  } X_n = (-1)^n \quad \frac{1}{n}+1 \quad \text{bechränkt r = 3/2}\]

    \[ -3/2 \quad \leq X_n \leq 3/2 \quad \forall n \in \mathbb{N} \]
    %
    \begin{tikzpicture}
        \begin{axis}[
        ymin = -2,
        ymax = 2,
        xmin = -4,
        xmax = 4,
        axis x line=center,
        axis y line=center]
            \addplot[dashed, name path =A] coordinates {(-4,3/2) (4,3/2)};
            \addplot[dashed,samples at={0,...,3},only marks,mark size=1][ domain=(0):(2)] {log10(3*x + 1)};
        \end{axis}
    \end{tikzpicture}
\end{example}

\begin{example}{Standard:}\\

{Die folge}
$ \bigg(\big(1 + \frac{1}{n} \big)^n \bigg)^\infty_{n=1}$
{ist beschränkt durch 3}\\

Zu zeigen: \quad $ -3 \leq X_n \leq 3 \quad \text{für alle} \quad n \in \mathbb{N} $


\[ { (a+b)^n = \sum_{k=0}^{n} \binom{n}{k} a^k . b^{n .k} = \sum_{k=0}^{n} \binom{n}{k} a^{n.k} b^k }  \]
\[  \binom{n}{k} = \frac{n!}{k!(n-k!)} =  \frac{n(n-1) -(n-k-1))}{k!} \]
\[  \sum_{K=1}^{n} \frac{1}{k} = 1+ \frac{1}{2} + \frac{1}{2.3} + \frac{1}{2.3.4} + \dots \]
\end{example}


\section{geometrische Summen Formel (Tafelwerk)}
\begin{definition}

    Die Folge $(X_n)$ heißt monoton $\Big\{ \text{wachsend fallend} \Big\}$
    \[ wenn \quad gilt: \forall n \in \mathbb{N}:
    \left\{
    \begin{array}{ll}
        X_n  & \leq X_n +1 \\
        X_n  & \geq X_n+1
    \end{array}
    \right. \]
    \text{man spricht von Streng monotonie}
    $ wenn \leqq durch > und \geqq durch < \dots  $
\end{definition}
\begin{remark}
    \[ X_n \leq X_{n+1} \ \Leftrightarrow  X_n - X_{n+1} \leq 0 \quad \Leftrightarrow  \frac{X_n}{X_{n+1}} \leq 1 \]
\end{remark}

\begin{example}


    \[	(X_n) \text{ mit }\quad X_0  := 1 \quad, X_{n+1}   := \sqrt{X_n +6} \]
    \text{ist Streng monoton wachsend Beweis mit Vollständiger Induktion}
    \paragraph{Standard Bsp:}
    $ \big( \big(1+ \frac{1}{n} \big)^n \big) $ ist streng monoton wachsend
\end{example}
\begin{remark}

    \[
        \begin{tabular}{|c| c c |}
            \hline
            monoton & ja & nein  \\
            \hline
            \rule{0pt}{3ex}
            Beschränkkeit & $(\frac{1}{n})$ & $(-1)^n$ \\
            nein & (n) & $(-1)^n$ \\
            \hline
        \end{tabular}
    \]

\end{remark}
\begin{definition}
    $(X_n)$ heißt $\boldsymbol{Konvergenz}$ wenn $(X_n)$ ein grenzwert hat.\\
    $(X_n)$ heißt $\boldsymbol{Divergenz}$ wenn sie keinen grenzwert hat.
\end{definition}
\begin{definition}[grenzwert]
 $ a \in$ $\mathbb{R}$ heißt grenzwert von $(X_n)$, wenn gilt:
\[ \underbrace{\forall \epsilon > 0 }_{beliebes \; klein} \quad \underbrace{\exists \mathbf{N} \in \mathbb{N}}_{beliebes \;  klein \; \underbrace{\Rightarrow |X_n -a|< \varepsilon}_{a- \varepsilon \leq X_n \leq a+\varepsilon } } , \forall n \in \mathbb{N} : n \geq \mathbb{N}  \]
\[ \text{Sei  } \varepsilon > 0 ; \varepsilon \text{  fest} \]
\[ \text{alle folglieder$ X_n$ mit n } \geq \mathbb{N} \curvearrowright \]
\begin{tikzpicture}[scale=1.1],

\begin{axis}[
height=8cm,
width=10cm,
ymax=2,
ymin=-2,
xmin=1,
xmax=5.5,
axis y line=left,
axis x line=bottom,
yticklabels={,, $ a - \varepsilon$, a , $a + \varepsilon$ },xtick={1,...,10},
xticklabels={,,N,n},
]

    \addplot[dashed, name path =A] coordinates {(0,1) (5.5,1)};


    \addplot[thick, samples=50, smooth,domain=0:6,magenta, name path=V] coordinates {(3,-2)(3,3)};
    \addplot[dashed, name path =B] coordinates {(0,-1) (5.5,-1)};
    \addplot[gray, pattern=north west lines] fill between[of=A and B, soft clip={domain=3:6}];
    \addplot[samples at={0,...,5},only marks,mark size=1] { 1/x };
    \addplot[samples at={0,...,5},only marks,mark size=1] { (1-x)/x^2 };
   



\end{axis}
\end{tikzpicture}%


\end{definition}

\begin{text}
    ist die folge beschränkt , monoton ?\\

    $(X_n)$ konvergierend : $\iff \exists a \in\mathbb{R} \quad \forall \epsilon > 0 \quad \exists n \quad \in N \quad \forall n \in N \quad \\
    n \geq N \Rightarrow |X_n - a |< \epsilon $
\end{text}

\begin{theorem}

    $(X_n)$ konvergierend : $\Rightarrow$ Der Grenzwert ist eindeutig beschränkt.

\end{theorem}

\begin{proof}
    Sei a eine Grenzwert von $(X_n)$ , b eine Grenzwert von $(X_n)$ \\
    d.h sei $\epsilon > 0$,$\epsilon$ beliebig , $\epsilon$ fest \\

    \begin{equation}
        \exists  N_a \quad \forall n \geq N_a : |X_n-a|< \epsilon
    \end{equation}

    \begin{equation}
        \exists  N_b \quad \forall n \geq N_b : |X_n-b|< \epsilon
    \end{equation}

    Sei max ${N_a,N_b}=N$
    dann gilt : \\
    \begin{equation}
        n \geq N \Rightarrow |X_n - a| < \epsilon
    \end{equation}
    und \begin{equation}
            |X_n -b| < \epsilon \Rightarrow |X_n -a|+|X_n - b|< 2\epsilon
    \end{equation}\\

    Annahme :- a $\neq$ b , d.h $|a-b|\neq 0 $
    \[|a-b|=|a+0-b|
    =|(a-X_n)+(X_n-b)| \leq |X_n - a|+|X_n-b|< 2 \epsilon \]
    also $|a - b|< 2 \epsilon$


    \begin{example}
        \[\epsilon = \frac{|a-b|}\epsilon
        \quad \text{dann gilt}\ :|a-b|<2 \frac{|a-b|}{3}\]\\

        \[ \Rightarrow 1 < \frac{2}{3} \quad falls \quad Aussage, Widerspruch \quad also \quad ist \quad die \quad Annahme \quad falsch \quad also \quad gilt \quad a=b\]

    \end{example}
\end{proof}

\begin{example}

    $X_n$ mit $X_n = \frac{1}{n}$ (harmonische Folge)

\end{example}

\begin{proof}
    Sei $\epsilon > 0 , \epsilon belibig , \epsilon fest$
    gesucht : N mit $n \geq$ N

    \begin{gather}
        \Rightarrow |X_n-a|= |\frac{1}{n} =0|=\frac{1}{n}<\epsilon
    \end{gather}

    wähle N:= $\lceil \frac{1}{\epsilon} \rceil +1$

\end{proof}

\begin{example}
    $\epsilon = \frac{1}{100}$ , gesucht N mit $n \geq N$
    $\Rightarrow \frac{1}{n} < \frac{1}{100}$ wähle $N=101$\\


    Schreibweise: $X_n$ hat den Grenzwert a Limes
    $\lim\limits_{n \rightarrow \infty}{x_n}=a$
    $X_n$ geht gegen a für n gegen Unendlich.
\end{example}

\begin{definition}
    $X_n$ heißt Nullfolge ,wenn $\lim\limits{X_n}=0$ gilt.
\end{definition}

\begin{remark}

    Es ist leichter, die konvergente einer Folge zu beweisen, als den Grenzwert auszurechnen.

\end{remark}

\begin{example}
    $X_n = \frac{1}{3} + \big(\frac{11-n}{9-n}\big)^9$\\


    Behauptung: $\lim\limits_{n \rightarrow \infty}{x_n}=\frac{-2}{3}$

    \begin{lemma}
        \begin{gather}
            \lim\limits_{n \rightarrow \infty}{x_n+y_n}=
            (\lim\limits_{n \rightarrow \infty}{x_n}) +
            (\lim\limits_{n \rightarrow \infty}{y_n})
        \end{gather}
    \end{lemma}

    \begin{gather}
        =\lim\limits_{n \rightarrow \infty}{\bigg(\big(\frac{1}{3}\big)+\bigg(\frac{11-n}{9+n}\bigg)^9\bigg)}
        = \lim\limits_{n \rightarrow \infty}{\frac{1}{3}+
        \lim\limits_{n \rightarrow \infty}{\bigg(\frac{11-n}{9+n}\bigg)^9}}
    \end{gather}

    \begin{gather}
        = \frac{1}{3} + \bigg(\lim\limits_{n \rightarrow \infty}{\frac{11-n}{9+n}}\bigg)^9
    \end{gather}


    \begin{gather}
        = \frac{1}{3} + \lim\limits_{n \rightarrow \infty}{\Bigg(\frac{n(\frac{1}{n}-1)}{n(\frac{9}{n}+1)}\Bigg)^9}
    \end{gather}

    \begin{gather}
        = \frac{1}{3}+\Bigg(\frac{\lim\limits_{n \rightarrow \infty}{(\frac{11}{n})}}{\lim\limits_{n \rightarrow \infty}{(\frac{9}{n}+1})}\Bigg)^9
    \end{gather}

    \begin{gather}
        = \frac{1}{3} + \Bigg(
        \frac{\lim\limits_{n \rightarrow \infty}{\frac{11}{n}} - \lim\limits_{n \rightarrow \infty}{1}}{\lim\limits_{n \rightarrow \infty}{\frac{9}{n}+\lim\limits_{n \rightarrow \infty}{1} } }   \Bigg)^9
    \end{gather}


    \begin{gather}
        =\Bigg(
        \frac{\lim\limits_{n \rightarrow \infty}{11} \times \lim\limits_{n \rightarrow \infty}{(\frac{1}{n}-1)}}{\lim\limits_{n \rightarrow \infty}{9 \times \lim\limits_{n \rightarrow \infty}{(\frac{1}{n}+1)} } }   \Bigg)^9
    \end{gather}

    \begin{gather}
        \frac{1}{3}+(-1)^9 = \frac{1}{3}-1 = \frac{-2}{3}
    \end{gather}
\end{example}

\begin{definition}
    Eine Folge $(X_n)$ hat den unendliche Grenzwert $\infty$, wenn gilt : \\
    \[\forall r \in \mathbb{R} \quad \exists N \in N \quad \forall n \geq N : X_n > r \]

    Schreibweise : $\lim\limits_{n \rightarrow \infty}{X_n}= \infty$
\end{definition}

\begin{remark}
    $\infty$ ist keine Grenzwerte und keine reelle Zahl.
\end{remark}

\begin{remark}
    Grenzwertsätze gelten nicht für uneigentliche Grenzwerte.
\end{remark}

\begin{remark}
    gilt $\lim\limits_{n \rightarrow \infty}{X_n}= \infty$ dann schreibt man $\lim\limits_{n \rightarrow \infty}{X_n}= -\infty$
\end{remark}

\begin{example}
    $X_n$ mit $X_n = q^n$ , $q \in \mathbb{R}$ , $q$ fest.\\

    $ \lim\limits_{n \rightarrow \infty}{q^n} = \begin{cases}
                                                    0 ,\quad |q|<1 \\
                                                    1 ,\quad |q|=1 \\
                                                    \infty ,\quad\quad q > 1  \\
                                                    ex. nicht ,\quad q\leq -1
    \end{cases}$
\end{example}

\vfil
\vfil


\section{Konvergenzkriterien}
(zum Beweis der Existenz eine Grenzwert, nicht zum berechnen von Grenzwert) \\


(1) $X_n$ konvergent $\Rightarrow$ $(X_n)$ beschränkt. \\

wenn $(X_n)$ nicht beschränkt $\Rightarrow$ $(X_n)$ nicht konvergent.\\


(2) Monotonie Kriterium:
wenn $(X_n)$ beschränkt ist können wir fragen ob $(X_n)$    konvergent.\\


$(X_n)$ beschränkt von Monotonie $\Rightarrow$ $(X_n)$ konvergent.

\begin{remark}
    % noch zu schreiben
\end{remark}

\begin{example}

\end{example}
\begin{equation}
    \lim\limits_{n \rightarrow \infty}{\frac{11+1}{9-n}}\quad ? \\
    \\\quad X_n = \frac{11+1}{9-n}=\frac{n}{n} \frac{\frac{11}{n}+1}{\frac{9}{n}-1}
\end{equation}

\begin{equation}
    \lim\limits_{n \rightarrow \infty}{\bigg(\frac{11}{n}+1\bigg)}=1
\end{equation}

\begin{equation}
    \lim\limits_{n \rightarrow \infty}{\bigg(\frac{9}{n}+1\bigg)}=-1
\end{equation}

\begin{equation}
    \lim\limits_{n \rightarrow \infty}{(X_n)}= \frac{1}{-1}=-1
\end{equation}

\begin{lemma}
    Seien $(x_n)=(y_n)$ Folgen auf $\lim\limits_{n \rightarrow \infty}{(x_n)}= \lim\limits_{n \rightarrow \infty}{(y_n)}= a$ und es gelte
    $x_n \leq z_n \leq y_n$ für "fest alle " $n \in \mathbb{N}$\\

    Dann gilt für die Folge $(Z_n) \lim\limits_{n \rightarrow \infty}{(z_n)}=a$
\end{lemma}

\begin{example}
    Ist die Folge $(-1)^n\frac{1}{n})$ konvergent ?\\

    \[ - \frac{1}{n} \leq(-1)^n(\frac{1}{n}) \leq 1 \frac{1}{n}\]

    \[ \lim\limits_{n \rightarrow \infty}{- \big(\frac{1}{n} \big)}= -1 \]
    \[ \lim\limits_{n \rightarrow \infty}{ \big(\frac{1}{n} \big)}= 0 \Rightarrow \lim\limits_{n \rightarrow \infty}{(-1)^n \frac{1}{n}}= 0
    \]
\end{example}

\newpage

\begin{example}
    \begin{equation}
        \begin{aligned}
            x_n \leq  = \frac{a^n}{n!} = \frac{a}{n} \times \frac{a^{a-1}}{n-1!} %
        \end{aligned}
    \end{equation}\\

    denn $ x_n = 0 \leq \frac{a_n}{n!} \leq y_n$
    , gesucht! $\underbrace{y_n}_{\lim\limits_{n \rightarrow \infty}{y_n}=0}$  für hinreichend großes n.

    \begin{equation}
        \begin{aligned}
            \frac{a^n}{n!} = \frac{a}{n} \times \frac{a^{n-1}}{(n-1)!} \\ \leq
            \frac{1}{2} \times
            \frac{a^{n-1}}{(n-1)!} \\ =
            \frac{1}{2} \times
            \frac{a}{(n-1)} \times
            \frac{a^{n-2}}{(n-2)!} \\ \leq
            \frac{1}{2} \times
            \frac{1}{2} \times
            \frac{a^{n-2}}{(n-2)!} \\ \leq
            \frac{1}{2} \times
            \frac{1}{2} \times
            \frac{1}{2} \times
            \frac{a^{n-3}}{(n-3)!}\\
            %
            y_n = (\frac{1}{2})^{n-k} \times \frac{a^k}{k!} \quad \text{k ist fest}
        \end{aligned}
    \end{equation}\\

    {Es gilt} $\frac{a^n}{n!} \leq y_n$  für hinreichend großes n und
    $\lim\limits_{n \rightarrow \infty}{(y_n)}$ \\

    \begin{equation}
        \begin{aligned}
            &=
            \lim\limits_{n \rightarrow \infty}{(\frac{1}{2})^{n-k}} \times
            \underbrace{\frac{a^l}{k!}}_{Konst} \\
            &=
            \lim\limits_{n \rightarrow \infty}{(\frac{1}{2})^{n}} \times
            \underbrace{\lim\limits_{n \rightarrow \infty}{(\frac{1}{2})^{-k}}}_{\in \mathbb{R}} \times
            \underbrace{\lim\limits_{n \rightarrow \infty}{(\frac{a^k}{k!})}}_{\in \mathbb{R}} \\
            &= 0 . (\frac{1}{2})^{-k} \times \frac{a^k}{k!}=0
            \\
        \end{aligned}
    \end{equation}
\end{example}



\newpage

\section{Grenzwerte rekursive definierte Folgen:}

man kann oft durch lösen "Fixpunktgleichung" berechnen.\\
$x_0 \quad , x_n+1 = ln(x_n)$

\begin{example}
\[(x_n) \quad x_0 = \frac{7}{5} \quad,\quad x_n+1= \frac{1}{3}(x_n^2+2)  \]

Ü $(x_n)$ ist monoton fallend , beschränkt , konvergent . 

\[\lim\limits_{n \rightarrow \infty}{x_n}=a \quad,\quad 
\lim\limits_{n \rightarrow \infty}{x_n+1}=a \]

\begin{equation*}
\begin{aligned}
\lim\limits_{n \rightarrow \infty}{x_n+1} 
= \linebreak  
\lim\limits_{n \rightarrow \infty}{\frac{1}{3}(x_n^2 + 2)
\frac{1}{3}} \lim\limits_{n \rightarrow \infty}{(x_n^2 + 2)}
=
\frac{1}{3} (\lim\limits_{n \rightarrow \infty}{(x_n))^2 + 2)}
\end{aligned}
\end{equation*}
\end{example}

\subsubsection{Fixpunktgleichung }
 $a = \frac{1}{3}(a^2 + 2) $  , gesucht = a
 
\[ 3a = a^2 +2 \Leftrightarrow a^2 -3a+2 = 0 \] \\
\[ \Leftrightarrow a_{1/2} = \frac{3}{2} \pm \sqrt{\frac{9}{4}-\frac{8}{4}}= \frac{3}{2} \pm \frac{1}{2}\]
Lösung:  $a_1 = 2$ (keine Lösung),  $a_2 =1 $

\begin{example}{$(x_n)$ mit $(x_0) = c \in \mathbb{R} , c  $ fest $x_{n+1}= \frac{1}{2}(x_n+\frac{c}{x_n})$ }\\
(1) $(x_n)$ beschränkt \checkmark\\
(2) $(x_n)$ Monoton \checkmark\\
Also $(x_n)$ konvergent \\
Sei $\lim\limits_{n \rightarrow \infty}{x_n}= a $. 
Dann $\underbrace{\lim\limits_{n \rightarrow \infty}{x_{n-1}}= }_{a}$ $\lim\limits_{n \rightarrow \infty}{\frac{1}{2}}(x_n) + \frac{c}{x_n} = \frac{1}{2}(a + \frac{a}{c})= a \\
 \Leftrightarrow 2a = a + \frac{c}{a} \Leftrightarrow a = \frac{c}{a } \Leftrightarrow a^2 = c \Leftrightarrow a = \sqrt{c}$
\end{example}

\begin{remark}
Der Nachweis der konvergent der rekursiv definierte Folge darf nicht weggelassen werden, denn Z.B $x_0=2$ , $x_n+1=x_n^2$ \quad \quad \quad 2 , 4 ,16 ,256 , $\dots $ divergent gegen + $\infty$  \\

Annahme: $\lim\limits_{n \rightarrow \infty}{x_n}= a $ 
$\underbrace{\quad \lim\limits_{n \rightarrow \infty}{x_{n+1}}}_{a}$ = 
$\underbrace{\lim\limits_{n \rightarrow \infty}{x_n^2}}_{a} \Rightarrow a \in \{ 0,1 \}$
\end{remark}  

%new 
 
\newpage
\section{Reihen :}
\begin{definition}
Sei $(a_n)$ eine reellefolge (komplexwertig) Folge\\
$$\sum_{k = 0}^{n} {a_k} = a_a , a_1, \dots , a_n , $$
heißt n-k heißt partielle Summe.
$(S_n)$ heißt unendliche Reihe.

schriebweise : $(S_n)^\infty =$ bsw 
$(S_n)$ $$ \bigg( \sum_{l=0}^{n} {a_l} \bigg)$$ bzw
 $$ \bigg( \sum_{l=0}^{\infty} {a_l} \bigg)$$  
\end{definition}

\begin{remark}
Reihen sind spezielle Folgen , alle konvergent oder divergent. 
\end{remark}

\begin{definition}
Für eine konvergente Reihen wird der Grenzwert auch wert der Reihe genannt.\\
Schreibweise :  $\lim\limits_{n \rightarrow \infty}{S_n}= $
$$\lim\limits_{n \rightarrow \infty}{ \sum_{k=0}^{n} {a_k} }  $$ 
bzw 
$$ \sum_{k=0}^{\infty} {a_k}  $$
\end{definition}
 
\begin{example}
geometrische Reihe $$ \sum_{k=0}^{\infty} {q^k} $$
ist für $|q|<1$ konvergent . wert der Reihe für $|q|<1$ : 
$$ \sum_{k=0}^{\infty} {q^k}= \frac{1}{1-q}$$ für $|q| < 1 $ 
\end{example}






\begin{example}
\begin{equation}
\begin{aligned}
0,4 \overline{3} = \frac{3}{4} + \frac{3}{100} + \frac{3}{10000}+ \dots \\
\frac{4}{10} + \frac{3}{100}(\frac{1}{10})^0 + \frac{1}{10} + \frac{3}{10^2} + \dots \\
=\frac{4}{10} + \frac{3}{100} \times \frac{1}{1-\frac{1}{10}}\\
= \frac{4}{10}+ \frac{1}{30} = \frac{12+1}{30}= \frac{13}{30}
\end{aligned}
\end{equation}
\end{example}

\begin{example}
\begin{equation}
\begin{aligned}
\sum_{K=1}^\infty{\frac{1}{k}} \text{ist divergent , denn }\\
\lim_{n \to \infty} \sum_{K=1}^n{\frac{1}{k}} \text{ex. nicht ! }\\
S_n = \frac{1}{1} + \frac{1}{2} + \big(\frac{1}{3} + \frac{1}{4} \big)+
\big(\frac{1}{5} + \frac{1}{6} + \frac{1}{7} + \frac{1}{8} \big) 
+ \frac{1}{9} \dots + \frac{1}{16} + \dots \frac{1}{n} \\
 > 1 + \frac{1}{2} \big(\frac{1}{4} + \frac{1}{4} \big) + 
 \big(\frac{1}{8}+ \frac{1}{8}+ \frac{1}{8}+ \frac{1}{8}+ \big)+ \\
 \big( \frac{1}{10} + \dots + \dots + \frac{1}{10} \big)+ \dots + \frac{1}{n}= 1+ \frac{1}{2} + \frac{1}{2} + \frac{1}{2} + \frac{1}{2}+ \dots + \frac{1}{n}\\
 \Rightarrow \lim_{n \to \infty}S_n=\infty
\end{aligned}
\end{equation}
\end{example}

\section{Alleemiene Reihen}
\begin{lemma}
\begin{equation}
\begin{aligned}
\sum_{k=1}^\infty {\frac{1}{k^x}} \text{x fest}\\
\text{falls:}\\
x > 1 \Rightarrow konvergent\\
x \leq q \Rightarrow Divergent\\
\end{aligned}
\end{equation}
\end{lemma}


\begin{proof}{mit Monotoniekriterium}
\begin{equation}
(1)\bigg(\sum_{k=1}^n {\frac{1}{k^2}} \bigg) \text{\quad Monoton (wachsend)} 
\end{equation}
%fellend
\begin{equation}
(2)(\bigg(\sum_{k=1}^n {\frac{1}{k^2}} \bigg)) \text{ \quad ist beschränkt}
\end{equation}  \\




\end{proof}
\section{Rechnenreglen für Regeln:}
Konvergenden Reige kann man addieren , subtrahieren, mit einem Skaler multiplizieren  wie endlichen Summen \\ \underline{ABER:} \\
Das gilt im Allgemein nicht für das multiplizieren
\section{Vorlesung 3}

\begin{example}

\end{example}
\begin{equation}
    \lim\limits_{n \rightarrow \infty}{\frac{11+1}{9-n}}\quad ? \\
    \\\quad X_n = \frac{11+1}{9-n}=\frac{n}{n} \frac{\frac{11}{n}+1}{\frac{9}{n}-1}
\end{equation}

\begin{equation}
    \lim\limits_{n \rightarrow \infty}{\bigg(\frac{11}{n}+1\bigg)}=1
\end{equation}

\begin{equation}
    \lim\limits_{n \rightarrow \infty}{\bigg(\frac{9}{n}+1\bigg)}=-1
\end{equation}

\begin{equation}
    \lim\limits_{n \rightarrow \infty}{(X_n)}= \frac{1}{-1}=-1
\end{equation}

\begin{lemma}
    Seien $(x_n)=(y_n)$ Folgen auf $\lim\limits_{n \rightarrow \infty}{(x_n)}= \lim\limits_{n \rightarrow \infty}{(y_n)}= a$ und es gelte
    $x_n \leq z_n \leq y_n$ für "fest alle " $n \in \mathbb{N}$\\

    Dann gilt für die Folge $(Z_n) \lim\limits_{n \rightarrow \infty}{(z_n)}=a$
\end{lemma}

\begin{example}
    Ist die Folge $(-1)^n\frac{1}{n})$ konvergent ?\\

    \[ - \frac{1}{n} \leq(-1)^n(\frac{1}{n}) \leq 1 \frac{1}{n}\]

    \[ \lim\limits_{n \rightarrow \infty}{- \big(\frac{1}{n} \big)}= -1 \]
    \[ \lim\limits_{n \rightarrow \infty}{ \big(\frac{1}{n} \big)}= 0 \Rightarrow \lim\limits_{n \rightarrow \infty}{(-1)^n \frac{1}{n}}= 0
    \]
\end{example}

\newpage

\begin{example}
    \begin{equation}
        \begin{aligned}
            x_n \leq  = \frac{a^n}{n!} = \frac{a}{n} \times \frac{a^{a-1}}{n-1!} %
        \end{aligned}
    \end{equation}\\

    denn $ x_n = 0 \leq \frac{a_n}{n!} \leq y_n$
    , gesucht! $\underbrace{y_n}_{\lim\limits_{n \rightarrow \infty}{y_n}=0}$  für hinreichend großes n.

    \begin{equation}     
    \begin{aligned}   
            \frac{a^n}{n!} = \frac{a}{n} \times \frac{a^{n-1}}{(n-1)!} \\ \leq
            \frac{1}{2} \times
            \frac{a^{n-1}}{(n-1)!} \\ =
            \frac{1}{2} \times
            \frac{a}{(n-1)} \times
            \frac{a^{n-2}}{(n-2)!} \\ \leq
            \frac{1}{2} \times
            \frac{1}{2} \times
            \frac{a^{n-2}}{(n-2)!} \\ \leq
            \frac{1}{2} \times
            \frac{1}{2} \times
            \frac{1}{2} \times
            \frac{a^{n-3}}{(n-3)!}\\
            %
            y_n = (\frac{1}{2})^{n-k} \times \frac{a^k}{k!} \quad \text{k ist fest}
        \end{aligned}
    \end{equation}\\

    {Es gilt} $\frac{a^n}{n!} \leq y_n$  für hinreichend großes n und
    $\lim\limits_{n \rightarrow \infty}{(y_n)}$ \\

    \begin{equation}
        \begin{aligned}
            &=
            \lim\limits_{n \rightarrow \infty}{(\frac{1}{2})^{n-k}} \times
            \underbrace{\frac{a^l}{k!}}_{Konst} \\
            &=
            \lim\limits_{n \rightarrow \infty}{(\frac{1}{2})^{n}} \times
            \underbrace{\lim\limits_{n \rightarrow \infty}{(\frac{1}{2})^{-k}}}_{\in \mathbb{R}} \times
            \underbrace{\lim\limits_{n \rightarrow \infty}{(\frac{a^k}{k!})}}_{\in \mathbb{R}} \\
            &= 0 . (\frac{1}{2})^{-k} \times \frac{a^k}{k!}=0
            \\
        \end{aligned}
    \end{equation}
\end{example}



\newpage

\section{Grenzwerte rekursive definierte Folgen:}

man kann oft durch lösen "Fixpunktgleichung" berechnen.\\
$x_0 \quad , x_n+1 = ln(x_n)$

\begin{example}
\[(x_n) \quad x_0 = \frac{7}{5} \quad,\quad x_n+1= \frac{1}{3}(x_n^2+2)  \]

Ü $(x_n)$ ist monoton fallend , beschränkt , konvergent . 

\[\lim\limits_{n \rightarrow \infty}{x_n}=a \quad,\quad 
\lim\limits_{n \rightarrow \infty}{x_n+1}=a \]

\begin{equation*}
\begin{aligned}
\lim\limits_{n \rightarrow \infty}{x_{(n+1)}} 
= \linebreak  
\lim\limits_{n \rightarrow \infty}{\frac{1}{3}(x_n^2 + 2)
\frac{1}{3}} \lim\limits_{n \rightarrow \infty}{(x_n^2 + 2)}
=
\frac{1}{3} (\lim\limits_{n \rightarrow \infty}{(x_n))^2 + 2)}
\end{aligned}
\end{equation*}
\end{example}

\subsubsection{Fixpunktgleichung }
 $a = \frac{1}{3}(a^2 + 2) $  , gesucht = a
 
\[ 3a = a^2 +2 \Leftrightarrow a^2 -3a+2 = 0 \] \\
\[ \Leftrightarrow a_{1/2} = \frac{3}{2} \pm \sqrt{\frac{9}{4}-\frac{8}{4}}= \frac{3}{2} \pm \frac{1}{2}\]
Lösung:  $a_1 = 2$ (keine Lösung),  $a_2 =1 $

\begin{example}{$(x_n)$ mit $(x_0) = c \in \mathbb{R} , c  $ fest $x_{n+1}= \frac{1}{2}(x_n+\frac{c}{x_n})$ }\\
(1) $(x_n)$ beschränkt \checkmark\\
(2) $(x_n)$ Monoton \checkmark\\
Also $(x_n)$ konvergent \\
Sei $\lim\limits_{n \rightarrow \infty}{x_n}= a $. 
Dann $\underbrace{\lim\limits_{n \rightarrow \infty}{x_{n-1}}= }_{a}$ $\lim\limits_{n \rightarrow \infty}{\frac{1}{2}}(x_n) + \frac{c}{x_n} = \frac{1}{2}(a + \frac{a}{c})= a \\
 \Leftrightarrow 2a = a + \frac{c}{a} \Leftrightarrow a = \frac{c}{a } \Leftrightarrow a^2 = c \Leftrightarrow a = \sqrt{c}$
\end{example}

\begin{remark}
Der Nachweis der konvergent der rekursiv definierte Folge darf nicht weggelassen werden, denn Z.B $x_0=2$ , $x_n+1=x_n^2$ \quad \quad \quad 2 , 4 ,16 ,256 , $\dots $ divergent gegen + $\infty$  \\

Annahme: $\lim\limits_{n \rightarrow \infty}{x_n}= a $ 
$\underbrace{\quad \lim\limits_{n \rightarrow \infty}{x_{n+1}}}_{a}$ = 
$\underbrace{\lim\limits_{n \rightarrow \infty}{x_n^2}}_{a} \Rightarrow a \in \{ 0,1 \}$
\end{remark}  

%new 
 
\newpage
\section{Reihen :}
\begin{definition}
Sei $(a_n)$ eine reellefolge (komplexwertig) Folge\\
$$\sum_{k = 0}^{n} {a_k} = a_a , a_1, \dots , a_n , $$
heißt n-k heißt partielle Summe.
$(S_n)$ heißt unendliche Reihe.

schriebweise : $(S_n)^\infty =$ bsw 
$(S_n)$ $$ \bigg( \sum_{l=0}^{n} {a_l} \bigg)$$ bzw
 $$ \bigg( \sum_{l=0}^{\infty} {a_l} \bigg)$$  
\end{definition}

\begin{remark}
Reihen sind spezielle Folgen , alle konvergent oder divergent. 
\end{remark}

\begin{definition}
Für eine konvergente Reihen wird der Grenzwert auch wert der Reihe genannt.\\

\begin{schreibweise}
 :  $\lim\limits_{n \rightarrow \infty}{S_n}= $
$$\lim\limits_{n \rightarrow \infty}{ \sum_{k=0}^{n} {a_k} }  $$ 
bzw 
$$ \sum_{k=0}^{\infty} {a_k}  $$
\end{schreibweise}

\end{definition}

\begin{example}{Teleskopreihe} 
\begin{equation}
\begin{aligned}
\sum_{k=1}^\infty(\frac{1}{k}-\frac{1}{k+1}) \text{in Grenzwert der Reihe ist}\\ 
\sum_{k=1}^\infty(\frac{1}{k}-\frac{1}{k-1})=1 \\
\lim_{n \to \infty}{S_n} = \lim_{n \to \infty}{\sum_{k=1}^n(\frac{1}{k}- \frac{1}{k-1}) }\\
= lim_{n \to \infty}{(\frac{-1}{2})+\frac{1}{2}(\frac{1}{3} + \frac{1}{3})(-\frac{1}{4})+) \dots +(\frac{1}{n})-\frac{1}{n+1}}\\
= lim_{n \in \infty}{(1- \frac{1}{n+1})}= 1-0 =1 
\end{aligned}
\end{equation}
\end{example}

\begin{example}
geometrische Reihe $ \sum_{k=0}^{\infty} {q^k} $
ist für $$ {|q|} < 1 $$ konvergent . wert der Reihe für $ {|q|} < 1 $  $\sum_{k=0}^{\infty} {q^k}= \frac{1}{1-q} $ für 
$ {|q|} < 1 $  konvergent , werte der Reihe für 
$$ {|q|} <1 : \sum_{k=0}^n{q^k}= \dots $$
\begin{equation}
\begin{aligned}
S_n = q^0 + q^1 + \dots + q^n | *q \\
-q S_n = q^1 + q^2 + \dots + q^{n+1} \\
(1-q)S_n=q^0 - q^n+1 \\
S_n = \frac{1-q^{n+1}}{1-q} = \frac{1}{1-q}(1-q)^{n+1}\\
\Rightarrow lim_{n \to \infty}{S_n} = \frac{1}{1-q} \times 
\lim_{n \to \infty}{((1-q)^{n+1})}\\
=\frac{1}{1-q}(1- \lim_{n \to \infty}{q^{n+1}})
\end{aligned}
\end{equation}
\subsection{Rechnen für Reihen}
konvergent Reihe kann man addieren oder subtrahieren mit einem Skalar multiplizieren
wie endliche Summen.
aber das gilt im Allgemein nicht für das Multiplizieren  
\end{example}
%%
%% Author: mahmoud
%% 19/04/18
%%


%fehlend ein Beispie mit g=A 
%2A=1^2+(1/2)^2

\chapter{Vorlesung 4}

\section{Reihen}
\begin{example}{Zur geometrischen Reihen}\\
gesucht : A
\[2A = 1^2 + (\frac{1}{2})^2+(\frac{1}{4})^2 + \dots + (\frac{1}{k})^2 + ... \]\\
\[= (\frac{1}{4})^0 + (\frac{1}{4})^1+ (\frac{1}{4})^2+(\frac{1}{2^2})^3+(\frac{1}{2^2})^k+ \dots \]\\
\[9 = \frac{1}{4} = \frac{1}{1- \frac{1}{4}} = \frac{1}{\frac{3}{4}} = \frac{4}{3}= 2A \Rightarrow A = \frac{2}{3}\] 
\end{example}

\begin{example}
\begin{equation}
\begin{aligned}
0,4 \overline{3} = \frac{3}{4} + \frac{3}{100} + \frac{3}{10000}+ \dots \\
\frac{4}{10} + \frac{3}{100}(\frac{1}{10})^0 + \frac{1}{10} + \frac{3}{10^2} + \dots \\
=\frac{4}{10} + \frac{3}{100} \times \frac{1}{1-\frac{1}{10}}\\
= \frac{4}{10}+ \frac{1}{30} = \frac{12+1}{30}= \frac{13}{30}
\end{aligned}
\end{equation}
wenn $0,4 \overline{3}$ erlaubt wäre, dann,\\
\[\frac{4}{10} + \frac{9}{100} \times \frac{10}{9} = \frac{4}{10} + \frac{1}{10} = \frac{5}{10} = \frac{1}{2} = 0.5  \]
\end{example}

\newpage
\begin{example}
	\[\sum_{R=1}^{\infty} \frac{1}{K} \text{ ist divergent ,denn  } \lim\limits_{\infty} \sum_{K=1}^{n} \frac{1}{k} \text{ ex. nicht } \]
	
	\[\frac{1}{1}+\frac{1}{2}+\frac{1}{3}+\frac{1}{4}+\frac{1}{5}+\frac{1}{6}+\frac{1}{7}+\frac{1}{8}+\frac{1}{9}+\dots+\frac{1}{16}+\dots+\frac{1}{n}\]
	
	\[ > 1+\frac{1}{2}+  \underbrace{\frac{1}{4}+\frac{1}{4}}+\underbrace{\frac{1}{8}+\frac{1}{8}+\frac{1}{8}+\frac{1}{8}}+\frac{1}{10}+\frac{1}{16}+\dots+\frac{1}{n}\] 
	
	\[1+\frac{1}{2} + \qquad \frac{1}{2}+   \quad  \qquad \frac{1}{2}  + \dots + \frac{1}{n}  \rightarrow \lim\limits_{n\rightarrow \infty }s_n = \infty \]
	
\end{example}
\section{Allgemeine harmonische Reihe}

				\begin{equation*}
				\begin{split}
				 \sum_{K=1}^{\infty} \frac{1}{k^\alpha} \quad( \infty \text{fest}) \qquad
				& \alpha > 1 \rightarrow \mathbb{R}  \\
				& \alpha \leqq \rightarrow dev
				\end{split}
				\end{equation*}
 \begin{example}
 	\[ \sum_{k=1}^{\infty} \frac{1}{k^2} \qquad \text{ist konvergent}\]
 	
\begin{beweis}{mit Monotoniekriterium für Folge}  
		\[ \text{Reihe ist konvergent} \left\{ \begin{array}{ll}
	   (1) & \mbox{ $\sumIn \frac{1}{k^2} $ \quad ist monoton wachsend};\\
	   (2) & \mbox{$\sumIn \fracKetwo $ \text{ist beschränkt}}.\end{array} \right. \] 
	   	
	   \[ \sumIin \fracKetwo = \frac{1}{1}+\frac{1}{2^2}+\frac{1}{2^2}+\frac{1}{4^2}+\frac{1}{5^2}+  \dots + \frac{1}{8^2}  \]
	   \[ <  1 + \frac{1}{4}+\underbrace{\frac{1}{2^2}+\frac{1}{2^2}}_{2. \frac{1}{4}}+\underbrace{\frac{1}{4^2}+\dots + \frac{1}{4^2}}_{4.\frac{1}{4^2}}+\]
	   
	   \[ 1+\frac{1}{4}+\frac{1}{2}.1+\underbrace{\frac{1}{4}}_{(\frac{1}{2})^2}+\underbrace{\frac{1}{8}}_{(\frac{1}{2})^3} =1+\frac{1}{4}+\frac{\frac{9}{4} }{1-\frac{1}{2}-1} \]
	   
 \end{beweis}
 \end{example}
%photo auf der Tafel
\section{Expotentiale Reihe}

\[ \sumOin \frac{1}{k!} = \limNin ( 1-\frac{1}{n})^n =:e \text{ist konvergent} \]

 \pgfkeys{/forest,
	tria/.append style={ellipse, draw},
}
	\begin{forest}
	for tree={l+=0.5cm, s sep+=1cm}
	[Konvergentkreterium für Reihen 
	[	für Folge ]
	[Kreterien für absolute konvergenz [konvergent Reihen] ]
	[Hauptkriterium ]
	]
	\end{forest}

\section{Hauptkriterium}  
$$\sum_{k=0}^{\infty}{a_k} \quad \text{konvergent}
 \Rightarrow (a_k) \text{Nullfolge.}$$ \\
$$\lim\limits_{k \rightarrow \infty}{a_k} \neq 0 \Rightarrow  \sum_{k=0}^{\infty}{a_k} \quad \underbrace{null konvergent }_{divergend}$$\\
oder
			$$\lim\limits_{k \rightarrow -\infty}{ a_k} \quad ex.null$$

						\begin{example}
						
						\[ \sumIin \frac{3k^2+1}{4k^2-1} \quad divergend, \quad aber \quad \sumIin \fracKone \quad divergend \quad \text{und $\fracKone$ Null folge }  \]
						

						\end{example}
\begin{beweis}
\[  \sumOin a_k konv. \Rightarrow \underbrace{(a_k Nullfolge) }_{\limKin a_k=0}  \] 
\[s_n =\sumOn a_k , s_{n+1} =\sum_{k=0}^{n+1} \qquad s_{n+1}=s_n + a_{n+1} \]
\[s= \limNin s_n= \limNin s_{n+1} \qquad \limNin a_{n+1} = \limNin s_{n+1} - \limNin s_n = s-s=0 \]
\end{beweis}

\newpage
\section{Kriterium für Alternierende  Reihe}
\begin{beweis}{Alternierende }
	$\sumOin (-1)^k \fracKone$ ist konvergent\\
	$\sumOin (-1)^k a_k$\\
	wobei $(a_k)$ einer Streng monoton fallend Nullfolge mit $a_k \geqq 0 $ \\
	$\Rightarrow$ Die Reihe ist konvergent. Also $\sumOin (-1)^k \fracKone$ ist konvergent.

\end{beweis}
\begin{definition}[Reihe]
	
	\text{Reihe} $\sumOin  a_k$  \text{heiß absolute konvergent wenn} $\sumOin $ $|a_k|$ \text{konvergent ist.
	}
\end{definition}
\begin{example}

\[	\sumIin (-1)^k \fracKone \text{ ist konvergent , aber nicht absolote konvergent } \]
\end{example}

\begin{example}
\[ \sumIin (-1)^k \fracKetwo  \text{ ist kovergend und abslote konvergent } \]
\end{example}

\begin{theorem}
Reihe $\sumOin$ $a_k$ abslot konvergent $\Rightarrow$ Reihe $\sumOin$ $a_k$ ist kovergend	
\end{theorem}

\begin{remark}	
Absolute konvergente Reihe kann man multiplizieren wie endliche summen d Reihen null
\end{remark}

\section{Quotionkriterium (QK):}
für endliche Konvergenz\\
\[\limKin |\frac{a_{k+1}}{a_k}|\] \\$< 1 \Rightarrow \sumOin a_k $ in absolut konvergent \\
$>1 \Rightarrow $ ist divergent\\
$=1 $ Kriterium ist nicht anwendbar 
\section{Wurzel kriterium : WK }
für (absolute) konvergent 
\\ $\limKin \sqrt[k]{|a_k|}$\\
$<1 \Rightarrow \sumOin a_k$ in (absolute) konvergent\\
$>1 \Rightarrow$ divergent\\
$=1 $ Kriterium ist nicht anwendbar

\begin{example}[QK]
	 \begin{equation*}
		 \begin{split}
	    	\sumOin \dfrac{1}{k!}
	 	   \limKin {|d\frac{\frac{1}{(k+1)!}}{\frac{1}{k!}}|} &= \limKin \frac{k!}{(k+1)!} \\
													   	   &=\limNin \frac{1}{k+1}   \\
													   	   &=0<1 \Rightarrow  Reihe \quad als \quad konv. 
		    \end{split}	 
		 \end{equation*}     
	
\end{example}
\begin{example}[WK]
	\begin{equation*}
	\begin{split}
	\limKin \sqrt[k]{\frac{1}{k!}}= \limKin \frac{\sqrt[k]{1}}{\sqrt[k]{k!}}=\frac{1}{\limKin \sqrt[k]{k!}}=0\\ <1 \\
	\Rightarrow Reihe \quad als \quad konv.
	\end{split}	 
	\end{equation*}     
	
\end{example}
\section{Vorlesung 5}

Zusammenfassung :\\
Folgen / Reihen / Konvergenz ? / Grenzwert ? \\ 
Neu : Funktionen\\ 
Approximation von Funktionen \\
Potenzreihen\\
Taylorreihen\\
fourierreihen\\
Näherungsweise Berechnung
\begin{definition}
$ f : \mathbb{D} \rightarrow \mathbb{R} $
heißt reelle Funktion in einer reellen veränderlichen  

\end{definition}

\begin{remark}[Definitionsbereich]
Bild von $f$ \[ f(D)= \{ f(x) \quad | \quad x \in D \} \] \\
Graph von $f$ \[ Graph(f)= \{( x \quad |\quad f(x)) \quad | \quad x \in D \} \] \\

\end{remark}   

\begin{definition}
Sei $f : D \rightarrow \mathbb{R} , D \subseteq \mathbb{R} , a \in D \quad $\\
$f$ heißt in a stetig , wenn gilt :\\
$\forall (X_n) : X_n \in D$ und 
$\limNin{f(x_n)} = f(a)$ für alle Folgen $(x_n)$\\
Die Folgenglieder sollen in Definitionsbereich liegen (Die in Definitionsbereich liegen können und den Grenzwert a haben)
\end{definition}

* Ich weiß , dass $f(x_n)$ existiert $(f(x_n)\quad ex.)$\\
 Folge $f(x_n)ex.$ , soll einen Grenzwert besitzen.$\checkmark$\\
$f(\limNin{x_n})\checkmark \checkmark$

\begin{remark}
\[ \limNin {f(x_n)= f(\limNin{x_n})} \]
$\bigstar$ Grenzwertbildung und Funktion Wertberechnung sind bei stetig Funktion in der Reihenfolge vertauschbar !  
\end{remark} 

\begin{rechnen}
$$\lim_{x \to a} f(x)$$ \\
$\textbf{d.h }$ für jede Folge $x_n$ , die gegen a konvergiert , konvergiert die Folge der Funktionierte gegen $f(a)$.
\end{rechnen}

\begin{remark}
$f$ stetig in a $\Leftrightarrow$ \\
1) $f(a)$ und \\
2) $lim_{x \to a}{f(x)}$ ex. und \\
3) Grenzwert = Funktionswert 
$\lim\limits_{x \rightarrow a}{f(x)} = f(a)$ \\


\begin{tikzpicture}[scale=1.1],

\begin{axis}[xlabel=$x$,
ylabel= $\delta$,
height=6cm,
width=10cm,
ymax=11,
ymin=0,
xmin=0,
xmax=6.5,
axis y line=left,
axis x line=bottom,
ytick={1,...,10},
yticklabels={ ,,$f(a)$ ,,},
xtick={1,...,5},
xticklabels={$x_0$,$x_2$,$a$,$x_3$,$x_1$,}
]
  \addplot[dashed] coordinates {(0,1.5) (1,1.5)};
  \addplot[dashed] coordinates {(1,0) (1,1.5)};
  \addplot[dashed] coordinates {(0,2) (2,2)};
  \addplot[dashed] coordinates {(2,0) (2,2)};
  \addplot[blue,dashed] coordinates {(0,3) (3,3)};
  \addplot[blue,dashed] coordinates {(3,0) (3,3)};
  \addplot[dashed] coordinates {(0,5) (4,5)};
  \addplot[dashed] coordinates {(4,0) (4,5)};
  \addplot[dashed] coordinates {(0,9) (5,9)};
  \addplot[dashed] coordinates {(5,0) (5,9)};

    \addplot[black,domain=1:9]  { 2^(\x-2)+1  }node {f};
    


\end{axis}
\end{tikzpicture}


\end{remark}


\begin{example}
\begin{itemize}
1)
\begin{align*}
f(x) = \frac{x^2 - 1}{x - 1} =   
\frac{(x-1)(x+1)}{(x-1)}\\ 
\end{align*}

Ist $f(x)$ stetig in $a = 1$ ?\\

a) \quad $f(1)$ ex ? nein , d.h $f$ ist in $a = 1$ nicht stetig\\ 
\\
b) $$ \lim_{x \to 1}f(x) = \lim_{x \to 1}{\frac{(x-1)(x+1)}{x-1}} = \quad ? $$

Sei $(x_n)$ eine beliebige Folge und $x_n \in D(f)$ und $\lim_{x\to \infty}(x_n)=1$
\end{itemize}
\begin{gather*}
\lim_{n \to \infty }{f(x_n)} = \lim_{n \to \infty }
{\frac{(x-1)(x+1)}{(x-1)}} = \lim_{n \to \infty }{(x_n + 1) } = \lim_{n \to \infty }{x_n} + \lim_{n \to \infty }{1} = 1 + 1 = 2 
\end{gather*}

d.h Grenzwert ex. (und es ist 2 ).

\begin{tikzpicture}[scale=1.1],

\begin{axis}[xlabel=$x$,
ylabel= $\delta$,
height=6cm,
width=5cm,
ymax=3,
ymin=0,
xmin=0,
xmax=2,
axis y line=left,
axis x line=bottom,
ytick={1,...,10},
yticklabels={ 1,2},
xtick={1,...,5},
xticklabels={$1$}
]
    \addplot[black,domain=0:1]{x+1}node[above, sloped, pos = 0.3] {g};
    \addplot[black,domain=1:2]{x+1}node[above, sloped, pos = 0.65] {f};
    \draw [blue,  -stealth    ] (3,1) -- (1.1,2) node [right] {Lücke};
    \addplot[mark=*,fill=white] coordinates {(1,2)};

\end{axis}
\end{tikzpicture}

Man sagt , $f$ hat an der stelle 1 eine Lücke.
\end{example}

\begin{example}
(2) $$ f(x)=\frac{1}{x} \quad ,\quad  a = 0 $$
\begin{tikzpicture}[scale=1.1],
\begin{axis}[xlabel=$x$,
ylabel= $y$,
height=6cm,
width=5cm,
ymax=3,
ymin=-3,
xmin=-3,
xmax=3,
axis y line=center,
axis x line=center,
]
\addplot[domain= -3:-0.01] {1/x};
\addplot[domain= 0.01:3] {1/x};


\end{axis}
\end{tikzpicture}


(i) betrachte $? \lim\limits_{x \rightarrow 0^-} {f(x):}$ d.h wir
betrachten alle Folgen $(x_n)$ \\


\begin{align*}
X_n \in D , X_n \leq 0 \limNin (x_n) = 0 \\
\limNin f(x_n) = \limNin \frac{1}{x_n} \\
= \frac{\limNin {1}}{\lim\limits_{n \rightarrow - \infty}{x_n}} 
= \frac{1}{ \lim\limits_{n \rightarrow - \infty}{x_n}} = - \infty\\
\\
\text{d.h} \quad \lim_{x \to 0^-}{f(x)} \text{ex .nicht}
\end{align*}
\\
(ii) Betrachte $\lim\limits_{n \rightarrow + 0}{f(x_n)}$ , ex .nicht\\
\newpage
(3)\\
\begin{tikzpicture}[scale=1.1],
\begin{axis}[xlabel=$x$,
ylabel= $y$,
height=6cm,
width=5cm,
ymax=3,
ymin=-3,
xmin=-3,
xmax=3,
axis y line=center,
axis x line=center,
]
\addplot[domain= -3:-0.01] {1/x};
\addplot[domain= 0.01:3] {1};


\end{axis}
\end{tikzpicture}


$$
f(x) = \left\{\begin{array}{lr}
        1 , & x \geq 0\\
        \frac{1}{x} , & x < 0 
        \end{array}\right\} a = 0 \quad , \quad f(0) = 1 \quad \text{ex.}  
$$

$$ \lim\limits_{x \rightarrow 0^+}{f(x)} = 1 , 
\lim\limits_{x \rightarrow  0^-}{f(x)}= - \infty \quad \text{ex. nicht} $$ \\
 
(4)\\

\begin{tikzpicture}[scale=1.1],
\begin{axis}[xlabel=$x$,
ylabel= $y$,
height=6cm,
width=5cm,
ymax=3,
ymin=-3,
xmin=-3,
xmax=3,
axis y line=center,
axis x line=center,
]
\addplot[domain= -3:0] {-1};
\addplot[domain= 0:3] {1} node[above ,pos=0.8] {f(x)};
\addplot[mark=*,fill=white] coordinates {(0,-1)};
  \draw [decorate, decoration={brace,amplitude=5pt,raise=4pt,mirror}] (0,1) -- (0,-1) 
node [midway, xshift=-0.5mm, yshift=1mm,auto, swap, outer sep=9pt,font=\tiny]{Sprung};
\end{axis}
\end{tikzpicture}

\begin{definition}[ sgn(x)]
Die Vorzeichenfunktion oder \textbf{Signumfunktion} (von lateinisch signum ‚Zeichen‘) ist in der Mathematik eine Funktion, die einer reellen oder komplexen Zahl ihr Vorzeichen zuordnet.\\
Die reelle Signumfunktion bildet von der Menge der reellen Zahlen in die Menge \{-1,0,1\} ab und wird in der Regel wie folgt definiert:
$$ f(x) = \underbrace{sgn(x)}_{sprung}  =  \left\{\begin{array}{lr}
+ 1 , & x \geq 0\\
0 , & x = 0 \\
-1 , & x < 0 
        \end{array}\right\}$$ 
\end{definition}
$$
\neq \left\{\begin{array}{lr}
  \lim\limits_{x \rightarrow 0^-}{f(x)} = -1 \quad \text{ex.} \\
        \lim\limits_{x \rightarrow 0^+}{f(x)} = 1 \quad \text{ex.} 
        \end{array}\right\} \lim\limits_{x \rightarrow 0}{f(x)} \quad \text{ex. nicht , O heißt Sprungstelle}
$$  
\end{example}
\newpage
\begin{definition}
\begin{align*}
f : \rightarrow \mathbb{R} , \quad D \subseteq \mathbb{R} \quad \text{heißt \textbf{stetig} , wenn f für alle } a \in D \quad \textbf{stetig} 
\end{align*}
\end{definition}
\begin{example}
elementare Funktionen und deren Verfügungen sind stetig auf dem gesamten Definitionsbereich. \\
$\textbf{Z.B}$ \\
Polynomfunktion , rationale Funktionen, Winkelfunktionen , Potenzfunktionen , Wurzelfunktionen , Exponentialfunktionen und Logarithmusfunktion.
     
\end{example}

\begin{example}
%fehlende Skizze 
\begin{align*}
f : D \rightarrow \mathbb{R} : x \rightarrow \frac{1}{x} = x^{-1} \text{ist stetig auf dem gesamten  Defintionsbereich } D = \mathbb{R}   \backslash \{0\}  
\end{align*}

\begin{tikzpicture}[scale=1],

\begin{axis}[xlabel=$x$,
ylabel= $y$,
height=12cm,
width=17.5cm,
ymax=5,
ymin=-3,
xmin=-3,
xmax=3,
axis y line=center,
axis x line=center,
]
\addplot[domain= -3:-0.01] {1/x};
\addplot[domain= 0.01:3] {1/x};
\draw [red,  -stealth    ] (-0.5,0.7) -- (-0.1,0.1) node [left,pos=0] {kein intervall im  Definition Bereich};
\addplot[mark=*,fill=white] coordinates {(0,0)};
\addplot[ very thick,mark options={solid},blue,domain= 0.8:2.3] {1/x};
\addplot[ very thick,mark options={solid},blue,domain= 0.8:2.3] {0};
\draw [blue,  -stealth    ] (0.5,-2) -- (1.2,-0.6) node [below,pos=0] { intervall };

\end{axis}
\end{tikzpicture}

\end{example}
\newpage
\begin{beweis}
Sei $ a \in D = \mathbb{R} \backslash \{0 \}$ (d.h a $\neq 0$) 
\begin{align*}
f(a)  &= \frac{1}{a} \tag{1} \\ 
\lim_{n \to \infty}{f(x)}  &=
\lim_{n \to \infty}{\frac{1}{x}} \tag{2}
\end{align*} 

\begin{align*}
\text{Sei} \quad x_n \quad \text{eine beliebige Folge und }\quad x_n \in \underbrace{D}_{\mathbb{R}\backslash \{ 0 \}}  \quad \text{und}\quad \lim_{n \to \infty}{x_n}= a
\end{align*}
\begin{align*}
\lim_{n \to \infty}{f(x_n)}
&=  \lim_{n \to \infty}{\frac{1}{x_1}} \\
&= \frac{\lim\limits_{n \rightarrow \infty}{1}}{\lim\limits_{n \rightarrow \infty}{x_2}} \\
&= \frac{1}{a} \in \mathbb{R} \quad \text{ex.}
\end{align*}
\end{beweis}  
\subsection{Rechnenregln für Funktionen (GWS anwenden)}
%fehlende Skizze 
\begin{gather*} 
\lim\limits_{x \rightarrow \infty}{(f(x) \pm g(x))} =\lim\limits_{x \rightarrow \infty}{f(x)} \pm  \lim\limits_{x \rightarrow \infty}{g(x)} , \text{ wo bei } g(x)\neq 0  \\
\lim\limits_{n \rightarrow \infty}{(f(n) \pm g(n))} =\lim\limits_{n \rightarrow \infty}{f(n)} \pm  \lim\limits_{n \rightarrow \infty}{g(n)} 
\end{gather*}


\begin{theorem}
\begin{gather}
f: D \Rightarrow \mathbb{R} , \quad D \subseteq \mathbb{R} \text{ ist in } a \in D \text{ Stetig }
\Leftrightarrow \forall_{\epsilon} > 0 \quad \exists \delta > 0 : |x-a| < \delta \Rightarrow |f(x)- f(a)|< \epsilon 
\end{gather} 
\end{theorem}
%fehlende Skizze
  
\begin{tikzpicture}[scale=1.2],

\begin{axis}[
height=6cm,
width=10cm,
ymax=11,
ymin=0,
xmin=0,
xmax=6.5,
axis y line=left,
axis x line=bottom,
ytick={1,...,10},
yticklabels={ ,$f(a)-\epsilon$ ,,,,$f(x)$,,,$f(a)+\epsilon$},
xtick={1,...,5},
xticklabels={,,$a-\delta$,$a$,$a+\delta$}
]
\addplot[dashed] coordinates {(0,2) (2,2)};
\addplot[dashed] coordinates {(2,0) (2,2)};
\addplot[blue,dashed] coordinates {(0,3) (3,3)};
\addplot[blue,dashed] coordinates {(3,0) (3,3)};
\addplot[dashed] coordinates {(0,5) (4,5)};
\addplot[dashed] coordinates {(4,0) (4,5)};
\addplot[blue,dashed] coordinates {(0,9) (5,9)};
\addplot[blue,dashed] coordinates {(5,0) (5,9)};

\addplot[black,domain=1:9,name path=A]  { 2^(\x-2)+1  } node at (5.5, 8.5) {f};
\addplot[very thick,blue,name path=B,domain= 3:5] {0};
\addplot[blue,very thick] coordinates {(0,3) (0,9)};
\addplot[gray, pattern=north west lines] fill between[of=A and B, soft clip={domain=3:5}];
\legend{$qeg \; \epsilon$, $gce \; \delta$}
\end{axis}
\end{tikzpicture}
\listoftheorems[ ignoreall,show={definition}]
\listoftheorems[ignoreall,show={example}]
\end{document}
