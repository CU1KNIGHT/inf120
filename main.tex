% Mathematische Methoden für Informatiker: Institut %Algebra Algebra
% TU Dresden SS19
% Prof Dr Ulrike Baumann

% Setup document
\documentclass[a4paper,12pt]{report}
\usepackage{graphicx}
\usepackage{svg}
\usepackage{inputenc,fontenc}
\usepackage{longtable,tabularx,tabulary,array,booktabs,multicol}
\usepackage{graphicx,caption,color,xcolor}
\usepackage{chngcntr}
\usepackage{verbatim}
\usepackage{url}
\usepackage{amsmath,amsthm,amssymb,amsfonts}
\usepackage[nottoc]{tocbibind}
\usepackage{mathtools}
\usepackage{mathabx}
\usepackage{thmtools}
\usepackage[a4paper,margin=3cm]{geometry}
\usepackage{pgfplots}
\usepackage[english,german]{babel}
\usepackage{hyperref}
\usepackage{MnSymbol} %sch\"{o}nere Mathe-fonts
\usepackage{eucal}    %sch\"{o}nere Skript-Buchstaben
\usepackage{dsfont}   %sch\"{o}nere "Doppelstrich"-Fonts
\usepackage{tikz}
\usepackage{forest}



% Fontauswahl

\pgfplotsset{compat=1.12}

\usepgfplotslibrary{fillbetween}
\usetikzlibrary{patterns}

\newcommand{\mathscr}{\mathcal}
%\usepackage{picinpar} %Einbinden von Bildern
%\usepackage{picins}   %Einbinden von Bildern

% Define oftenly used commands

\newcommand{\nameE}{Abdelshafi}
\newcommand{\vornameE}{Mohamed}
\newcommand{\emailE}{m.abdelshafi@mail.de}
\newcommand{\nameS}{Kiki}
\newcommand{\vornameS}{Mahmoud}
\newcommand{\emailS}{mahmoud.kiki@tu-dresden.de}



%% Created defnintions 
\def\checkmark{\tikz\fill[scale=0.4](0,.35) -- (.25,0) -- (1,.7) -- (.25,.15) -- cycle;} 




%% Theorem environments etc.
%% See also: ftp://ftp.ams.org/ams/doc/amscls/amsthdoc.pdf
%%
%created theorem
\swapnumbers
\theoremstyle{plain} %Text ist Kursiv
\newtheoremstyle{break}% name
{}%         Space above, empty = `usual value'
{}%         Space below
{\itshape}% Body font
{}%         Indent amount (empty = no indent, \parindent = para indent)
{\bfseries}% Thm head font
{.}%        Punctuation after thm head
{\newline}% Space after thm head: \newline = linebreak
{}%         Thm head spec

\newtheorem{theorem}{Satz}[chapter]
\newtheorem{lemma}[theorem]{Lemma}
\newtheorem{proposition}[theorem]{Proposition}
\newtheorem{corollary}[theorem]{Korollar}


%\theoremstyle{definition} %Text ist \"upright"
\theoremstyle{break}
\newtheorem{remark}[theorem]{Bemerkung}
\newtheorem{definition}[theorem]{Definition}
\newtheorem{example}[theorem]{Beispiel}
\newtheorem{beweis}[theorem]{Beweis}
\newtheorem{schreibweise}[theorem]{Schreibweise}
\newtheorem{rechnen}[theorem]{Berechnung}



\hypersetup{
	colorlinks=true,
	linkcolor=blue,
	filecolor=magenta,
	urlcolor=cyan}


{\author{ \vornameE \nameE \vornameS \nameS}}
{\title{Mathematische Methoden für Informatiker}}







\newcommand{\institut}{Institut Ihres Betreuers / Ihrer Betreuerin}

\newcommand{\thema}{Mathematische Methoden für Informatiker}

\newcommand{\datum}{\today}%Format tt.\ mm.\ jjjj

%shortcut :Mahmoud
\newcommand{\sumIn}{\sum_{K=1}^{n}}
\newcommand{\sumIin}{\sum_{K=1}^{\infty}}
\newcommand{\sumOin}{\sum_{K=0}^{\infty}}
\newcommand{\sumOn}{\sum_{K=0}^{n}}
\newcommand{\fracKetwo}{\frac{1}{k^2}} %1/(k^2)
\newcommand{\fracKone}{\frac{1}{k}}   %1/k
\newcommand{\limNin}{\lim\limits_{n \rightarrow \infty}}
\newcommand{\limKin}{\lim\limits_{k \rightarrow \infty}}
\newcommand{\limXin}{\lim\limits_{x \to \infty} }
\newcommand{\limXo}{\lim\limits_{x \to 0} }


%created defintions 
\newcommand*\circled[1]{\tikz[baseline=(char.base)]{
    \node[shape=circle,draw,inner sep=2pt] (char) {#1};}}

\begin{document}
	\selectlanguage{german}
	%\selectlanguage{english} %Entferne "%", wenn Sprache Englisch ist



	%% Titelseite

	\thispagestyle{empty}


    \begin{center}
    {\Large
    Technische Universit\"{a}t Dresden\  \ \textbullet\ \ Fakult\"{a}t Informatik
    }

        \vfil

        {\bfseries\Huge\thema}

        \vfil

        {\LARGE
Mitschrift zur Vorlesung Sommer Semester 2019  \\[\bigskipamount]

        \bfseries{\itshape Bachelor of Science  \textup{(}B.Sc.\textup{)}}\\[\bigskipamount]
        }% Ende Large

        \vfil\vfil\vfil
        Dozent: Prof. Dr. Ulrike Baumann \\
        vorgelegt von\\
        \item "..." \\
        \item \textsc{\vornameE\ \nameE } \\ \texttt{\emailE} \\  \item
        \textsc{\vornameS\ \nameS \qquad } \\ \texttt{\emailS}  \\
        \item ... \\
        Tag der Einreichung: \datum\\[\bigskipamount]

    \end{center}


    \cleardoublepage



    \tableofcontents

    \thispagestyle{empty}

    %\makeatletter
    %\begin{titlepage}

    %	The title is \@title
    %	It was written by \@author\space on \@date

    %\end{titlepage}
    \setcounter{page}{0}
    \chapter*{Einleitung}
    Wir schreiben hier die Vorlesungen von INF-120-1( Mathematische Methoden für Informatiker) mit.
    wenn Ihr Fragen habt oder Fehlern gefunden Sie können gerne uns eine E-mail schreiben oder Sie können einfach bei github eine  \href{https://github.com/CU1KNIGHT/inf120/issues}{Issue (link)} erstellen.
    wir freuen uns wenn Sie mit uns mitschreiben möchten, oder helfen mit der Fehlerbehebung.
    \newline \newline
    \vornameE \ \nameE \newline
    \vornameS  \ \nameS
    
    
%%
%% Author: mahmoud
%% 19/04/12
%%

\chapter{Folge und Reihen}
\section{Folgen}
\numberwithin{equation}{theorem}
\begin{definition}[Folgen]
    Ein folge ist eine Abbildung

    \[ f: \mathbb{N} \rightarrow \underbrace{\mathbf{M}}_{Menge} : \mathrm{n} \mapsto \underbrace{X_n}_{folgenglied} \]

\end{definition}
\begin{remark}

    \begin{align}	\mathbf{M} &= \mathbb{R} \quad \text{reelewert Folge} \notag \\
    \mathbf{M}&= \mathbb{C} \quad	\text{komplexwertig Folge}   \notag \\
    \mathbf{M}&= \mathbb{R}^n \quad \text{vertical Folge} \notag
    \end{align}




\end{remark}
\begin{description}

    \item[Bezeichnung]

    \quad $(X_n)$ \space \text{mit} \space $ \left( X_n   \right)$=$ \frac{n}{n+1} $
    \\ \\ Aufzählung der folglieder: 0 , $\frac{1}{2}$ ,$\frac{2}{3}$ , $\frac{3}{4}$ , \dots

\end{description}
\begin{remark}
    zuwerten wird $\mathbb{N}$ durch $\mathbb{N}$ {0,1 \dots} erstellt.


\end{remark}
\begin{tikzpicture}
    \begin{axis}[
    ymin = -1,
    ymax = 1,
    xmin = -1,
    xmax = 5,
    axis x line=center,
    axis y line=center]
    \addplot[samples at={1.5,...,4.5},only marks,mark size=1] { x/(x+1)};
    \end{axis}
\end{tikzpicture}
\begin{example}
    \[\]
    \begin{enumerate}

        \item Konstante Folge $(X_n)$ mit \quad $X_n = a \in \mathbf{M },a \dots$ \\
        \[ X_n = a \in \mathbf{M} \]
        \item Harmonische Folge $(X_n)$  mit $X_n$ =  $\frac{1}{n+1}$ \quad$ n \geq 1$
        \item Geometrische folge $(X_n)$ mit $X_n = q^n \:, \: q \in \mathbb{R}, \dots $
        \item Fibonaccifolge $(X_n)$ mit
        \[ X_n =\frac{1}{\sqrt{5}} \Big(  \big( \frac{1+ \sqrt{5}}{2} \big)^n - \big( \frac{1- \sqrt{5}}{2}\big)^n   \Big)    \]

        \item Fibonacci folgen $(X_n)$
        \begin{align}
            X_0 &=0 \notag \\X_1 &= 1 \notag \\
            X_n+1 &= X_n+X_{n-1} \quad (n>0)  \notag
        \end{align}

        \item conway  folge
        \[ 1, 11 ,21 , 1211, 111217, 312211 \dots \]

        \item folge aller Primzahlen: \[ 2, 3 ,5 ,7 ,11, 13 , \dots \]

    \end{enumerate}
\end{example}

\section{Rechnen mit Folgen }
\begin{align*}
    ( M  = \mathbb{R} \quad & oder \quad M = \mathbb{C} ) \\
    (X_n)+(y_n) &:= (X_n+y_n)\\
    K(X_n)&:=(KX_n)\in \mathbb{R} \quad  oder \quad \in  \mathbb{C}
\end{align*}
\begin{remark}

    Die Folge bildet ein Vektorraum.
\end{remark}
\newpage

\begin{definition}$ \newline$
        \begin{enumerate}

        \item Eine reellwertige Funktion ist in der Mathematik eine Funktion, deren Funktionswerte reelle Zahlen sind.

        \item Eine reellwertige heißt beschränkt wenn gilt

        \[	\exists r \in \mathbb{R}_+ , \forall r \in \mathbb{N}: \underbrace{|X_n|}_{\mathclap{\text{Betrag einer reellen oder komplexer Zahl}}} \leqq r   \]

    \end{enumerate}
\end{definition}

\begin{example}
    \[(X_n)\quad mit \quad X_n = (-1)^n \times \frac{1}{n} \]
    \[-1 ,\quad \frac{1}{2}, \quad \frac{-1}{3}, \quad \frac{1}{4} ,\quad \frac{-1}{5},\dots \]
\end{example}



\begin{tikzpicture}
    \begin{axis}[
    ymin = -2,
    ymax = 2,
    xmin = -3,
    xmax = 3,
    axis x line=center,
    axis y line=center]
        \addplot[dashed, name path =A] coordinates {(-3,1) (3,1)};
        \addplot[samples at={-3,...,3},only marks,mark size=1]{(-1)^x*(1/x)};
    \end{axis}
\end{tikzpicture}
\begin{remark}

    $(X_n)$ ist beschränkt mit $r = 1$ denn $|(-1)^n \frac{1}{n}|=|\frac{1}{n}| \leqq 1 \hookleftarrow r $

\end{remark}
\begin{example}
    \[  (X_n) \text{ mit  } X_n = (-1)^n \quad \frac{1}{n}+1 \quad \text{bechränkt r = 3/2}\]

    \[ -3/2 \quad \leq X_n \leq 3/2 \quad \forall n \in \mathbb{N} \]
    %
    \begin{tikzpicture}
        \begin{axis}[
        ymin = -2,
        ymax = 2,
        xmin = -4,
        xmax = 4,
        axis x line=center,
        axis y line=center]
            \addplot[dashed, name path =A] coordinates {(-4,3/2) (4,3/2)};
            \addplot[dashed,samples at={0,...,3},only marks,mark size=1][ domain=(0):(2)] {log10(3*x + 1)};
        \end{axis}
    \end{tikzpicture}
\end{example}

\begin{example}{Standard:}\\

{Die folge}
$ \bigg(\big(1 + \frac{1}{n} \big)^n \bigg)^\infty_{n=1}$
{ist beschränkt durch 3}\\

Zu zeigen: \quad $ -3 \leq X_n \leq 3 \quad \text{für alle} \quad n \in \mathbb{N} $


\[ { (a+b)^n = \sum_{k=0}^{n} \binom{n}{k} a^k . b^{n .k} = \sum_{k=0}^{n} \binom{n}{k} a^{n.k} b^k }  \]
\[  \binom{n}{k} = \frac{n!}{k!(n-k!)} =  \frac{n(n-1) -(n-k-1))}{k!} \]
\[  \sum_{K=1}^{n} \frac{1}{k} = 1+ \frac{1}{2} + \frac{1}{2.3} + \frac{1}{2.3.4} + \dots \]
\end{example}


\section{geometrische Summen Formel (Tafelwerk)}
\begin{definition}

    Die Folge $(X_n)$ heißt monoton $\Big\{ \text{wachsend fallend} \Big\}$
    \[ wenn \quad gilt: \forall n \in \mathbb{N}:
    \left\{
    \begin{array}{ll}
        X_n  & \leq X_n +1 \\
        X_n  & \geq X_n+1
    \end{array}
    \right. \]
    \text{man spricht von Streng monotonie}
    $ wenn \leqq durch > und \geqq durch < \dots  $
\end{definition}
\begin{remark}
    \[ X_n \leq X_{n+1} \ \Leftrightarrow  X_n - X_{n+1} \leq 0 \quad \Leftrightarrow  \frac{X_n}{X_{n+1}} \leq 1 \]
\end{remark}

\begin{example}


    \[	(X_n) \text{ mit }\quad X_0  := 1 \quad, X_{n+1}   := \sqrt{X_n +6} \]
    \text{ist Streng monoton wachsend Beweis mit Vollständiger Induktion}
    \paragraph{Standard Bsp:}
    $ \big( \big(1+ \frac{1}{n} \big)^n \big) $ ist streng monoton wachsend
\end{example}
\begin{remark}

    \[
        \begin{tabular}{|c| c c |}
            \hline
            monoton & ja & nein  \\
            \hline
            \rule{0pt}{3ex}
            Beschränkkeit & $(\frac{1}{n})$ & $(-1)^n$ \\
            nein & (n) & $(-1)^n$ \\
            \hline
        \end{tabular}
    \]

\end{remark}
\begin{definition}
    $(X_n)$ heißt $\boldsymbol{Konvergenz}$ wenn $(X_n)$ ein grenzwert hat.\\
    $(X_n)$ heißt $\boldsymbol{Divergenz}$ wenn sie keinen grenzwert hat.
\end{definition}
\begin{definition}[grenzwert]
 $ a \in$ $\mathbb{R}$ heißt grenzwert von $(X_n)$, wenn gilt:
\[ \underbrace{\forall \epsilon > 0 }_{beliebes \; klein} \quad \underbrace{\exists \mathbf{N} \in \mathbb{N}}_{beliebes \;  klein \; \underbrace{\Rightarrow |X_n -a|< \varepsilon}_{a- \varepsilon \leq X_n \leq a+\varepsilon } } , \forall n \in \mathbb{N} : n \geq \mathbb{N}  \]
\[ \text{Sei  } \varepsilon > 0 ; \varepsilon \text{  fest} \]
\[ \text{alle folglieder$ X_n$ mit n } \geq \mathbb{N} \curvearrowright \]
\begin{tikzpicture}[scale=1.1],

\begin{axis}[
height=8cm,
width=10cm,
ymax=2,
ymin=-2,
xmin=1,
xmax=5.5,
axis y line=left,
axis x line=bottom,
yticklabels={,, $ a - \varepsilon$, a , $a + \varepsilon$ },xtick={1,...,10},
xticklabels={,,N,n},
]

    \addplot[dashed, name path =A] coordinates {(0,1) (5.5,1)};


    \addplot[thick, samples=50, smooth,domain=0:6,magenta, name path=V] coordinates {(3,-2)(3,3)};
    \addplot[dashed, name path =B] coordinates {(0,-1) (5.5,-1)};
    \addplot[gray, pattern=north west lines] fill between[of=A and B, soft clip={domain=3:6}];
    \addplot[samples at={0,...,5},only marks,mark size=1] { 1/x };
    \addplot[samples at={0,...,5},only marks,mark size=1] { (1-x)/x^2 };
   



\end{axis}
\end{tikzpicture}%


\end{definition}

\begin{text}
    ist die folge beschränkt , monoton ?\\

    $(X_n)$ konvergierend : $\iff \exists a \in\mathbb{R} \quad \forall \epsilon > 0 \quad \exists n \quad \in N \quad \forall n \in N \quad \\
    n \geq N \Rightarrow |X_n - a |< \epsilon $
\end{text}

\begin{theorem}

    $(X_n)$ konvergierend : $\Rightarrow$ Der Grenzwert ist eindeutig beschränkt.

\end{theorem}

\begin{proof}
    Sei a eine Grenzwert von $(X_n)$ , b eine Grenzwert von $(X_n)$ \\
    d.h sei $\epsilon > 0$,$\epsilon$ beliebig , $\epsilon$ fest \\

    \begin{equation}
        \exists  N_a \quad \forall n \geq N_a : |X_n-a|< \epsilon
    \end{equation}

    \begin{equation}
        \exists  N_b \quad \forall n \geq N_b : |X_n-b|< \epsilon
    \end{equation}

    Sei max ${N_a,N_b}=N$
    dann gilt : \\
    \begin{equation}
        n \geq N \Rightarrow |X_n - a| < \epsilon
    \end{equation}
    und \begin{equation}
            |X_n -b| < \epsilon \Rightarrow |X_n -a|+|X_n - b|< 2\epsilon
    \end{equation}\\

    Annahme :- a $\neq$ b , d.h $|a-b|\neq 0 $
    \[|a-b|=|a+0-b|
    =|(a-X_n)+(X_n-b)| \leq |X_n - a|+|X_n-b|< 2 \epsilon \]
    also $|a - b|< 2 \epsilon$


    \begin{example}
        \[\epsilon = \frac{|a-b|}\epsilon
        \quad \text{dann gilt}\ :|a-b|<2 \frac{|a-b|}{3}\]\\

        \[ \Rightarrow 1 < \frac{2}{3} \quad falls \quad Aussage, Widerspruch \quad also \quad ist \quad die \quad Annahme \quad falsch \quad also \quad gilt \quad a=b\]

    \end{example}
\end{proof}

\begin{example}

    $X_n$ mit $X_n = \frac{1}{n}$ (harmonische Folge)

\end{example}

\begin{proof}
    Sei $\epsilon > 0 , \epsilon belibig , \epsilon fest$
    gesucht : N mit $n \geq$ N

    \begin{gather}
        \Rightarrow |X_n-a|= |\frac{1}{n} =0|=\frac{1}{n}<\epsilon
    \end{gather}

    wähle N:= $\lceil \frac{1}{\epsilon} \rceil +1$

\end{proof}

\begin{example}
    $\epsilon = \frac{1}{100}$ , gesucht N mit $n \geq N$
    $\Rightarrow \frac{1}{n} < \frac{1}{100}$ wähle $N=101$\\


    Schreibweise: $X_n$ hat den Grenzwert a Limes
    $\lim\limits_{n \rightarrow \infty}{x_n}=a$
    $X_n$ geht gegen a für n gegen Unendlich.
\end{example}

\begin{definition}
    $X_n$ heißt Nullfolge ,wenn $\lim\limits{X_n}=0$ gilt.
\end{definition}

\begin{remark}

    Es ist leichter, die konvergente einer Folge zu beweisen, als den Grenzwert auszurechnen.

\end{remark}

\begin{example}
    $X_n = \frac{1}{3} + \big(\frac{11-n}{9-n}\big)^9$\\


    Behauptung: $\lim\limits_{n \rightarrow \infty}{x_n}=\frac{-2}{3}$

    \begin{lemma}
        \begin{gather}
            \lim\limits_{n \rightarrow \infty}{x_n+y_n}=
            (\lim\limits_{n \rightarrow \infty}{x_n}) +
            (\lim\limits_{n \rightarrow \infty}{y_n})
        \end{gather}
    \end{lemma}

    \begin{gather}
        =\lim\limits_{n \rightarrow \infty}{\bigg(\big(\frac{1}{3}\big)+\bigg(\frac{11-n}{9+n}\bigg)^9\bigg)}
        = \lim\limits_{n \rightarrow \infty}{\frac{1}{3}+
        \lim\limits_{n \rightarrow \infty}{\bigg(\frac{11-n}{9+n}\bigg)^9}}
    \end{gather}

    \begin{gather}
        = \frac{1}{3} + \bigg(\lim\limits_{n \rightarrow \infty}{\frac{11-n}{9+n}}\bigg)^9
    \end{gather}


    \begin{gather}
        = \frac{1}{3} + \lim\limits_{n \rightarrow \infty}{\Bigg(\frac{n(\frac{1}{n}-1)}{n(\frac{9}{n}+1)}\Bigg)^9}
    \end{gather}

    \begin{gather}
        = \frac{1}{3}+\Bigg(\frac{\lim\limits_{n \rightarrow \infty}{(\frac{11}{n})}}{\lim\limits_{n \rightarrow \infty}{(\frac{9}{n}+1})}\Bigg)^9
    \end{gather}

    \begin{gather}
        = \frac{1}{3} + \Bigg(
        \frac{\lim\limits_{n \rightarrow \infty}{\frac{11}{n}} - \lim\limits_{n \rightarrow \infty}{1}}{\lim\limits_{n \rightarrow \infty}{\frac{9}{n}+\lim\limits_{n \rightarrow \infty}{1} } }   \Bigg)^9
    \end{gather}


    \begin{gather}
        =\Bigg(
        \frac{\lim\limits_{n \rightarrow \infty}{11} \times \lim\limits_{n \rightarrow \infty}{(\frac{1}{n}-1)}}{\lim\limits_{n \rightarrow \infty}{9 \times \lim\limits_{n \rightarrow \infty}{(\frac{1}{n}+1)} } }   \Bigg)^9
    \end{gather}

    \begin{gather}
        \frac{1}{3}+(-1)^9 = \frac{1}{3}-1 = \frac{-2}{3}
    \end{gather}
\end{example}

\begin{definition}
    Eine Folge $(X_n)$ hat den unendliche Grenzwert $\infty$, wenn gilt : \\
    \[\forall r \in \mathbb{R} \quad \exists N \in N \quad \forall n \geq N : X_n > r \]

    Schreibweise : $\lim\limits_{n \rightarrow \infty}{X_n}= \infty$
\end{definition}

\begin{remark}
    $\infty$ ist keine Grenzwerte und keine reelle Zahl.
\end{remark}

\begin{remark}
    Grenzwertsätze gelten nicht für uneigentliche Grenzwerte.
\end{remark}

\begin{remark}
    gilt $\lim\limits_{n \rightarrow \infty}{X_n}= \infty$ dann schreibt man $\lim\limits_{n \rightarrow \infty}{X_n}= -\infty$
\end{remark}

\begin{example}
    $X_n$ mit $X_n = q^n$ , $q \in \mathbb{R}$ , $q$ fest.\\

    $ \lim\limits_{n \rightarrow \infty}{q^n} = \begin{cases}
                                                    0 ,\quad |q|<1 \\
                                                    1 ,\quad |q|=1 \\
                                                    \infty ,\quad\quad q > 1  \\
                                                    ex. nicht ,\quad q\leq -1
    \end{cases}$
\end{example}

\vfil
\vfil


\section{Konvergenzkriterien}
(zum Beweis der Existenz eine Grenzwert, nicht zum berechnen von Grenzwert) \\


(1) $X_n$ konvergent $\Rightarrow$ $(X_n)$ beschränkt. \\

wenn $(X_n)$ nicht beschränkt $\Rightarrow$ $(X_n)$ nicht konvergent.\\


(2) Monotonie Kriterium:
wenn $(X_n)$ beschränkt ist können wir fragen ob $(X_n)$    konvergent.\\


$(X_n)$ beschränkt von Monotonie $\Rightarrow$ $(X_n)$ konvergent.

\begin{remark}
    % noch zu schreiben
\end{remark}

\begin{example}

\end{example}
\begin{equation}
    \lim\limits_{n \rightarrow \infty}{\frac{11+1}{9-n}}\quad ? \\
    \\\quad X_n = \frac{11+1}{9-n}=\frac{n}{n} \frac{\frac{11}{n}+1}{\frac{9}{n}-1}
\end{equation}

\begin{equation}
    \lim\limits_{n \rightarrow \infty}{\bigg(\frac{11}{n}+1\bigg)}=1
\end{equation}

\begin{equation}
    \lim\limits_{n \rightarrow \infty}{\bigg(\frac{9}{n}+1\bigg)}=-1
\end{equation}

\begin{equation}
    \lim\limits_{n \rightarrow \infty}{(X_n)}= \frac{1}{-1}=-1
\end{equation}

\begin{lemma}
    Seien $(x_n)=(y_n)$ Folgen auf $\lim\limits_{n \rightarrow \infty}{(x_n)}= \lim\limits_{n \rightarrow \infty}{(y_n)}= a$ und es gelte
    $x_n \leq z_n \leq y_n$ für "fest alle " $n \in \mathbb{N}$\\

    Dann gilt für die Folge $(Z_n) \lim\limits_{n \rightarrow \infty}{(z_n)}=a$
\end{lemma}

\begin{example}
    Ist die Folge $(-1)^n\frac{1}{n})$ konvergent ?\\

    \[ - \frac{1}{n} \leq(-1)^n(\frac{1}{n}) \leq 1 \frac{1}{n}\]

    \[ \lim\limits_{n \rightarrow \infty}{- \big(\frac{1}{n} \big)}= -1 \]
    \[ \lim\limits_{n \rightarrow \infty}{ \big(\frac{1}{n} \big)}= 0 \Rightarrow \lim\limits_{n \rightarrow \infty}{(-1)^n \frac{1}{n}}= 0
    \]
\end{example}

\newpage

\begin{example}
    \begin{equation}
        \begin{aligned}
            x_n \leq  = \frac{a^n}{n!} = \frac{a}{n} \times \frac{a^{a-1}}{n-1!} %
        \end{aligned}
    \end{equation}\\

    denn $ x_n = 0 \leq \frac{a_n}{n!} \leq y_n$
    , gesucht! $\underbrace{y_n}_{\lim\limits_{n \rightarrow \infty}{y_n}=0}$  für hinreichend großes n.

    \begin{equation}
        \begin{aligned}
            \frac{a^n}{n!} = \frac{a}{n} \times \frac{a^{n-1}}{(n-1)!} \\ \leq
            \frac{1}{2} \times
            \frac{a^{n-1}}{(n-1)!} \\ =
            \frac{1}{2} \times
            \frac{a}{(n-1)} \times
            \frac{a^{n-2}}{(n-2)!} \\ \leq
            \frac{1}{2} \times
            \frac{1}{2} \times
            \frac{a^{n-2}}{(n-2)!} \\ \leq
            \frac{1}{2} \times
            \frac{1}{2} \times
            \frac{1}{2} \times
            \frac{a^{n-3}}{(n-3)!}\\
            %
            y_n = (\frac{1}{2})^{n-k} \times \frac{a^k}{k!} \quad \text{k ist fest}
        \end{aligned}
    \end{equation}\\

    {Es gilt} $\frac{a^n}{n!} \leq y_n$  für hinreichend großes n und
    $\lim\limits_{n \rightarrow \infty}{(y_n)}$ \\

    \begin{equation}
        \begin{aligned}
            &=
            \lim\limits_{n \rightarrow \infty}{(\frac{1}{2})^{n-k}} \times
            \underbrace{\frac{a^l}{k!}}_{Konst} \\
            &=
            \lim\limits_{n \rightarrow \infty}{(\frac{1}{2})^{n}} \times
            \underbrace{\lim\limits_{n \rightarrow \infty}{(\frac{1}{2})^{-k}}}_{\in \mathbb{R}} \times
            \underbrace{\lim\limits_{n \rightarrow \infty}{(\frac{a^k}{k!})}}_{\in \mathbb{R}} \\
            &= 0 . (\frac{1}{2})^{-k} \times \frac{a^k}{k!}=0
            \\
        \end{aligned}
    \end{equation}
\end{example}



\newpage

\section{Grenzwerte rekursive definierte Folgen:}

man kann oft durch lösen "Fixpunktgleichung" berechnen.\\
$x_0 \quad , x_n+1 = ln(x_n)$

\begin{example}
\[(x_n) \quad x_0 = \frac{7}{5} \quad,\quad x_n+1= \frac{1}{3}(x_n^2+2)  \]

Ü $(x_n)$ ist monoton fallend , beschränkt , konvergent . 

\[\lim\limits_{n \rightarrow \infty}{x_n}=a \quad,\quad 
\lim\limits_{n \rightarrow \infty}{x_n+1}=a \]

\begin{equation*}
\begin{aligned}
\lim\limits_{n \rightarrow \infty}{x_n+1} 
= \linebreak  
\lim\limits_{n \rightarrow \infty}{\frac{1}{3}(x_n^2 + 2)
\frac{1}{3}} \lim\limits_{n \rightarrow \infty}{(x_n^2 + 2)}
=
\frac{1}{3} (\lim\limits_{n \rightarrow \infty}{(x_n))^2 + 2)}
\end{aligned}
\end{equation*}
\end{example}

\subsubsection{Fixpunktgleichung }
 $a = \frac{1}{3}(a^2 + 2) $  , gesucht = a
 
\[ 3a = a^2 +2 \Leftrightarrow a^2 -3a+2 = 0 \] \\
\[ \Leftrightarrow a_{1/2} = \frac{3}{2} \pm \sqrt{\frac{9}{4}-\frac{8}{4}}= \frac{3}{2} \pm \frac{1}{2}\]
Lösung:  $a_1 = 2$ (keine Lösung),  $a_2 =1 $

\begin{example}{$(x_n)$ mit $(x_0) = c \in \mathbb{R} , c  $ fest $x_{n+1}= \frac{1}{2}(x_n+\frac{c}{x_n})$ }\\
(1) $(x_n)$ beschränkt \checkmark\\
(2) $(x_n)$ Monoton \checkmark\\
Also $(x_n)$ konvergent \\
Sei $\lim\limits_{n \rightarrow \infty}{x_n}= a $. 
Dann $\underbrace{\lim\limits_{n \rightarrow \infty}{x_{n-1}}= }_{a}$ $\lim\limits_{n \rightarrow \infty}{\frac{1}{2}}(x_n) + \frac{c}{x_n} = \frac{1}{2}(a + \frac{a}{c})= a \\
 \Leftrightarrow 2a = a + \frac{c}{a} \Leftrightarrow a = \frac{c}{a } \Leftrightarrow a^2 = c \Leftrightarrow a = \sqrt{c}$
\end{example}

\begin{remark}
Der Nachweis der konvergent der rekursiv definierte Folge darf nicht weggelassen werden, denn Z.B $x_0=2$ , $x_n+1=x_n^2$ \quad \quad \quad 2 , 4 ,16 ,256 , $\dots $ divergent gegen + $\infty$  \\

Annahme: $\lim\limits_{n \rightarrow \infty}{x_n}= a $ 
$\underbrace{\quad \lim\limits_{n \rightarrow \infty}{x_{n+1}}}_{a}$ = 
$\underbrace{\lim\limits_{n \rightarrow \infty}{x_n^2}}_{a} \Rightarrow a \in \{ 0,1 \}$
\end{remark}  

%new 
 
\newpage
\section{Reihen :}
\begin{definition}
Sei $(a_n)$ eine reellefolge (komplexwertig) Folge\\
$$\sum_{k = 0}^{n} {a_k} = a_a , a_1, \dots , a_n , $$
heißt n-k heißt partielle Summe.
$(S_n)$ heißt unendliche Reihe.

schriebweise : $(S_n)^\infty =$ bsw 
$(S_n)$ $$ \bigg( \sum_{l=0}^{n} {a_l} \bigg)$$ bzw
 $$ \bigg( \sum_{l=0}^{\infty} {a_l} \bigg)$$  
\end{definition}

\begin{remark}
Reihen sind spezielle Folgen , alle konvergent oder divergent. 
\end{remark}

\begin{definition}
Für eine konvergente Reihen wird der Grenzwert auch wert der Reihe genannt.\\
Schreibweise :  $\lim\limits_{n \rightarrow \infty}{S_n}= $
$$\lim\limits_{n \rightarrow \infty}{ \sum_{k=0}^{n} {a_k} }  $$ 
bzw 
$$ \sum_{k=0}^{\infty} {a_k}  $$
\end{definition}
 
\begin{example}
geometrische Reihe $$ \sum_{k=0}^{\infty} {q^k} $$
ist für $|q|<1$ konvergent . wert der Reihe für $|q|<1$ : 
$$ \sum_{k=0}^{\infty} {q^k}= \frac{1}{1-q}$$ für $|q| < 1 $ 
\end{example}






\begin{example}
\begin{equation}
\begin{aligned}
0,4 \overline{3} = \frac{3}{4} + \frac{3}{100} + \frac{3}{10000}+ \dots \\
\frac{4}{10} + \frac{3}{100}(\frac{1}{10})^0 + \frac{1}{10} + \frac{3}{10^2} + \dots \\
=\frac{4}{10} + \frac{3}{100} \times \frac{1}{1-\frac{1}{10}}\\
= \frac{4}{10}+ \frac{1}{30} = \frac{12+1}{30}= \frac{13}{30}
\end{aligned}
\end{equation}
\end{example}

\begin{example}
\begin{equation}
\begin{aligned}
\sum_{K=1}^\infty{\frac{1}{k}} \text{ist divergent , denn }\\
\lim_{n \to \infty} \sum_{K=1}^n{\frac{1}{k}} \text{ex. nicht ! }\\
S_n = \frac{1}{1} + \frac{1}{2} + \big(\frac{1}{3} + \frac{1}{4} \big)+
\big(\frac{1}{5} + \frac{1}{6} + \frac{1}{7} + \frac{1}{8} \big) 
+ \frac{1}{9} \dots + \frac{1}{16} + \dots \frac{1}{n} \\
 > 1 + \frac{1}{2} \big(\frac{1}{4} + \frac{1}{4} \big) + 
 \big(\frac{1}{8}+ \frac{1}{8}+ \frac{1}{8}+ \frac{1}{8}+ \big)+ \\
 \big( \frac{1}{10} + \dots + \dots + \frac{1}{10} \big)+ \dots + \frac{1}{n}= 1+ \frac{1}{2} + \frac{1}{2} + \frac{1}{2} + \frac{1}{2}+ \dots + \frac{1}{n}\\
 \Rightarrow \lim_{n \to \infty}S_n=\infty
\end{aligned}
\end{equation}
\end{example}

\section{Alleemiene Reihen}
\begin{lemma}
\begin{equation}
\begin{aligned}
\sum_{k=1}^\infty {\frac{1}{k^x}} \text{x fest}\\
\text{falls:}\\
x > 1 \Rightarrow konvergent\\
x \leq q \Rightarrow Divergent\\
\end{aligned}
\end{equation}
\end{lemma}


\begin{proof}{mit Monotoniekriterium}
\begin{equation}
(1)\bigg(\sum_{k=1}^n {\frac{1}{k^2}} \bigg) \text{\quad Monoton (wachsend)} 
\end{equation}
%fellend
\begin{equation}
(2)(\bigg(\sum_{k=1}^n {\frac{1}{k^2}} \bigg)) \text{ \quad ist beschränkt}
\end{equation}  \\




\end{proof}
\section{Rechnenreglen für Regeln:}
Konvergenden Reige kann man addieren , subtrahieren, mit einem Skaler multiplizieren  wie endlichen Summen \\ \underline{ABER:} \\
Das gilt im Allgemein nicht für das multiplizieren
\section{Vorlesung 3}

\begin{example}

\end{example}
\begin{equation}
    \lim\limits_{n \rightarrow \infty}{\frac{11+1}{9-n}}\quad ? \\
    \\\quad X_n = \frac{11+1}{9-n}=\frac{n}{n} \frac{\frac{11}{n}+1}{\frac{9}{n}-1}
\end{equation}

\begin{equation}
    \lim\limits_{n \rightarrow \infty}{\bigg(\frac{11}{n}+1\bigg)}=1
\end{equation}

\begin{equation}
    \lim\limits_{n \rightarrow \infty}{\bigg(\frac{9}{n}+1\bigg)}=-1
\end{equation}

\begin{equation}
    \lim\limits_{n \rightarrow \infty}{(X_n)}= \frac{1}{-1}=-1
\end{equation}

\begin{lemma}
    Seien $(x_n)=(y_n)$ Folgen auf $\lim\limits_{n \rightarrow \infty}{(x_n)}= \lim\limits_{n \rightarrow \infty}{(y_n)}= a$ und es gelte
    $x_n \leq z_n \leq y_n$ für "fest alle " $n \in \mathbb{N}$\\

    Dann gilt für die Folge $(Z_n) \lim\limits_{n \rightarrow \infty}{(z_n)}=a$
\end{lemma}

\begin{example}
    Ist die Folge $(-1)^n\frac{1}{n})$ konvergent ?\\

    \[ - \frac{1}{n} \leq(-1)^n(\frac{1}{n}) \leq 1 \frac{1}{n}\]

    \[ \lim\limits_{n \rightarrow \infty}{- \big(\frac{1}{n} \big)}= -1 \]
    \[ \lim\limits_{n \rightarrow \infty}{ \big(\frac{1}{n} \big)}= 0 \Rightarrow \lim\limits_{n \rightarrow \infty}{(-1)^n \frac{1}{n}}= 0
    \]
\end{example}

\newpage

\begin{example}
    \begin{equation}
        \begin{aligned}
            x_n \leq  = \frac{a^n}{n!} = \frac{a}{n} \times \frac{a^{a-1}}{n-1!} %
        \end{aligned}
    \end{equation}\\

    denn $ x_n = 0 \leq \frac{a_n}{n!} \leq y_n$
    , gesucht! $\underbrace{y_n}_{\lim\limits_{n \rightarrow \infty}{y_n}=0}$  für hinreichend großes n.

    \begin{equation}     
    \begin{aligned}   
            \frac{a^n}{n!} = \frac{a}{n} \times \frac{a^{n-1}}{(n-1)!} \\ \leq
            \frac{1}{2} \times
            \frac{a^{n-1}}{(n-1)!} \\ =
            \frac{1}{2} \times
            \frac{a}{(n-1)} \times
            \frac{a^{n-2}}{(n-2)!} \\ \leq
            \frac{1}{2} \times
            \frac{1}{2} \times
            \frac{a^{n-2}}{(n-2)!} \\ \leq
            \frac{1}{2} \times
            \frac{1}{2} \times
            \frac{1}{2} \times
            \frac{a^{n-3}}{(n-3)!}\\
            %
            y_n = (\frac{1}{2})^{n-k} \times \frac{a^k}{k!} \quad \text{k ist fest}
        \end{aligned}
    \end{equation}\\

    {Es gilt} $\frac{a^n}{n!} \leq y_n$  für hinreichend großes n und
    $\lim\limits_{n \rightarrow \infty}{(y_n)}$ \\

    \begin{equation}
        \begin{aligned}
            &=
            \lim\limits_{n \rightarrow \infty}{(\frac{1}{2})^{n-k}} \times
            \underbrace{\frac{a^l}{k!}}_{Konst} \\
            &=
            \lim\limits_{n \rightarrow \infty}{(\frac{1}{2})^{n}} \times
            \underbrace{\lim\limits_{n \rightarrow \infty}{(\frac{1}{2})^{-k}}}_{\in \mathbb{R}} \times
            \underbrace{\lim\limits_{n \rightarrow \infty}{(\frac{a^k}{k!})}}_{\in \mathbb{R}} \\
            &= 0 . (\frac{1}{2})^{-k} \times \frac{a^k}{k!}=0
            \\
        \end{aligned}
    \end{equation}
\end{example}



\newpage

\section{Grenzwerte rekursive definierte Folgen:}

man kann oft durch lösen "Fixpunktgleichung" berechnen.\\
$x_0 \quad , x_n+1 = ln(x_n)$

\begin{example}
\[(x_n) \quad x_0 = \frac{7}{5} \quad,\quad x_n+1= \frac{1}{3}(x_n^2+2)  \]

Ü $(x_n)$ ist monoton fallend , beschränkt , konvergent . 

\[\lim\limits_{n \rightarrow \infty}{x_n}=a \quad,\quad 
\lim\limits_{n \rightarrow \infty}{x_n+1}=a \]

\begin{equation*}
\begin{aligned}
\lim\limits_{n \rightarrow \infty}{x_{(n+1)}} 
= \linebreak  
\lim\limits_{n \rightarrow \infty}{\frac{1}{3}(x_n^2 + 2)
\frac{1}{3}} \lim\limits_{n \rightarrow \infty}{(x_n^2 + 2)}
=
\frac{1}{3} (\lim\limits_{n \rightarrow \infty}{(x_n))^2 + 2)}
\end{aligned}
\end{equation*}
\end{example}

\subsubsection{Fixpunktgleichung }
 $a = \frac{1}{3}(a^2 + 2) $  , gesucht = a
 
\[ 3a = a^2 +2 \Leftrightarrow a^2 -3a+2 = 0 \] \\
\[ \Leftrightarrow a_{1/2} = \frac{3}{2} \pm \sqrt{\frac{9}{4}-\frac{8}{4}}= \frac{3}{2} \pm \frac{1}{2}\]
Lösung:  $a_1 = 2$ (keine Lösung),  $a_2 =1 $

\begin{example}{$(x_n)$ mit $(x_0) = c \in \mathbb{R} , c  $ fest $x_{n+1}= \frac{1}{2}(x_n+\frac{c}{x_n})$ }\\
(1) $(x_n)$ beschränkt \checkmark\\
(2) $(x_n)$ Monoton \checkmark\\
Also $(x_n)$ konvergent \\
Sei $\lim\limits_{n \rightarrow \infty}{x_n}= a $. 
Dann $\underbrace{\lim\limits_{n \rightarrow \infty}{x_{n-1}}= }_{a}$ $\lim\limits_{n \rightarrow \infty}{\frac{1}{2}}(x_n) + \frac{c}{x_n} = \frac{1}{2}(a + \frac{a}{c})= a \\
 \Leftrightarrow 2a = a + \frac{c}{a} \Leftrightarrow a = \frac{c}{a } \Leftrightarrow a^2 = c \Leftrightarrow a = \sqrt{c}$
\end{example}

\begin{remark}
Der Nachweis der konvergent der rekursiv definierte Folge darf nicht weggelassen werden, denn Z.B $x_0=2$ , $x_n+1=x_n^2$ \quad \quad \quad 2 , 4 ,16 ,256 , $\dots $ divergent gegen + $\infty$  \\

Annahme: $\lim\limits_{n \rightarrow \infty}{x_n}= a $ 
$\underbrace{\quad \lim\limits_{n \rightarrow \infty}{x_{n+1}}}_{a}$ = 
$\underbrace{\lim\limits_{n \rightarrow \infty}{x_n^2}}_{a} \Rightarrow a \in \{ 0,1 \}$
\end{remark}  

%new 
 
\newpage
\section{Reihen :}
\begin{definition}
Sei $(a_n)$ eine reellefolge (komplexwertig) Folge\\
$$\sum_{k = 0}^{n} {a_k} = a_a , a_1, \dots , a_n , $$
heißt n-k heißt partielle Summe.
$(S_n)$ heißt unendliche Reihe.

schriebweise : $(S_n)^\infty =$ bsw 
$(S_n)$ $$ \bigg( \sum_{l=0}^{n} {a_l} \bigg)$$ bzw
 $$ \bigg( \sum_{l=0}^{\infty} {a_l} \bigg)$$  
\end{definition}

\begin{remark}
Reihen sind spezielle Folgen , alle konvergent oder divergent. 
\end{remark}

\begin{definition}
Für eine konvergente Reihen wird der Grenzwert auch wert der Reihe genannt.\\

\begin{schreibweise}
 :  $\lim\limits_{n \rightarrow \infty}{S_n}= $
$$\lim\limits_{n \rightarrow \infty}{ \sum_{k=0}^{n} {a_k} }  $$ 
bzw 
$$ \sum_{k=0}^{\infty} {a_k}  $$
\end{schreibweise}

\end{definition}

\begin{example}{Teleskopreihe} 
\begin{equation}
\begin{aligned}
\sum_{k=1}^\infty(\frac{1}{k}-\frac{1}{k+1}) \text{in Grenzwert der Reihe ist}\\ 
\sum_{k=1}^\infty(\frac{1}{k}-\frac{1}{k-1})=1 \\
\lim_{n \to \infty}{S_n} = \lim_{n \to \infty}{\sum_{k=1}^n(\frac{1}{k}- \frac{1}{k-1}) }\\
= lim_{n \to \infty}{(\frac{-1}{2})+\frac{1}{2}(\frac{1}{3} + \frac{1}{3})(-\frac{1}{4})+) \dots +(\frac{1}{n})-\frac{1}{n+1}}\\
= lim_{n \in \infty}{(1- \frac{1}{n+1})}= 1-0 =1 
\end{aligned}
\end{equation}
\end{example}

\begin{example}
geometrische Reihe $ \sum_{k=0}^{\infty} {q^k} $
ist für $$ {|q|} < 1 $$ konvergent . wert der Reihe für $ {|q|} < 1 $  $\sum_{k=0}^{\infty} {q^k}= \frac{1}{1-q} $ für 
$ {|q|} < 1 $  konvergent , werte der Reihe für 
$$ {|q|} <1 : \sum_{k=0}^n{q^k}= \dots $$
\begin{equation}
\begin{aligned}
S_n = q^0 + q^1 + \dots + q^n | *q \\
-q S_n = q^1 + q^2 + \dots + q^{n+1} \\
(1-q)S_n=q^0 - q^n+1 \\
S_n = \frac{1-q^{n+1}}{1-q} = \frac{1}{1-q}(1-q)^{n+1}\\
\Rightarrow lim_{n \to \infty}{S_n} = \frac{1}{1-q} \times 
\lim_{n \to \infty}{((1-q)^{n+1})}\\
=\frac{1}{1-q}(1- \lim_{n \to \infty}{q^{n+1}})
\end{aligned}
\end{equation}
\subsection{Rechnen für Reihen}
konvergent Reihe kann man addieren oder subtrahieren mit einem Skalar multiplizieren
wie endliche Summen.
aber das gilt im Allgemein nicht für das Multiplizieren  
\end{example}
%%
%% Author: mahmoud
%% 19/04/18
%%


%fehlend ein Beispie mit g=A 
%2A=1^2+(1/2)^2

\chapter{Vorlesung 4}

\section{Reihen}
\begin{example}{Zur geometrischen Reihen}\\
gesucht : A
\[2A = 1^2 + (\frac{1}{2})^2+(\frac{1}{4})^2 + \dots + (\frac{1}{k})^2 + ... \]\\
\[= (\frac{1}{4})^0 + (\frac{1}{4})^1+ (\frac{1}{4})^2+(\frac{1}{2^2})^3+(\frac{1}{2^2})^k+ \dots \]\\
\[9 = \frac{1}{4} = \frac{1}{1- \frac{1}{4}} = \frac{1}{\frac{3}{4}} = \frac{4}{3}= 2A \Rightarrow A = \frac{2}{3}\] 
\end{example}

\begin{example}
\begin{equation}
\begin{aligned}
0,4 \overline{3} = \frac{3}{4} + \frac{3}{100} + \frac{3}{10000}+ \dots \\
\frac{4}{10} + \frac{3}{100}(\frac{1}{10})^0 + \frac{1}{10} + \frac{3}{10^2} + \dots \\
=\frac{4}{10} + \frac{3}{100} \times \frac{1}{1-\frac{1}{10}}\\
= \frac{4}{10}+ \frac{1}{30} = \frac{12+1}{30}= \frac{13}{30}
\end{aligned}
\end{equation}
wenn $0,4 \overline{3}$ erlaubt wäre, dann,\\
\[\frac{4}{10} + \frac{9}{100} \times \frac{10}{9} = \frac{4}{10} + \frac{1}{10} = \frac{5}{10} = \frac{1}{2} = 0.5  \]
\end{example}

\newpage
\begin{example}
	\[\sum_{R=1}^{\infty} \frac{1}{K} \text{ ist divergent ,denn  } \lim\limits_{\infty} \sum_{K=1}^{n} \frac{1}{k} \text{ ex. nicht } \]
	
	\[\frac{1}{1}+\frac{1}{2}+\frac{1}{3}+\frac{1}{4}+\frac{1}{5}+\frac{1}{6}+\frac{1}{7}+\frac{1}{8}+\frac{1}{9}+\dots+\frac{1}{16}+\dots+\frac{1}{n}\]
	
	\[ > 1+\frac{1}{2}+  \underbrace{\frac{1}{4}+\frac{1}{4}}+\underbrace{\frac{1}{8}+\frac{1}{8}+\frac{1}{8}+\frac{1}{8}}+\frac{1}{10}+\frac{1}{16}+\dots+\frac{1}{n}\] 
	
	\[1+\frac{1}{2} + \qquad \frac{1}{2}+   \quad  \qquad \frac{1}{2}  + \dots + \frac{1}{n}  \rightarrow \lim\limits_{n\rightarrow \infty }s_n = \infty \]
	
\end{example}
\section{Allgemeine harmonische Reihe}

				\begin{equation*}
				\begin{split}
				 \sum_{K=1}^{\infty} \frac{1}{k^\alpha} \quad( \infty \text{fest}) \qquad
				& \alpha > 1 \rightarrow \mathbb{R}  \\
				& \alpha \leqq \rightarrow dev
				\end{split}
				\end{equation*}
 \begin{example}
 	\[ \sum_{k=1}^{\infty} \frac{1}{k^2} \qquad \text{ist konvergent}\]
 	
\begin{beweis}{mit Monotoniekriterium für Folge}  
		\[ \text{Reihe ist konvergent} \left\{ \begin{array}{ll}
	   (1) & \mbox{ $\sumIn \frac{1}{k^2} $ \quad ist monoton wachsend};\\
	   (2) & \mbox{$\sumIn \fracKetwo $ \text{ist beschränkt}}.\end{array} \right. \] 
	   	
	   \[ \sumIin \fracKetwo = \frac{1}{1}+\frac{1}{2^2}+\frac{1}{2^2}+\frac{1}{4^2}+\frac{1}{5^2}+  \dots + \frac{1}{8^2}  \]
	   \[ <  1 + \frac{1}{4}+\underbrace{\frac{1}{2^2}+\frac{1}{2^2}}_{2. \frac{1}{4}}+\underbrace{\frac{1}{4^2}+\dots + \frac{1}{4^2}}_{4.\frac{1}{4^2}}+\]
	   
	   \[ 1+\frac{1}{4}+\frac{1}{2}.1+\underbrace{\frac{1}{4}}_{(\frac{1}{2})^2}+\underbrace{\frac{1}{8}}_{(\frac{1}{2})^3} =1+\frac{1}{4}+\frac{\frac{9}{4} }{1-\frac{1}{2}-1} \]
	   
 \end{beweis}
 \end{example}
%photo auf der Tafel
\section{Expotentiale Reihe}

\[ \sumOin \frac{1}{k!} = \limNin ( 1-\frac{1}{n})^n =:e \text{ist konvergent} \]

 \pgfkeys{/forest,
	tria/.append style={ellipse, draw},
}
	\begin{forest}
	for tree={l+=0.5cm, s sep+=1cm}
	[Konvergentkreterium für Reihen 
	[	für Folge ]
	[Kreterien für absolute konvergenz [konvergent Reihen] ]
	[Hauptkriterium ]
	]
	\end{forest}

\section{Hauptkriterium}  
$$\sum_{k=0}^{\infty}{a_k} \quad \text{konvergent}
 \Rightarrow (a_k) \text{Nullfolge.}$$ \\
$$\lim\limits_{k \rightarrow \infty}{a_k} \neq 0 \Rightarrow  \sum_{k=0}^{\infty}{a_k} \quad \underbrace{null konvergent }_{divergend}$$\\
oder
			$$\lim\limits_{k \rightarrow -\infty}{ a_k} \quad ex.null$$

						\begin{example}
						
						\[ \sumIin \frac{3k^2+1}{4k^2-1} \quad divergend, \quad aber \quad \sumIin \fracKone \quad divergend \quad \text{und $\fracKone$ Null folge }  \]
						

						\end{example}
\begin{beweis}
\[  \sumOin a_k konv. \Rightarrow \underbrace{(a_k Nullfolge) }_{\limKin a_k=0}  \] 
\[s_n =\sumOn a_k , s_{n+1} =\sum_{k=0}^{n+1} \qquad s_{n+1}=s_n + a_{n+1} \]
\[s= \limNin s_n= \limNin s_{n+1} \qquad \limNin a_{n+1} = \limNin s_{n+1} - \limNin s_n = s-s=0 \]
\end{beweis}

\newpage
\section{Kriterium für Alternierende  Reihe}
\begin{beweis}{Alternierende }
	$\sumOin (-1)^k \fracKone$ ist konvergent\\
	$\sumOin (-1)^k a_k$\\
	wobei $(a_k)$ einer Streng monoton fallend Nullfolge mit $a_k \geqq 0 $ \\
	$\Rightarrow$ Die Reihe ist konvergent. Also $\sumOin (-1)^k \fracKone$ ist konvergent.

\end{beweis}
\begin{definition}[Reihe]
	
	\text{Reihe} $\sumOin  a_k$  \text{heiß absolute konvergent wenn} $\sumOin $ $|a_k|$ \text{konvergent ist.
	}
\end{definition}
\begin{example}

\[	\sumIin (-1)^k \fracKone \text{ ist konvergent , aber nicht absolote konvergent } \]
\end{example}

\begin{example}
\[ \sumIin (-1)^k \fracKetwo  \text{ ist kovergend und abslote konvergent } \]
\end{example}

\begin{theorem}
Reihe $\sumOin$ $a_k$ abslot konvergent $\Rightarrow$ Reihe $\sumOin$ $a_k$ ist kovergend	
\end{theorem}

\begin{remark}	
Absolute konvergente Reihe kann man multiplizieren wie endliche summen d Reihen null
\end{remark}

\section{Quotionkriterium (QK):}
für endliche Konvergenz\\
\[\limKin |\frac{a_{k+1}}{a_k}|\] \\$< 1 \Rightarrow \sumOin a_k $ in absolut konvergent \\
$>1 \Rightarrow $ ist divergent\\
$=1 $ Kriterium ist nicht anwendbar 
\section{Wurzel kriterium : WK }
für (absolute) konvergent 
\\ $\limKin \sqrt[k]{|a_k|}$\\
$<1 \Rightarrow \sumOin a_k$ in (absolute) konvergent\\
$>1 \Rightarrow$ divergent\\
$=1 $ Kriterium ist nicht anwendbar

\begin{example}[QK]
	 \begin{equation*}
		 \begin{split}
	    	\sumOin \dfrac{1}{k!}
	 	   \limKin {|d\frac{\frac{1}{(k+1)!}}{\frac{1}{k!}}|} &= \limKin \frac{k!}{(k+1)!} \\
													   	   &=\limNin \frac{1}{k+1}   \\
													   	   &=0<1 \Rightarrow  Reihe \quad als \quad konv. 
		    \end{split}	 
		 \end{equation*}     
	
\end{example}
\begin{example}[WK]
	\begin{equation*}
	\begin{split}
	\limKin \sqrt[k]{\frac{1}{k!}}= \limKin \frac{\sqrt[k]{1}}{\sqrt[k]{k!}}=\frac{1}{\limKin \sqrt[k]{k!}}=0\\ <1 \\
	\Rightarrow Reihe \quad als \quad konv.
	\end{split}	 
	\end{equation*}     
	
\end{example}
\section{Vorlesung 5}

Zusammenfassung :\\
Folgen / Reihen / Konvergenz ? / Grenzwert ? \\ 
Neu : Funktionen\\ 
Approximation von Funktionen \\
Potenzreihen\\
Taylorreihen\\
fourierreihen\\
Näherungsweise Berechnung
\begin{definition}
$ f : \mathbb{D} \rightarrow \mathbb{R} $
heißt reelle Funktion in einer reellen veränderlichen  

\end{definition}

\begin{remark}[Definitionsbereich]
Bild von $f$ \[ f(D)= \{ f(x) \quad | \quad x \in D \} \] \\
Graph von $f$ \[ Graph(f)= \{( x \quad |\quad f(x)) \quad | \quad x \in D \} \] \\

\end{remark}   

\begin{definition}
Sei $f : D \rightarrow \mathbb{R} , D \subseteq \mathbb{R} , a \in D \quad $\\
$f$ heißt in a stetig , wenn gilt :\\
$\forall (X_n) : X_n \in D$ und 
$\limNin{f(x_n)} = f(a)$ für alle Folgen $(x_n)$\\
Die Folgenglieder sollen in Definitionsbereich liegen (Die in Definitionsbereich liegen können und den Grenzwert a haben)
\end{definition}

* Ich weiß , dass $f(x_n)$ existiert $(f(x_n)\quad ex.)$\\
 Folge $f(x_n)ex.$ , soll einen Grenzwert besitzen.$\checkmark$\\
$f(\limNin{x_n})\checkmark \checkmark$

\begin{remark}
\[ \limNin {f(x_n)= f(\limNin{x_n})} \]
$\bigstar$ Grenzwertbildung und Funktion Wertberechnung sind bei stetig Funktion in der Reihenfolge vertauschbar !  
\end{remark} 

\begin{rechnen}
$$\lim_{x \to a} f(x)$$ \\
$\textbf{d.h }$ für jede Folge $x_n$ , die gegen a konvergiert , konvergiert die Folge der Funktionierte gegen $f(a)$.
\end{rechnen}

\begin{remark}
$f$ stetig in a $\Leftrightarrow$ \\
1) $f(a)$ und \\
2) $lim_{x \to a}{f(x)}$ ex. und \\
3) Grenzwert = Funktionswert 
$\lim\limits_{x \rightarrow a}{f(x)} = f(a)$ \\


\begin{tikzpicture}[scale=1.1],

\begin{axis}[xlabel=$x$,
ylabel= $\delta$,
height=6cm,
width=10cm,
ymax=11,
ymin=0,
xmin=0,
xmax=6.5,
axis y line=left,
axis x line=bottom,
ytick={1,...,10},
yticklabels={ ,,$f(a)$ ,,},
xtick={1,...,5},
xticklabels={$x_0$,$x_2$,$a$,$x_3$,$x_1$,}
]
  \addplot[dashed] coordinates {(0,1.5) (1,1.5)};
  \addplot[dashed] coordinates {(1,0) (1,1.5)};
  \addplot[dashed] coordinates {(0,2) (2,2)};
  \addplot[dashed] coordinates {(2,0) (2,2)};
  \addplot[blue,dashed] coordinates {(0,3) (3,3)};
  \addplot[blue,dashed] coordinates {(3,0) (3,3)};
  \addplot[dashed] coordinates {(0,5) (4,5)};
  \addplot[dashed] coordinates {(4,0) (4,5)};
  \addplot[dashed] coordinates {(0,9) (5,9)};
  \addplot[dashed] coordinates {(5,0) (5,9)};

    \addplot[black,domain=1:9]  { 2^(\x-2)+1  }node {f};
    


\end{axis}
\end{tikzpicture}


\end{remark}


\begin{example}
\begin{itemize}
1)
\begin{align*}
f(x) = \frac{x^2 - 1}{x - 1} =   
\frac{(x-1)(x+1)}{(x-1)}\\ 
\end{align*}

Ist $f(x)$ stetig in $a = 1$ ?\\

a) \quad $f(1)$ ex ? nein , d.h $f$ ist in $a = 1$ nicht stetig\\ 
\\
b) $$ \lim_{x \to 1}f(x) = \lim_{x \to 1}{\frac{(x-1)(x+1)}{x-1}} = \quad ? $$

Sei $(x_n)$ eine beliebige Folge und $x_n \in D(f)$ und $\lim_{x\to \infty}(x_n)=1$
\end{itemize}
\begin{gather*}
\lim_{n \to \infty }{f(x_n)} = \lim_{n \to \infty }
{\frac{(x-1)(x+1)}{(x-1)}} = \lim_{n \to \infty }{(x_n + 1) } = \lim_{n \to \infty }{x_n} + \lim_{n \to \infty }{1} = 1 + 1 = 2 
\end{gather*}

d.h Grenzwert ex. (und es ist 2 ).

\begin{tikzpicture}[scale=1.1],

\begin{axis}[xlabel=$x$,
ylabel= $\delta$,
height=6cm,
width=5cm,
ymax=3,
ymin=0,
xmin=0,
xmax=2,
axis y line=left,
axis x line=bottom,
ytick={1,...,10},
yticklabels={ 1,2},
xtick={1,...,5},
xticklabels={$1$}
]
    \addplot[black,domain=0:1]{x+1}node[above, sloped, pos = 0.3] {g};
    \addplot[black,domain=1:2]{x+1}node[above, sloped, pos = 0.65] {f};
    \draw [blue,  -stealth    ] (3,1) -- (1.1,2) node [right] {Lücke};
    \addplot[mark=*,fill=white] coordinates {(1,2)};

\end{axis}
\end{tikzpicture}

Man sagt , $f$ hat an der stelle 1 eine Lücke.
\end{example}

\begin{example}
(2) $$ f(x)=\frac{1}{x} \quad ,\quad  a = 0 $$
\begin{tikzpicture}[scale=1.1],
\begin{axis}[xlabel=$x$,
ylabel= $y$,
height=6cm,
width=5cm,
ymax=3,
ymin=-3,
xmin=-3,
xmax=3,
axis y line=center,
axis x line=center,
]
\addplot[domain= -3:-0.01] {1/x};
\addplot[domain= 0.01:3] {1/x};


\end{axis}
\end{tikzpicture}


(i) betrachte $? \lim\limits_{x \rightarrow 0^-} {f(x):}$ d.h wir
betrachten alle Folgen $(x_n)$ \\


\begin{align*}
X_n \in D , X_n \leq 0 \limNin (x_n) = 0 \\
\limNin f(x_n) = \limNin \frac{1}{x_n} \\
= \frac{\limNin {1}}{\lim\limits_{n \rightarrow - \infty}{x_n}} 
= \frac{1}{ \lim\limits_{n \rightarrow - \infty}{x_n}} = - \infty\\
\\
\text{d.h} \quad \lim_{x \to 0^-}{f(x)} \text{ex .nicht}
\end{align*}
\\
(ii) Betrachte $\lim\limits_{n \rightarrow + 0}{f(x_n)}$ , ex .nicht\\
\newpage
(3)\\
\begin{tikzpicture}[scale=1.1],
\begin{axis}[xlabel=$x$,
ylabel= $y$,
height=6cm,
width=5cm,
ymax=3,
ymin=-3,
xmin=-3,
xmax=3,
axis y line=center,
axis x line=center,
]
\addplot[domain= -3:-0.01] {1/x};
\addplot[domain= 0.01:3] {1};


\end{axis}
\end{tikzpicture}


$$
f(x) = \left\{\begin{array}{lr}
        1 , & x \geq 0\\
        \frac{1}{x} , & x < 0 
        \end{array}\right\} a = 0 \quad , \quad f(0) = 1 \quad \text{ex.}  
$$

$$ \lim\limits_{x \rightarrow 0^+}{f(x)} = 1 , 
\lim\limits_{x \rightarrow  0^-}{f(x)}= - \infty \quad \text{ex. nicht} $$ \\
 
(4)\\

\begin{tikzpicture}[scale=1.1],
\begin{axis}[xlabel=$x$,
ylabel= $y$,
height=6cm,
width=5cm,
ymax=3,
ymin=-3,
xmin=-3,
xmax=3,
axis y line=center,
axis x line=center,
]
\addplot[domain= -3:0] {-1};
\addplot[domain= 0:3] {1} node[above ,pos=0.8] {f(x)};
\addplot[mark=*,fill=white] coordinates {(0,-1)};
  \draw [decorate, decoration={brace,amplitude=5pt,raise=4pt,mirror}] (0,1) -- (0,-1) 
node [midway, xshift=-0.5mm, yshift=1mm,auto, swap, outer sep=9pt,font=\tiny]{Sprung};
\end{axis}
\end{tikzpicture}

\begin{definition}[ sgn(x)]
Die Vorzeichenfunktion oder \textbf{Signumfunktion} (von lateinisch signum ‚Zeichen‘) ist in der Mathematik eine Funktion, die einer reellen oder komplexen Zahl ihr Vorzeichen zuordnet.\\
Die reelle Signumfunktion bildet von der Menge der reellen Zahlen in die Menge \{-1,0,1\} ab und wird in der Regel wie folgt definiert:
$$ f(x) = \underbrace{sgn(x)}_{sprung}  =  \left\{\begin{array}{lr}
+ 1 , & x \geq 0\\
0 , & x = 0 \\
-1 , & x < 0 
        \end{array}\right\}$$ 
\end{definition}
$$
\neq \left\{\begin{array}{lr}
  \lim\limits_{x \rightarrow 0^-}{f(x)} = -1 \quad \text{ex.} \\
        \lim\limits_{x \rightarrow 0^+}{f(x)} = 1 \quad \text{ex.} 
        \end{array}\right\} \lim\limits_{x \rightarrow 0}{f(x)} \quad \text{ex. nicht , O heißt Sprungstelle}
$$  
\end{example}
\newpage
\begin{definition}
\begin{align*}
f : \rightarrow \mathbb{R} , \quad D \subseteq \mathbb{R} \quad \text{heißt \textbf{stetig} , wenn f für alle } a \in D \quad \textbf{stetig} 
\end{align*}
\end{definition}
\begin{example}
elementare Funktionen und deren Verfügungen sind stetig auf dem gesamten Definitionsbereich. \\
$\textbf{Z.B}$ \\
Polynomfunktion , rationale Funktionen, Winkelfunktionen , Potenzfunktionen , Wurzelfunktionen , Exponentialfunktionen und Logarithmusfunktion.
     
\end{example}

\begin{example}
%fehlende Skizze 
\begin{align*}
f : D \rightarrow \mathbb{R} : x \rightarrow \frac{1}{x} = x^{-1} \text{ist stetig auf dem gesamten  Defintionsbereich } D = \mathbb{R}   \backslash \{0\}  
\end{align*}

\begin{tikzpicture}[scale=1],

\begin{axis}[xlabel=$x$,
ylabel= $y$,
height=12cm,
width=17.5cm,
ymax=5,
ymin=-3,
xmin=-3,
xmax=3,
axis y line=center,
axis x line=center,
]
\addplot[domain= -3:-0.01] {1/x};
\addplot[domain= 0.01:3] {1/x};
\draw [red,  -stealth    ] (-0.5,0.7) -- (-0.1,0.1) node [left,pos=0] {kein intervall im  Definition Bereich};
\addplot[mark=*,fill=white] coordinates {(0,0)};
\addplot[ very thick,mark options={solid},blue,domain= 0.8:2.3] {1/x};
\addplot[ very thick,mark options={solid},blue,domain= 0.8:2.3] {0};
\draw [blue,  -stealth    ] (0.5,-2) -- (1.2,-0.6) node [below,pos=0] { intervall };

\end{axis}
\end{tikzpicture}

\end{example}
\newpage
\begin{beweis}
Sei $ a \in D = \mathbb{R} \backslash \{0 \}$ (d.h a $\neq 0$) 
\begin{align*}
f(a)  &= \frac{1}{a} \tag{1} \\ 
\lim_{n \to \infty}{f(x)}  &=
\lim_{n \to \infty}{\frac{1}{x}} \tag{2}
\end{align*} 

\begin{align*}
\text{Sei} \quad x_n \quad \text{eine beliebige Folge und }\quad x_n \in \underbrace{D}_{\mathbb{R}\backslash \{ 0 \}}  \quad \text{und}\quad \lim_{n \to \infty}{x_n}= a
\end{align*}
\begin{align*}
\lim_{n \to \infty}{f(x_n)}
&=  \lim_{n \to \infty}{\frac{1}{x_1}} \\
&= \frac{\lim\limits_{n \rightarrow \infty}{1}}{\lim\limits_{n \rightarrow \infty}{x_2}} \\
&= \frac{1}{a} \in \mathbb{R} \quad \text{ex.}
\end{align*}
\end{beweis}  
\subsection{Rechnenregln für Funktionen (GWS anwenden)}
%fehlende Skizze 
\begin{gather*} 
\lim\limits_{x \rightarrow \infty}{(f(x) \pm g(x))} =\lim\limits_{x \rightarrow \infty}{f(x)} \pm  \lim\limits_{x \rightarrow \infty}{g(x)} , \text{ wo bei } g(x)\neq 0  \\
\lim\limits_{n \rightarrow \infty}{(f(n) \pm g(n))} =\lim\limits_{n \rightarrow \infty}{f(n)} \pm  \lim\limits_{n \rightarrow \infty}{g(n)} 
\end{gather*}


\begin{theorem}
\begin{gather}
f: D \Rightarrow \mathbb{R} , \quad D \subseteq \mathbb{R} \text{ ist in } a \in D \text{ Stetig }
\Leftrightarrow \forall_{\epsilon} > 0 \quad \exists \delta > 0 : |x-a| < \delta \Rightarrow |f(x)- f(a)|< \epsilon 
\end{gather} 
\end{theorem}
%fehlende Skizze
  
\begin{tikzpicture}[scale=1.2],

\begin{axis}[
height=6cm,
width=10cm,
ymax=11,
ymin=0,
xmin=0,
xmax=6.5,
axis y line=left,
axis x line=bottom,
ytick={1,...,10},
yticklabels={ ,$f(a)-\epsilon$ ,,,,$f(x)$,,,$f(a)+\epsilon$},
xtick={1,...,5},
xticklabels={,,$a-\delta$,$a$,$a+\delta$}
]
\addplot[dashed] coordinates {(0,2) (2,2)};
\addplot[dashed] coordinates {(2,0) (2,2)};
\addplot[blue,dashed] coordinates {(0,3) (3,3)};
\addplot[blue,dashed] coordinates {(3,0) (3,3)};
\addplot[dashed] coordinates {(0,5) (4,5)};
\addplot[dashed] coordinates {(4,0) (4,5)};
\addplot[blue,dashed] coordinates {(0,9) (5,9)};
\addplot[blue,dashed] coordinates {(5,0) (5,9)};

\addplot[black,domain=1:9,name path=A]  { 2^(\x-2)+1  } node at (5.5, 8.5) {f};
\addplot[very thick,blue,name path=B,domain= 3:5] {0};
\addplot[blue,very thick] coordinates {(0,3) (0,9)};
\addplot[gray, pattern=north west lines] fill between[of=A and B, soft clip={domain=3:5}];
\legend{$qeg \; \epsilon$, $gce \; \delta$}
\end{axis}
\end{tikzpicture}
\section{Vorlesung 6}
\[|x-a| < \delta \]
\begin{equation}
\\|x-a| = \begin{cases}
	x-a , \: x-a \geq 0  \\
	-(x-a),\:  x-a<0 \\
	\end{cases}
 = \begin{cases}
x-a , x \geq \\
a-x , x < a \\
\end{cases}	
 \begin{cases}
  x \leq a : x-a < \delta \Rightarrow x < a+\delta \\
  x<a : a-x < \delta \Rightarrow a-\delta < x	\end{cases} \Rightarrow 
\end{equation}


\begin{equation}
\\ \begin{cases}
a \leq x < a+\delta  \\
a+\delta < x < a \\
\end{cases}
\end{equation}
\subsection{Ergebnis}
$|x-a|<\delta \Leftrightarrow a-\delta < x < a+ \delta$ \\
$\Leftrightarrow x \in (a-\delta , a+d ) $ offenes Intervall \\
$|x-a| < \delta$ \\

 %skizze++++++++++++++++++++++++++++++++++++++
 
 x liegt in der $\delta $-Umgebung von a
	
\[	|f(x) -f(a)| < \epsilon  \Leftrightarrow f(x)\text{ liegt in der }\epsilon\text{-umgebung con} f(a)
							 \] \[\Leftrightarrow f(x) \in \left( f(a)-\epsilon , f(a)+ \epsilon \right) \epsilon > o	 \]
 
 %skizze ++++++++++++++++++++++++
 
 \begin{equation}
 	I(\frac{1}{e})= I(e^-1)= 1.k   \quad \text{rell}
 		I(e^{-n})= I(\underbrace{e^{-1} \dots e^{-1}}_n )= I (e^{-1})+\dots+ I(e^{-1})= k.n 
 		\end{equation}
 		 \begin{equation}
 		\frac{n}{m} \in Q : I ( e^{-\frac{n}{n}})= k. \frac{n}{m}\text{, denn}
 		\end{equation}
 		\begin{equation}
 		kn=I(e^{-n}) = I(e^{-\frac{n}{m} . m})= \underbrace{I ( e^{-\frac{n}{m}} \dots  e^{-\frac{n}{m}} )}_{m}+\dots + I (e^{-\frac{n}{m}})= I(e{-\frac{n}{m}})+ \dots + I(e^{-\frac{n}{m}}) 
 		\end{equation}
 		\begin{equation}
 		r \in \mathbb{R}_+ : I(e^{-r})=?
  		\end{equation}

\begin{gather*}
	\limNin \underbrace{q_n}_{\in \mathbb{Q_+}}= r \\
	I(e^{-r})= I(e^{-\limNin (q_n)})
	= I (e^{\limNin (-\frac{q}{n})})
		\overset{\overset{e \: stetig}{\downarrow}}{=} I(\limNin e^{-q_n})	\overset{\overset{I \: stetig}{\downarrow}}{=} \limNin I(e^{-\frac{q}{n}})	
	= \limNin k.q_n
	=\overbrace{k.\underbrace{\limNin q_n}}_{r}^{k.r}\\
	I(\frac{1}{e})= I(e^{-1})= \frac{1}{k} \text{rell} \\
	I(p)=I(e^{\ln p}) = \underbrace{k}_{>0}(- \ln p) =\underbrace{-k}_{<0} \ln p
\end{gather*}

%skizze +++++++++++++++++++++++++++++++

\begin{example}
	D(x)=  
$	\begin{cases}
		1, x \in \mathbb{Q} \quad \quad (rational)\\		
		0, x \in \mathbb{R} \backslash \mathbb{Q} \quad (irrational) 	
	\end{cases}$\\ \\
\text{stetig für welche a?}\\
\text{1. Fall : a rational} \\ 
\text{2. Fall : a irrational }
\\ \\ a rational: a fest 
\\ sei $\varepsilon = \frac{1}{2}$ , beliebig  $\exists \delta > 0 \forall x \in D: |x-a| < \delta \Rightarrow |D(x) -D(a) | < \frac{1}{2}$ 
Sei $\delta$ beliebig, $\delta$ > 0, $x$ irrational , fest \\
$|x-a|<d \Rightarrow |0-1| =|11= 1 < \frac{1}{2}$, Widerspruch \\
$\Rightarrow D$ ist nicht stetig, für jede $ a \in \mathbb{R}$  \\
Sei $\delta >0$ , beliebig, x rational , fest $|x-a| < \delta \Rightarrow |\underbrace{D(x)}_1 -\underbrace{D(a)}_0| < \frac{1}{2}= \varepsilon \Rightarrow 1 < \frac{1}{2} $ Widerspruch \\
$\Rightarrow$ D ist nicht stetig für jede a $\in \mathbb{R} \backslash \mathbb{Q}$ 
\end{example}
\begin{theorem}
Sei $ f: [a,b] \rightarrow \mathbb{R}$, stetig 
f besetzt in [a,b] ein globale Maximum und ein globales Minimum
\end{theorem}
\begin{remark}
Beide (unklar!)Veränderung sind wichtig
\end{remark}
\begin{remark}
	[a,k]= {x $\in$ $\mathbb{R}$ | a $\leq$ x $\leq$ b}
%skizze +++++++++++++++++++++
%kizze ++++++++++++++++++
\end{remark}
\subsection{Zwischenwertsatz}
\begin{theorem}[ZWS]
Sei $f: [a,b] \rightarrow \mathbb{R}$ stetig , $\frac{x_m}{x_M}$ eine globale Minimale stelle \\ 
	 eine globale Maximalastelle\\	
	 Sei  \^{y}
	 $ \in $[ $f(x_m) , f(x_M) :$ Dann existiert  \^{x} $ \in $ [a,b] mit  \^{y}  = f(\^{x}) 
\end{theorem}
\begin{remark}
	Jeder Zwischenwert wird als Funktionswert angenommen
\end{remark}
\begin{theorem}[Nullstellen]
	Sei $f: $[a,b] $\rightarrow \mathbb{R}$ stetig, $f(a) \times f(b) < 0$
Dann beliebig f in [a,b] eine Nullstelle $x_0$ , d.h. $\exists x_0 \in $ [a,b] : $f(x_0)=0$
	%skizze +++++++++++++++++++  
\end{theorem}
\begin{proof}
%skizze ....
$f(a) < 0 $ und $f(x) > 0$  (analog für $f(a) > 0 , f(b) <0$)\\
\\$(\frac{a_1+b_1}{2})=
\begin{cases}
 \; 0 , \frac{a_1+b_1}{2}\text{ ist die gesamte Nullstelle}\\
< 0 \: ,a_2= \frac{a1+a2}{2} , b2=b1\\
> 0 \:, a_2=a_1 , b_2=\frac{a_1+b_1}{2}
\end{cases}$


 \[\text{usw. } \frac{a_2+b_2}{2}  \text{berechnen}\]
\[ f(..)  \begin{cases} =0 \\ <0 \\ >0 \end{cases}\]
 
\begin{center} 
\begin{forest}
	for tree={l+=0.5cm, s sep+=1cm}
	[Betrachte $(a_n)$ 
	[	Stetigmax ]
	[beschränkt  ]
	]
\end{forest}
$\Rightarrow konvergent$
\end{center}
sei $\underbrace{\limNin a_n=:c}_{ex.}$
\\
sei $\underbrace{\limNin a_n=:c}_{ex.}$\\
$a\leqq \dots \leqq b_2 \leqq b_1 \leqq b$
ex. $\limNin b_n=2$

\begin{align*}
\limNin | a_n - b_n| &= \limNin \frac{|a-b|}{2^{n-1}}\\
  				   &= |a-b| \limNin \frac{1}{2^{ n-1}}\\
  				   &= |a-b|.0 \\
  				   &= 0\end{align*}
 \[ \limNin b_n=c \]
\begin{center}
\begin{forest}
	for tree={l+=0.5cm, s sep+=1cm}
	[Betrachte $(b_n)$ 
	[	Stetigmax ]
	[beschränkt  ]
	]
\end{forest}
$\Rightarrow konvergent$
\end{center}

Falls keine Nullstelle beim bilden von $a_n,b_n$ gefunden wurden\\ \\
\\


	$\begin{rcases}
	f(c) &=f(\limNin a_n) \overset{\overset{f stetig}{\downarrow}}{=}\limNin f(a_n) \geqq 0 \\
	\quad =\\
	f(c)&=f(\limNin b_n)\overset{\overset{f stetig}{\downarrow}}{=} \limNin f(b_n) \leqq 0 \\
	\end{rcases}$
$f(c)=0$
\end{proof}
\section{Vorlesung 7}

\begin{align*}
&	f:D \rightarrow \mathbb{R} , D \subseteq \mathbb{R} , a \notin D &\\
&\underbrace{\lim_{x\to a}}_{x \neq a} f(x)= r \in \mathbb{R}  \Leftrightarrow \forall (x_n) \limNin x_n= a \text{  und  } x_n \in D \\
&\Rightarrow \limNin f(x_n)=r
\end{align*}
\begin{example}

GWS nicht anwendbar $\limXo \overbrace{ x \sin x }^{f(x)} = \limXo x  .\quad \limXo \sin x=0. 0=0$
\end{example}
\begin{remark}
GWS nicht anwendbar $\underbrace{\limXo(x \ \sin \frac{1}{x}}_{f(x)} \underbrace{\limXo}_{0},\limXo \sin \frac{1}{x}$
\end{remark}
\begin{definition}
	Sei $f: (a,b) \to \mathbb{R} , x_0 \in (a,b) $\\
	 $x_0 \in (a,b) \Leftrightarrow  x_0 \in \mathbb{R} $  und  $ a<s_0 < b ( \Leftrightarrow (skizze  not complate))$\\
	$f$ ist in $x_0$  differenzierbar : $\Leftrightarrow f'(x_0) := \ \underbrace{\lim\limits_{x \rightarrow f(x)}}_{f \neq x_0} \frac{f(x) -f(x_0)}{x-x_0}$ existiert  $(f'(x_0) \in \mathbb{R}) $\\
	Falls der Grenzwert ex., nennt man $f'(x_0)$ die erste Ableitung von $f$ in $x_0$.\\
	Existiert $f'(x_0)$ für alle $x_0 \in (a,b)$ , dann nennt man $ f':(a,b) \rightarrow \mathbb{R} \longmapsto f'(x_0)$ die erste Ableitung von $f$.

	

\end{definition}
\begin{example}
	$f(x) =\frac{1}{x}$ auf $(0,r)$ $r\in \mathbb{R}_{>0}$ ,$r$ fest und $x_0 \in (0,)r $, ges: $ f'(x_0)$
	\begin{align*}
			f'(x_0) =
		\underbrace{\lim\limits_{x \to x_0}}_{x \neq x_0} \frac{\frac{1}{x}-\frac{1}{x_0}}{x-x_0}= 
		\underbrace{\lim\limits_{x \to x_0}}_{x \neq x_0} \frac{\frac{x_0-x}{x.x_0}}{x-x_0}=
		\underbrace{\lim\limits_{x \to x_0}}_{x \neq x_0} \frac{(x_0-x)}{x-x_0(x-x_0)}=
		\underbrace{\lim\limits_{x \to x_0}}_{x \neq x_0} \underbrace{(-\frac{1}{x_0})}_{konst.}\frac{1}{x}=
		-\frac{1}{x_0} \underbrace{\lim\limits_{x \to x_0}}_{x \neq x_0} \frac{1}{x}\\
		\overset{\overset{\frac{1}{x} stetig F.}{\downarrow}}{=}
		-\frac{1}{x}. \frac{1}{x}= - \frac{1}{x^2}
	\end{align*}
	\begin{flalign*}
		f' : (0,r) \rightarrow \mathbb{R} : x \longmapsto -\frac{1}{x^2} \text{in die erste Abbildung von } f(x)= \frac{1}{x}
	\end{flalign*}
  

\end{example}
\subsection{tafelwerk}   
$f \qquad \qquad$					 $f'$\\ \\
$x^n \qquad \qquad$				   	$nx^{n-1}$\\
$\downarrow n=-1 \qquad $		$ \downarrow $ \\
\\
$\frac{1}{x} \qquad \qquad$					$-\frac{1}{x^2}$
\begin{theorem}
	$f$ in $x_0$ differenzierbar $\Rightarrow$ $f$ in $x_0$ stetig
\end{theorem}
\begin{proof}
Sei $f$ in $x_0$ d.b $\Rightarrow$ $f'(x_0) = \limXin \frac{f(x) - f(x_0)}{x-x_0 }$ ex.
...
%fehlt den Rest von proof
\end{proof}

\begin{tikzpicture} 
  \newcommand*\funktion[1]{2*sin(0.5*deg(#1)) + 1.5}% dargestellte Funktion
  \newcommand*\ableitung[1]{cos(0.5*deg(#1))}% Ableitung der Funktion
  \newcommand*\tangente[2]{\ableitung{#2}*(#1-#2)+\funktion{#2}}

  \begin{axis}[axis lines=middle,,enlargelimits,
    xlabel=$x$,xlabel style={anchor=north},xtick=\empty,
    ylabel=$y$,ylabel style={anchor=east},ytick=\empty
  ]
  \addplot[domain=-1:10,samples=200]{\funktion{\x}};
  \addplot[domain=4:8]{\tangente{\x}{6}};
  \coordinate (P) at (axis cs:6,{\funktion{6}}) ;
  \coordinate (Q) at (axis cs:0,{\funktion{6}}) ;
  \coordinate (R) at (axis cs:6,0) ;
  \node[coordinate,pin=30:{$(x_0,f(x_0))$}] at (P) {};
  \draw[red,dotted] (P) -- (Q) node[left] {$f(x_0)$} ;
  \draw[red,dotted] (P) -- (R) node[below] {$x_0$} ;
  \end{axis}
\end{tikzpicture}

Die Linie repräsentiert die Tangente ((T)) an den Grenzwert von $f(x_0)$ im Punkt $(x_0 , f(x_0))$

\begin{gather*}
(x) = \dfrac{t(x)-t(x_0)}{x - x_0} = \dfrac{f(x)-f(x_0)}{x - x_0}
\end{gather*}

\subsection{Tangente Gleichung}
\begin{align*}
T : t(x) = f(x_0) + f'(x_0)(x-x_0)
\end{align*}

\begin{remark}
$f(x)$ gibt die Ableitung der Tangente an den Grenzwert der Funktion $f$ im Punkt $x_0 , f(x_0)$ an.
\end{remark}

\section{Berechnen an $f'(x)$ Ableitungsregeln:- }

\subsection{Linearität:-}
Sei $\underbrace{f(x)$ und $f(g)}_{h'(x)}$ gegeben sind , dann wie sieht die Ableitung von $h'(x)$ ?

\[(f(x)+g(x))'=f'(x)+g'(x)\]
\[ \underbrace{r f(x)'}_{h(x)}=\underbrace{r}_{\in \mathbb{R}}f'(x)\]
\begin{align*}
f'(x_0)=\dfrac{f(x)-f(x_0)}{x-x_0} = \lim \dfrac{f(x)-f(x_0)+g(x)-g(x_0)}{x-x_0} = f'(x_0)+g'(x_0)
\end{align*}
 
\subsection{Produktregel:-}
\[ (f(x) . g(x))' = f'(x) . g(x) + f(x) . g'(x)  \]
\subsection{kettenregel:-}
\begin{align*}
\underbrace{(f \circ g )'(x)}_{f(g(x))'} = f'(g(x)). g'(x)
\end{align*}
\subsection{Quotientenregeln:-}
In Tafelwerk : \\
\[ (\dfrac{f}{g})' = \dfrac{f'.g - f.g'}{g^2}\]
Herleitung :
\begin{align*}
\bigg(\dfrac{f(x)}{g(x)}\bigg)' =
\bigg(f(x)\frac{1}{g(x)}\bigg)'\\
&=f'(x)\frac{1}{g(x)} + 
f(x)\bigg(\frac{1}{g(x)}\bigg)'\\
&=\dfrac{f'(x).g(x)-f(x).g'(x)}{g(x)^2}
 \end{align*}
 
\begin{remark}[Tafelwerk]
\[(sin(x))'=cos(x)\]
\[(cos(x))'= - sin(x)\]
\end{remark}
\begin{example}
\begin{align*}
(tan(x))' &= \bigg( \dfrac{sin(x)}{cos(x)}\bigg)'\\
 &=  \dfrac{cos(x)cos(x)-sin(x)(-sin(x))}{(cos(x))^2}\\
 &=\dfrac{(cos(x))^2+(sin(x))^2}{(cos(x))^2}\\
 &= \dfrac{1}{(cos(x))^2}\\
 &= 1 + (tan(x))^2
\end{align*}
\end{example} 
\subsection{Ableitung der Umkehrfunktion $f^-1$ zu $f$ }
\begin{definition}
Ist $ y = f(x) $ eine umkehrbare differenzierbare Funktion, dann ist die Umkehrfunktion $ x = g(y)$ differenzierbar und es gilt: 
$g'(y)= \frac{1}{f'(g(y)}$ oder $\frac{dx}{dy}= \frac{1}{\frac{dy}{dx}}$ für $f'(x)\neq 0$.
Überlicherweise verraucht man die Variablen $x , y$ and schreibt $y = g(x)$ und $y'=g'(x)$. 
\end{definition}

\begin{example}
$f(x)=e^x$\\
$f'(x)=e^x$\\

\begin{proof}
Der Beweis ist einfach.Man geht wider von der Definition der Ableitung aus:
\begin{align*}
f'(x)= \lim_{h \to 0}{\frac{f(x+h)-f(x)}{h}}= \lim_{h \to 0}{\frac{e^{x+h}-e^x}{h}}
\end{align*}
Nutzt man die Potenzregln $e^{x+h}=e^x.e^h$ so ergibt sich : 
\begin{align*}
f'(x)= \lim_{h \to 0}{\frac{e^x.e^h-e^x}{h}}=e^x \lim_{h \to 0}{\frac{e^h-1}{h}}=1
\end{align*}
und weil $\lim_{h \to 0}{\frac{e^h-1}{h}}=1$ dann Also $f'(e^x)=e^x$
\end{proof}
\end{example}

\begin{remark}
\begin{align*}
f \circ f^{-1} = f^{-1} \circ f &= \quad \text{identisch}\\
e^{ln(x)} &= x \quad | \text{Abb} \\
e^{ln(x)}.(ln(x))' &= 1\\
\Rightarrow ln(x)'&= \frac{1}{e^{lnx}}= \frac{1}{x}
\end{align*}
\end{remark}

\begin{example}
\begin{align*}
f(x)=e^x\\
f'(x)=e^x\\
f^{-1}(x)=lnx\\
(f^{-1}(x))'=(lnx)=\frac{1}{x}
\end{align*}
\end{example}

\begin{example}
\begin{gather*}
f(x) = tan(x) \Rightarrow f'(x) = 1+(tan(x))^2\\
f^{-1}(x)= arctan(x) = x  | \quad Abl.\\
\Rightarrow 1 + ( \underbrace{tan(arctan x)}_{x})^2(arctanx)' = 1 \Rightarrow (arctanx)'=\frac{1}{1+x^2}
\end{gather*}
\end{example}


\section{Vorlesung 8}
\begin{definition}
Eine Reihe $\sum_{K=0}^{\infty}{a_k (x - X_0)^k} $ heißt Potenzreihe Dabei gilt $a_0,a_1 \dots \in \mathbb{R} , x_0 \in \mathbb{R} ,$ x ist eine reelle veränderlich $x_0$ heißt Mittelpunkt der Potenzreihe.    
\end{definition}

\begin{remark}
$(f_k(x))_{k=0}^{\infty}$ mit $f_k(x) = a_k(x-x_0)^k$. Folge von Funktionen $f_k(x)$
\begin{align*}
k &= 0 ,f_0(x) = a_0(x-x_0)^0 = a_0 \times 1 = a_0\\
k &= 1 ,f_1(x) = a_1(x-x_0)^1\\
k &= 2 ,f_2(x) = a_2(x-x_0)^2
\end{align*}
$ \big( \sum_{K=0}^{n}{f_n(x)} \big)_{n=0}^{\infty} $ Folge von Partielle Summen , Reihe $\sum_{K=0}^{\infty}{f_k(x)}$
\begin{gather*}
f_0(x)\\
f_0(x) + f_1(x)\\
f_0(x) + f_1(x) + f_2(x)
\end{gather*}
\end{remark}

\begin{remark}
wir fragen nicht nach der Konvergenz dieser Folge sondern für welche x ist diese Folge konvergent 
\end{remark}

\begin{example}
$$ \sum_{K=0}^{\infty}{(\frac{-2}{3})^k}\frac{1}{k}(x-0)^k $$
für welche $ x \in \mathbb{R} $ konvergent ?\\
Wurzelkriterium für absolute konvergent : 

\begin{align*}
&= \lim_{k \to \infty}{\sqrt[k]{\text{die potentzreihe}}} \\ &= 
\lim_{k \to \infty}{\frac{2}{3} \sqrt[k]{\frac{1}{k}}}|x| \\ &= 
\frac{2}{3}|x|.\frac{1}{1}=\frac{2}{3}|x|<1\\
\text{PR  abs. konv. } \Leftrightarrow |x| < \frac{3}{2}
\end{align*}
Wurzelkriterium:
\begin{gather*}
\frac{2}{3}|x|>1 \Leftrightarrow |x|
> \frac{3}{2} \Leftrightarrow \text{PR div } 
\end{gather*}
$ x = \frac{-3}{2} $ einsetzen  
\begin{gather*}
\sum_{k=1}^{\infty}{\big(\frac{-2}{3}\big)^k} \frac{1}{k} \big( \frac{-3}{2} \big)^k = \sum_{k=1}^{\infty}{\frac{1}{k}} \text{ div}
\end{gather*}
$ x = \frac{3}{2} $ einsetzen  
\begin{gather*}
\sum_{k=1}^{\infty}{\big(\frac{-2}{3}\big)^k} \frac{1}{k} \big( \frac{3}{2} \big)^k = \sum_{k=1}^{\infty}{(-1)^k} \frac{1}{k} \text{ kon.}
\end{gather*}

\end{example}

\begin{example}
\[ \sum_{k=1}^{\infty}{(\frac{-2}{3})^k} \frac{1}{k} (x-7)^k  \text{ ist für } x \in (7- \frac{3}{2}, 7 + \frac{3}{2}) \text{ abs konvergent }\]
\end{example}

\begin{definition}
Sei $\sum_{k=0}^{\infty}{a_k(x-x_0)^k}$ eine P.R . dann ex. ein r $\in \mathbb{R} \geq 0 $ oder $ x = \infty $ , so dass die P.R für alle x mit $ |x-x_0| \leq r $ oder $x \in \mathbb{R} $ absolut konvergent ist. Dieser ( r )  heißt \textbf{Konvergenzradius} der PR
\end{definition}

\begin{remark}
Der konvergenzradius r ist unabhängig von Mittelpunkt $X_0$  
\end{remark}

\begin{remark}
Jede PR ist für $x = x_0$ abs . konvergent , denn $\sum_{k=0}^{\infty}{a_k(x-x_0)^k} = \sum_{k=0}^{\infty}{a_k 0^k} = 0$\\
Sei $\sum_{k=0}^{\infty}{a_k(x-x_0)^k}$ eine Reihe mit konvergenzradius r Dann kann eine Funktion f definieren \\
 \[f : ( x_0 - r \quad ,\quad x_0 +r ) \rightarrow \mathbb{R} : \underbrace{x \longmapsto \sum_{k=0}^{\infty}{a_k(x-x_0)^k}}_{\text{Grenzwert der PR}}  \]
\end{remark} 

\begin{remark}
wegen der abs Konvergenz ist diese Funktion f 
- Stetig auf $( x_0 - r , x_0 + r)$ bsw. $\mathbb{R}$\\
- beliebig oft differenzierbar. 
\end{remark}

\begin{remark}
Analog kann man PR über $\mathbb{C}$ definieren.
Z.B $\sum_{k=0}^{\infty}{\frac{z^k}{k!}}(z \in \mathbb{C})$ 
$\sum_{k=0}^{\infty}{\frac{1}{k!}(z-0)^k}$ ist für ..... abs. konvergent.
Quotienten Kriterium : 
\begin{align*}
\lim_{k \to \infty}{\bigg| \dfrac{\frac{z^k+1}{(k+1)!}}{\frac{z^k}{k!}}\bigg|} = \lim_{k \to \infty}{\bigg|\dfrac{z^{k+1 \times k!}}{z^k(k+1)!}} = \lim{\frac{|z|}{k+1}} = \underbrace{|z|\times \lim_{k \to \infty}{\frac{1}{k+1}}}_{0}<0
\end{align*} 

exp : $\mathbb{C} \rightarrow \mathbb{C} , Z \longmapsto  \underbrace{\sum_{k=0^{\infty}}{\frac{z^k}{k!}}}_{\underbrace{\text{exp(z)}}_{e^z}}$

$Z = exp(i \varphi)= e ^{i \varphi}$.
\begin{align*}
e ^{i \varphi} &= \sum_{k=0}^{\infty}{\frac{(i \varphi)^k}{k!}} = \frac{(i \varphi)^0}{0!} + \frac{(i \varphi)^1}{1!} + \frac{(i \varphi)^2}{2!} + \frac{(i \varphi)^3}{3!} + \frac{(i \varphi)^4}{4!} + \frac{(i \varphi)^5}{5!}\\
 &= 1 +  i \frac{\varphi^1}{1!} - \frac{\varphi^2}{2!} - \frac{\varphi^3}{3!} - \frac{\varphi^4}{4!} - \frac{\varphi^5}{5!} \dots \\
 &= \underbrace{\sum_{k=0}^{\infty}{(-1)^k \frac{\varphi^{2k}}{(2k)!}}}_{cos(\varphi)} + i \times \underbrace{\sum_{k=0}^{\infty}{(-1)^k \frac{\varphi^{2k+1}}{(2k+1)!}}}_{sin(\varphi)}
\end{align*}
\end{remark}
\textbf{Approximation} stetiger Funktionen $f(x)$ durch \textbf{Taylorpolynom} $p_n(x)$ :\\
(1) \\
$$f(x) \approx f(x_0)+f'(x_0).(x-x_0)^1$$\\
$= t(x)$ Tangente an den Graph von $f(x)$ in Punkt $(x_0 , f(x_0) = p_1(x)$
 
lineare Approximation $(n=1)$
Linearisierung\\
fehlende Skizze !!! 

\begin{remark}
\[ f(x_0)= p_1(x_0) \]
\[ f'(x_0)= p_1'(x_0) \]
\end{remark}
(2) 
\end{example}


\section{Vorlesung 9 \href{https://tu-dresden.de/mn/math/algebra/das-institut/beschaeftigte/antje-noack/ressourcen/dateien/v120-1/MathMethInf09.pdf?lang=en}{(14.05.2019)} }

\begin{definition}[Taylorscher Satz]
Sei $f : [a,b] \rightarrow \mathbb{R} $ eine $C^n$ - Funktion , $ n \in \mathbb{N}$ und $x_0 \in ]a , b[$.\\
Dann gibt es genau ein Polynom $T_n(x;x_0)$ höchstens n-ten Grades mit Approximationsgüte 
 \[ f(x) = T_n(x;x_0) + o((x-x_0)^n), \]
das so genannte \textbf{Taylor Polynom n-ten Gerades} zum Entwicklungspunkt $x_0$
\[ T_n(x;x_0) := \sum_{k=0}^{n}{\dfrac{f^(k)(x_0)}{k!}(x-x_0)^k} \]
Ist $ f $ eine $C^{(n+1)}$- Funktion, so gilt für den Fehler die so genannte \textbf{Restgliedformel nach Lagrange}
\[ f(x) = \sum_{k=0}^{n}{\dfrac{f^(k)(x_0)}{k!}(x-x_0)^k} + R_n(x;x_0), \quad a \leq x \leq b \]    

\end{definition}



\begin{example}
$f: \mathbb{R} \rightarrow \mathbb{R} : x \mapsto x^2-1 $ gesucht
\end{example}
\subsection{Taylor-Polynom $P_n(x)$ von $f(x)$}
\begin{center}
\begin{tabular}{ c c c }
$f(x)= x^2-1$  & $f(0) = 1 $     & $f(1) = 0$ \\ 
$f'(x) = 2x $  & $f'(0) = 0 $    & $f'(1) = 2$ \\  
$f''(x) = 2 $  & $f''(0) = 2 $   & $f''(1) = 2$  \\
$f'''(x) = 0 $ & $f'''(0) = 0 $  & $f'''(1) = 0$  
\end{tabular}
\end{center}

\begin{align*}
p_n(x) &= \underbrace{f(0) + f'(0)(x-0)}_{t(x) \text{ lineare Approximation }}  + \frac{f''(0)}{2!}(x-0)^2 + \frac{f'''(0)}{3!}(x-0)^3 + \dots\\
&= -1 + 0 x + \frac{2}{2!} x^2 + 0 = -1 + x^2 = f(x)
\end{align*}
Das Polynom ist bei der Entwicklung zu einem Taylor-Polynom zum selben Polynom zurückgekommen

\begin{align*}
p_n(x)&= f(1) + f'(1)(x-1)+ \dots \\
&= 0 + 2(x-1) + \frac{2}{2!}(x-1)^2 + 0 \\
&= 2x -2 + x^2 -2x +1 = x^2 -1
\end{align*} 

\begin{example}
gegeben : $f(x) = \frac{e^x}{cos(x)}$  gesucht : $p_2(x)$ für $x_0=0$\\
Methode des Impliziten Differenzieren 
\begin{gather*}
f(x) cos(x) + f(x)(-sin(x)) = e^x \quad | \quad \text{abl.} \\
f'(x) cos(x) + f(x)(-sin(x))= e^x \quad | \quad \text{abl.} \\
f''(x)cos(x) + f'(x)(-sin(x))+ f'-(x)(-sin(x)) + f(x)(- cos(x)) = e^x \\
f(0)cos(0) = e^0 \Rightarrow f(0) = 1\\
f'(0)\times 1 + f(0) \times 0 = 1 \Rightarrow f'(0) = 1\\
f''(0)\times 1 + f(0) \times (-1) = 1 \Rightarrow f''(0) = 2\\
p_2(x) = 1 + 1 x + \frac{2}{2!}x^2 = 1 + x + x^2\\
f(x) = \frac{1}{x+1}
\end{gather*}
\end{example}

\begin{example}
$f^k(x) = (-1)^k \frac{k!}{(1+x)^{k+1}}$\\
$\textbf{Induktionsanfang}$ 
\[ f^0(x) = f(x) = \frac{1}{1+x} = (-1)^0 \frac{0!}{(x+1)^{0+1}} = 1 \frac{1}{x+1} = \frac{1}{x+1} \text{w.A}\]
$\textbf{Induktionsschritt}$\\
$\textbf{Induktionsvoraussetzung}$\\
Es gelte $f^k(x)=(-1)^k\frac{k!}{(x +1)^{k+^1}}$ für $k \in \mathbb{N}$\\
$\textbf{Induktionsbehauptung}$ : Dann gilt \\
\[ f^{(k+1)}(x)= (-1)^{(k+1)} \frac{(k+1)!}{(x+1)^{(k+2)}}\]
\textbf{Induktionsbeweis}\\
$(\dots \dots)$\\
\begin{align*}
f^{(f+1)}(x) = (f^x(x))' &= \big((-1)^k \frac{k!}{(x+1)^{(k+1)}}\big )' \\
&= (-1)^k k! (x+1)^{-(k+1)}\\
&= (-1)^k k! (-(k+1)(x+1))^{-(k+^2)}\\
&= (-1)^{k+1} (k+1)! \frac{1}{(x+1)^{t+2}} \Rightarrow \text{ Ind Beh . ist dann bewiesen. }\\
\text{Die behauptung gilt für alle } k \in \mathbb{N}
\end{align*}
\end{example}
\begin{example}
$f(x)= \frac{1}{1+x}$ 
\[ p_1(x) = 1-x \]
\[ p_2(x) = 1-x + x^2 \]
\[ p_3(x) = 1-x+x^2+x^3 \]
\end{example}
\begin{remark}
Bei : $p_2(x)$ wird der Fehler für große werte von x größer der Fehler bei $p_1(x) , p_2(x)$
\end{remark}
\subsection{Taylor-Formel:}
\[ F(x) = p_n(x)+ \underbrace{R_n(x,x_0)}_{=  \text{n-tes Restglied }} R_n(x,x_0) \text{Fehler bei der Approximation.} \]
\begin{theorem}
Darstellung von $R_n(x,x_0)$ nach Lagrange 
Sei $f:(a,b) \rightarrow \mathbb{R}$ eine $(n+1)$ und stetig differenzierbar Funktion und $x_0 \in (a,b)$
Dann gilt : $f(x)= p_n(x)+ R_n(x_1 , x_0)$ und $\forall x \in (a,b) \exists z \in \mathbb{R}$ zwischen $x$ und $x_0$ : \\
   \[ R_n(x , x_0) = \frac{f^{(n+1)}(z)}{(n+1)!}(x-x_0)^{(n+1)}) \]
\end{theorem}
\begin{example}
$f(x)=e^x$ , $x = 0$
\begin{gather*}
f^k(x)= e^x\\
f^k(0)= 1 \Rightarrow P_n(x) = \sum_{k=0}^{n}{\frac{p^k(0)}{k!}x^k} = \sum_{k=0}^{n}{\frac{x^k}{k!}}\\
f(x)= p_n(x) + R_n(x,0) \text{ und } R_n(x,0) = \frac{e^z}{(n+1)!} y^{n+1} \quad z \in (x,0)\\
\text{wir betrachten } f(x) = e^x \text{ für } |x| \leq 1\\
\big|R_n(x,0)|=|\frac{e^z}{(n+1)!}x^{n+1}| \leq \frac{e^1}{(n+1)!} \leq 10^{-2} \text{ für } n = 5
\end{gather*}
\subsection{Näherungsformel für $e^x$}
\[ p_5 (x) = 1 + x + \frac{x^2}{2} + \frac{x^3}{3!} + \frac{x^4}{4!} + \frac{x^5}{5!} \text{ für } x \leq 1 \]
\end{example}
\begin{definition}
Sei $f : (a,b) \rightarrow \mathbb{R}$ beliebig oft stetig differenzierbar und $x_0 \in (a,b)$\\
Die Reihe \[ \sum_{k=0}^{\infty}{\frac{f^k(x_0)}{k!}(x-x_0)^k}\] heißt \textbf{Taylor Reihe} von $f$ an der stelle $x_0$ 
\end{definition}
\begin{remark}
(1) Nicht für jede Funktion $f(x)$ ist dir \textbf{Taylor-Reihe konvergent}\\
(2) Ist die Taylor-Reihe konvergent , dann muss der Grenzwert ......... die Funktion $f$ sein.\\
(3) Ist die Taylor-reihe konvergent gegen $f$ , d.h \[f(x)= \sum_{k=0}^{\infty}{\frac{f^k(x_0)}{k!}(x - x_0)^k}\] , heißt die Funktion $f$ \textbf{reell analytisch} \\
\end{remark}
\begin{example}
\[ f(x) = \frac{1}{x+1} \text{ mit } x \in (-1 , 1) \text{ ist reell analytisch }  \]
Taylor-reihe $\sum_{k=0}^{\infty}{(-1)^kx^k}$ hat konvergenzradius 1(...) und Mittelpunkt 0 
\end{example}
\begin{theorem}
Sei $|x| \leq 1 $ Dann gilt: $f(x) = \frac{1}{1 + x} =\frac{1}{1-(-x)} = \sum_{k=0}^{\infty}{x^k} = \sum_{k=0}^{\infty}{(-1)^kx^k}$ ist die Taylor-reihe Darstellung von $f(x)$ 
\end{theorem}
\subsection{Rechnen mit Potenzreihen: }
\textbf{\href{https://tu-dresden.de/mn/math/algebra/das-institut/beschaeftigte/antje-noack/ressourcen/dateien/v120-1/MathMethInf09Zusatz.pdf?lang=en}{Sieh der Zusatz (Rechnen mit Potenzreihen)}}\\
Es Sei \[ \sum_{k=0}^{\infty}{a_k(x - x_0)^k} := a(x) , b(x) =\sum_{k=0}^{\infty}{b_k(x-x_0)^k}\]
mit konvergenzradius $r_1$  für $a(x)$ , $r_2$ für b(x) sei r := min \{$r_1$ , $r_2$ \}\\
Dann gilt :
\begin{align*}
a(x) \pm b(x)&= \sum_{k=0}^{\infty}{(a_k + b_k)(x - x_0)^k}  \text{ für } x \in (x_0 - r , x_0 + r) \\
 C \times a(x) &= \sum_{k=0}^{\infty}{c.a_k(x-x_0)^k} \text{ für } x \in (x_0 -r , x_0+r) c \in \mathbb{R}\\
 a(x) . b(x) &= \sum_{k=0}^{\infty}{(a_0 b_k + a_1 b_{k-1}+ \dot + \dots +a_kb_0)(x- x_0)^k}
\end{align*}
$\dfrac{1}{b(x)}$  für  $b(x) \neq 0$  kann mit der Methode unbestimmten Koeffizienten. 
\section{Vorlesung 10 \href{https://tu-dresden.de/mn/math/algebra/das-institut/beschaeftigte/antje-noack/ressourcen/dateien/v120-1/MathMethInf10.pdf?lang=en}{17.05.2019}}
\section{Spezielle Ableitungen }
\subsection{Einleitung}
Für viele Funktionen kann die Ableitung nicht mit Hilfe einfacher Ableitungsregel bestimmt werden. Daher befindet sich an dieser Stelle  den wichtigsten Funktionen und ihren Ableitungen
\begin{align*}
f(x) &= x^x \quad x>0\\
ln(f(x)) &= \underbrace{ln  x^x}_{x \: ln  x } \Rightarrow \frac{1}{p(x)}f'(x) = 1 \times ln \ x + \underbrace{x \ \frac{1}{x}}_{1}\\
\Rightarrow f'(x) &= \underbrace{x^x }_{f(x)}(ln \ x + 1) \text{ logarithmisches Differenzieren }\\
f(x) &= x^x = e^{ln  x^x} = e^{x \ ln x} \Rightarrow f'(x)= \underbrace{e^{x \ ln  x}}_{x^x}(x \ ln  x)' = x^x (ln  x + 1) \\
(\sinh  x)'&= \cos h x\\
(\cosh  x)'&= -\sin h x\\
\cosh x :   &= \frac{e^x + e^{-x}}{2} \text{ \href{https://de.wikipedia.org/wiki/Sinus_hyperbolicus_und_Kosinus_hyperbolicus}{Kosinus Hyperbolicus} }\\
\sinh x:&= \frac{e^x + e^{-x}}{2}\\
\end{align*}

\begin{tikzpicture}[scale=1],

\begin{axis}[xlabel=$x$,
ylabel= $y$,
height=12cm,
width=16cm,
ymax=5,
ymin=-5,
xmin=-5,
xmax=5,
axis y line=center,
axis x line=center,
]
\addplot [ very thick,mark options={solid},red]{sinh(x)}  node at (-1.5, -4.5) {$\sinh x$};
\addplot [ very thick,mark options={solid},blue]{cosh(x)} node at (1, 3.5) {$\cosh x$};
\addplot[dashed]{x}node at (3.1, 4) {$"kettenlinie"$};
\end{axis}
\end{tikzpicture}
\subsection{Taylor - Entwicklung der Kosinus hyperbolicus}
gesucht: \textbf{Taylor-Reihe Entwicklung} für $\cosh x$\\
\begin{align*}
e^x &= \sum_{ k = 0 }^{ \infty }{ \frac{x^k}{k!} }\\
e^{-x} &= \sum_{ k = 0 }^{\infty}{\frac{(-x)^k}{k!}}\\
&= \sum_{ k = 0 }^{\infty}{(-1)^k\frac{x^k}{k!}}
\qquad (x \in \mathbb{R})
\end{align*}
Reihe ist absolut konvergent für alle $ x \in \mathbb{R}$ \\
\[\cosh x= \frac{e^x + e^{-x}}{2}\]
\begin{gather*}
\rightsquigarrow \cosh x= \frac{1}{2}(1+\frac{x}{1!}+ \frac{x^2}{2!} + \frac{x^3}{3!}+\frac{x^4}{4!} + \dots )\\
+(1-\frac{x}{1!}+ \frac{x^2}{2!} - \frac{x^3}{3!}+\frac{x^4}{4!} + \dots)\\
=1 + \frac{x^2}{2!}+  \frac{x^4}{4!} = \sum_{k=0}^{\infty}{\frac{x^{2k}}{(2k)!}}
\end{gather*}
\section{Spezielle Grenzwerte}
\subsection{Regeln von Bernoulli l'Hospital -}
\textbf{Einleitung}\\
Bei den Regeln von de l'Hospital handelt es sich um nützliche Methoden zur Berechnung so genannten \textbf{"unbestimmten Ausdrücken"} der Form $\frac{0}{0} $ oder $\frac{\infty}{\infty} $. Gemeint sind hiermit Grenzwerte der Form $\lim_{n \to \infty}\frac{f(x)}{g(x)}$ wobei $f(x) \to 0 $ und $g(x) \to 0 $ oder aber $f(x) \to \infty $ und $g(x) \to \infty $konvergieren.\\
Die Existenz dieser Grenzwerte und ihr Wert hängt davon ab, wie schnell Zähler und Nenner gegen Null bzw. gegen $\infty$ konvergieren. Die Regel von de l'Hospital besagt in etwa, dass es hierzu genügt, den Quotienten der Ableitungen von f und g zu untersuchen.
\begin{example}
Seien $ f(x) $ , $g(x)$ reelle , zweimal stetig differenzierbare Funktionen auf $(a,b)$ und $f(x_0) = g(x_0) = 0$\\
gesucht: $$\lim_{x \to x_0}{\frac{f(x)}{g(x)}}$$\\
\begin{align*}
\dfrac{f(x)}{g(x)} &=
\dfrac{f(x_0) + f'(x_0)(x - x_0)+ \frac{1}{2}  f''(z_f)(x-x_0)^2}
{g(x_0) + g'(x_0) + (x - x_0) + \frac{1}{2}  g''(z_g)(x - x_0)^2 }\\
&= \dfrac{x - x_0}{x - x_0} . 
{ \dfrac 
{f'(x_0) + \frac{1}{2} f''(z_f )(x-x_0)}
{g'(x_0) + \frac{1}{2} g''(z_g )(x-x_0)} 
} \\
\lim\limits_{x \rightarrow x_0} \frac{f(x)}{g(x)}&=1.\lim_{x \to x_0}\dfrac{f'(x_0)+\frac{1}{2} f''(z_f).0}{g'(x_0)+\dots .0}\\
&= \lim_{x \to x_0}{\frac{f'(x_0)}{g'(x_0)}} \qquad \text{falls dieser existiert}
\end{align*}
\[\Rightarrow \lim_{x \to x_0}{\frac{f(x)}{g(x)}} =
\lim_{x \to x_0}{\frac{f'(x)}{g'(x)}}   \qquad \text{falls der Grenzwert existiert }\]
\end{example}
\begin{remark}
Seien f , g : ]a , b[ $\rightarrow \mathbb{R}$ differenzierbar , sei $x_0 \in$ ]a , b[ mit $f(x_0) = g(x_0) = 0 $ und gelte $g'(x)\neq 0 $ für $x \neq x_0$ Dann Folgt:
\[ \lim_{x \to x_0}{\dfrac{f(x)}{g(x)}} = \lim_{x \to x_0}{\dfrac{f'(x)}{g'(x)}} \]
sofern der rechts stehende Grenzwert existiert.
\end{remark}
\begin{example}
\begin{gather*}
\lim_{x \to 0}{\frac{sin x}{x}} =
 \lim_{x \to 0}{\frac{(sin x)'}{(x)'}}
= \lim_{x \to 0}{\frac{cos x}{1}} = cos(0) = 1
\end{gather*}
\end{example}
\begin{example}
\begin{align*}
\lim_{x \to 0}{\dfrac{e^x - sin(x) + cos(x)-2}{x^3 . cos(x)}} 
= \lim_{x \to 0 }{\frac{e^x - cos(x) - sin(x)}{3x^2 cos(x)-x^3(-sin(x))}} \dots \dots &= \frac{1}{3}
\end{align*}
\end{example}
\begin{remark}
Diese Methode kann man durch anwenden für $x \to + \infty$ , $x \to - \infty$ . und für $\frac{\infty}{\infty}$ , $\frac{-\infty}{-\infty}$ , $\frac{+\infty}{-\infty}$ , $\frac{-\infty}{+\infty}$\\
\[ \lim_{x \to x_0 \ \pm \infty}{\frac{f(x)}{g(x)}}= 
\lim_{x \to x_0 \ \pm \infty}{\frac{f'(x)}{g'(x)}} \text{falls der Grenzwert existiert. }\]
falls der Grenzwert existiert. 
\end{remark}
\begin{remark}
Man kann durch geeignetes Umformen auch Grenzwerte vom Typ $0.\infty $ berechnen , sowie $0^0 , 1^0 , 1^0$
\end{remark}
 \begin{example}
$\lim\limits_{x \rightarrow 0+} x \ lnx=$\\
1. Mögl. $\lim\limits_{x \rightarrow 0+} \dots = \lim\limits_{x \rightarrow 0+}\frac{x}{\frac{1}{lnx}}$  \\ \\
2.Mögl. $\lim\limits_{x \rightarrow 0+}=\lim\limits_{x \rightarrow 0+}\frac{lnx}{\frac{1}{x}}=\lim\limits_{x \rightarrow 0+}\frac{(lnx)'}{(\frac{1}{x})'}= \lim\limits_{x \rightarrow 0+} \frac{1.x^2}{x.1}(-1)=\lim\limits_{x \rightarrow 0+}(-x)=0$
 \end{example}
\begin{example}
$\lim\limits_{x \rightarrow 0}x^2=\lim\limits_{x \rightarrow 0}e \ lnx^x= \lim\limits_{x \rightarrow 0}e^{xlnx}=e^{\lim\limits_{x \rightarrow 0} xlnx}=e^0=1 \\ \\
\lim\limits_{x \rightarrow 0}x^{\frac{1}{lnx}}= \lim\limits_{x \rightarrow 0} e^{\frac{1}{lnx}lnx}= \lim\limits_{x \rightarrow 0} e^1=e$
\end{example}
\begin{example}
	\begin{align*} 
	\limXin (1+\frac{1}{x})^x &=\limXin e^{\ln(1+\frac{1}{x})^x} \\	&=\limXin e^{x \ln (1+\frac{1}{x})}\\
	&=e \limXin x \ln (1+\frac{1}{x})= \dots = e^1=e \\	
	\text{(Nebenrechnung) NR }\limXin x \ln (1+\frac{1}{x})=& \limXin\frac{\ln (1+\frac{1}{x})}{\frac{1}{x}} = \limXin \frac{\frac{1}{1+\frac{1}{x}}(1+\frac{1}{x})'}{(\frac{1}{x})'}= \limXin \frac{1}{1+\frac{1}{x}}=\frac{1}{1}=1\\
	\text{also auch } \limXin (1+\frac{1}{n})=e \\
	e^x = \sum_{k=0}^{\infty} \frac{x^k}{k!} \overbrace{\longrightarrow}^{x=1} e^1 =e= \sumOin \frac{1^k}{k!}= \sumOin \frac{1}{k!}
	\end{align*}
\end{example}
\section{Integral}
\begin{align*}
f(x)>0 \text{ auf } [a,b]\\
\underline{S_p}= \sumIin f_k (x_k - x_{k-n})und f_k =\min \{ f(x) | \in [x_{k-1},x_K ] \} \\
\overline{S_p}= \sumIin f_k (x_k - x_{k-n})und \overline{f_k} =\max \dots \\
\underbrace{ \lim\limits_{||p|| \rightarrow 0} \underline{S_p}}_{(1) ex.}\underbrace{=}_{(3)} \lim\limits_{||p|| \rightarrow 0 } \overline{S_p}= \underbrace{\int_a^b f(x) \mathrm{d}x}_{Integral von f(x) auf [a,b]}
\end{align*}
%skizze
\begin{example}
$ 	D(x)=\begin{cases} 
		0  \\    
		1 \\  
	\end{cases} auf[0,1]$
	%skizze
	 \t{skizze fehlt!}
\end{example}
\begin{remark}
In jeder reelle Intervall liegen rationale und irrationale Zahlen
	%skizze
\end{remark}
Riemann : $\lim \underline{S_p} \overset{\overset{ex. irrationale Zahl im Int.}{\downarrow}}{=} \lim \sum (x_k- x_{k-1}) 
= \lim 0 = 0 $ 
\[  \neq \lim \overline{S_p}= \lim \sum (x_k-x_{k-1}))> 0\]
Das Riemann - Integral von D(x) ex. nicht 

\subsection{Lebague-Integral }
 skizze fehlt!\\
 %skizze!
\[ \phi(x) \text{ treppenfunktion } \int_a^b \phi(x) = \sumIn c_k (x_k - x_{k-1} ) 
(\phi_k (x)) \]
\[ \text{ Folge von Treppenfunktion auf } [a,b] \backslash M \]
M := $ Nullmenge \textbf{ z.B } \phi \int_a^b D(x)\mathrm{d}x=0$ \\
\[\underbrace{\limKin \phi_k (x)}_{ex.}= f(x)\]
\[\underbrace{\limKin \int_a^b \phi(x)  }_{ex.}=\underbrace{\int_a^b f(x) \mathrm{d}x }_{\text{Lebague-Integral}} \]

\listoftheorems[ ignoreall,show={definition}]
\listoftheorems[ignoreall,show={example}]
\end{document}(a-\varepsilon ,a+\varepsilon ).

