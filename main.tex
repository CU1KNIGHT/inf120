% Mathematische Methoden für Informatiker: Institut %Algebra Algebra
% TU Dresden SS19
% Prof Dr Ulrike Baumann

% Setup document
\documentclass[a4paper,12pt]{report}

\usepackage{inputenc,fontenc}
\usepackage{longtable,tabularx,tabulary,array,booktabs,multicol}
\usepackage{graphicx,caption,color,xcolor}
\usepackage{chngcntr}
\usepackage{verbatim}
\usepackage{url}
\usepackage{amsmath,amsthm,amssymb,amsfonts}
\usepackage{amsthm}
\usepackage[nottoc]{tocbibind}
\usepackage{mathtools}
\usepackage{mathabx}
\usepackage{thmtools}
\usepackage[a4paper,margin=3cm]{geometry}
\usepackage{pgfplots}
\usepackage[english,german]{babel}
\usepackage{hyperref}
\usepackage{MnSymbol} %sch\"{o}nere Mathe-fonts
\usepackage{eucal}    %sch\"{o}nere Skript-Buchstaben
\usepackage{dsfont}   %sch\"{o}nere "Doppelstrich"-Fonts
\usepackage{tikz}
\usepackage{forest}



% Fontauswahl

\pgfplotsset{compat=1.12}

\usepgfplotslibrary{fillbetween}
\usetikzlibrary{patterns}

\newcommand{\mathscr}{\mathcal}
%\usepackage{picinpar} %Einbinden von Bildern
%\usepackage{picins}   %Einbinden von Bildern

% Define oftenly used commands

\newcommand{\nameE}{Mohamed}
\newcommand{\vornameE}{Abdelshafi}
\newcommand{\emailE}{m.abdelshafi@mail.de}
\newcommand{\nameS}{Kiki}
\newcommand{\vornameS}{Mahmoud}
\newcommand{\emailS}{mahmoud.kiki@tu-dresden.de}



%% Created defnintions 
\def\checkmark{\tikz\fill[scale=0.4](0,.35) -- (.25,0) -- (1,.7) -- (.25,.15) -- cycle;} 




%% Theorem environments etc.
%% See also: ftp://ftp.ams.org/ams/doc/amscls/amsthdoc.pdf
%%
%created theorem
\swapnumbers
\theoremstyle{plain} %Text ist Kursiv
\newtheoremstyle{break}% name
{}%         Space above, empty = `usual value'
{}%         Space below
{\itshape}% Body font
{}%         Indent amount (empty = no indent, \parindent = para indent)
{\bfseries}% Thm head font
{.}%        Punctuation after thm head
{\newline}% Space after thm head: \newline = linebreak
{}%         Thm head spec
\newtheorem{theorem}{Satz}[chapter]
\newtheorem{lemma}[theorem]{Lemma}
\newtheorem{proposition}[theorem]{Proposition}
\newtheorem{corollary}[theorem]{Korollar}
%\theoremstyle{definition} %Text ist \"upright"
\theoremstyle{break}
\newtheorem{remark}[theorem]{Bemerkung}
\newtheorem{definition}[theorem]{Definition}
\newtheorem{example}[theorem]{Beispiel}
\newtheorem{beweis}[theorem]{Beweis}
\newtheorem{schreibweise}[theorem]{Schreibweise}



\hypersetup{
	colorlinks=true,
	linkcolor=blue,
	filecolor=magenta,
	urlcolor=cyan}


{\author{ \vornameE \nameE \vornameS \nameS}}
{\title{Mathematische Methoden für Informatiker}}







\newcommand{\institut}{Institut Ihres Betreuers / Ihrer Betreuerin}

\newcommand{\thema}{Mathematische Methoden für Informatiker}

\newcommand{\datum}{\today}%Format tt.\ mm.\ jjjj

%shortcut :Mahmoud
\newcommand{\sumIn}{\sum_{K=1}^{n}}
\newcommand{\sumIin}{\sum_{K=1}^{\infty}}
\newcommand{\sumOin}{\sum_{K=0}^{\infty}}
\newcommand{\sumOn}{\sum_{K=0}^{n}}
\newcommand{\fracKetwo}{\frac{1}{k^2}} %1/(k^2)
\newcommand{\fracKone}{\frac{1}{k}}   %1/k
\newcommand{\limNin}{\lim\limits_{n \rightarrow \infty}}
\newcommand{\limKin}{\lim\limits_{k \rightarrow \infty}}




\begin{document}
	\selectlanguage{german}
	%\selectlanguage{english} %Entferne "%", wenn Sprache Englisch ist



	%% Titelseite

	\thispagestyle{empty}


    \begin{center}
    {\Large
    Technische Universit\"{a}t Dresden\  \ \textbullet\ \ Fakult\"{a}t Informatik
    }

        \vfil

        {\bfseries\Huge\thema}

        \vfil

        {\LARGE
Mitschrift zur Vorlesung Sommer Semester 2019  \\[\bigskipamount]

        \bfseries{\itshape Bachelor of Science  \textup{(}B.Sc.\textup{)}}\\[\bigskipamount]
        }% Ende Large

        \vfil\vfil\vfil
        Dozent: Prof. Dr. Ulrike Baumann \\
        vorgelegt von\\
        \item "..." \\
        \item \textsc{\vornameE\ \nameE } \\ \texttt{\emailE} \\  \item
        \textsc{\vornameS\ \nameS \qquad } \\ \texttt{\emailS}  \\
        \item ... \\
        Tag der Einreichung: \datum\\[\bigskipamount]

    \end{center}


    \cleardoublepage



    \tableofcontents

    \thispagestyle{empty}

    %\makeatletter
    %\begin{titlepage}

    %	The title is \@title
    %	It was written by \@author\space on \@date

    %\end{titlepage}
    \setcounter{page}{0}
    \chapter*{Einleitung}
    Wir schreiben hier die vorlesungen von INF-120-1( Mathematische Methoden für Informatiker) mit.
    wenn Ihr Fragen habt oder Fehlern gefunden Sie können gerne uns eine E-mail schreiben oder Sie können einfach bei github eine  \href{https://github.com/CU1KNIGHT/inf120/issues}{Issue (link)} erstellen.
    wir freuen uns wenn Sie mit uns mitschreiben möchten, oder helfen mit der Fehlerbehebung.
    \newline \newline
    \vornameE \ \nameE \newline
    \vornameS  \ \nameS
    
    
%%
%% Author: mahmoud
%% 19/04/12
%%
\chapter{Vorlesung 1}
\section{Folge und Reihen}
\subsection{Folge}
\numberwithin{equation}{theorem}
\begin{definition}[Folgen]
    Ein folge ist eine Abbildung

    \[ f: \mathbb{N} \rightarrow \underbrace{\mathbf{M}}_{Menge} : \mathrm{n} \mapsto \underbrace{X_n}_{folgenglied} \]

\end{definition}
\begin{remark}

    \begin{align}	\mathbf{M} &= \mathbb{R} \quad \text{reelewert Folge} \notag \\
    \mathbf{M}&= \mathbb{C} \quad	\text{komplexwertig Folge}   \notag \\
    \mathbf{M}&= \mathbb{R}^n \quad \text{vertical Folge} \notag
    \end{align}




\end{remark}
\begin{description}

    \item[Bezeichnung]

    \quad $(X_n)$ \space \text{mit} \space $ \left( X_n   \right)$=$ \frac{n}{n+1} $
    \\ \\ Aufzählung der folglieder: 0 , $\frac{1}{2}$ ,$\frac{2}{3}$ , $\frac{3}{4}$ , \dots

\end{description}
\begin{remark}
    zuwerten wird $\mathbb{N}$ durch $\mathbb{N}$ {0,1 \dots} erstellt.


\end{remark}
\begin{tikzpicture}
    \begin{axis}[
    ymin = -1,
    ymax = 1,
    xmin = -1,
    xmax = 5,
    axis x line=center,
    axis y line=center]
    \addplot[samples at={1.5,...,4.5},only marks,mark size=1] { x/(x+1)};
    \end{axis}
\end{tikzpicture}
\begin{example}
    \[\]
    \begin{enumerate}

        \item Konstante Folge $(X_n)$ mit \quad $X_n = a \in \mathbf{M },a \dots$ \\
        \[ X_n = a \in \mathbf{M} \]
        \item Harmonische Folge $(X_n)$  mit $X_n$ =  $\frac{1}{n+1}$ \quad$ n \geq 1$
        \item Geometrische folge $(X_n)$ mit $X_n = q^n \:, \: q \in \mathbb{R}, \dots $
        \item Fibonaccifolge $(X_n)$ mit
        \[ X_n =\frac{1}{\sqrt{5}} \Big(  \big( \frac{1+ \sqrt{5}}{2} \big)^n - \big( \frac{1- \sqrt{5}}{2}\big)^n   \Big)    \]

        \item Fibonacci folgen $(X_n)$
        \begin{align}
            X_0 &=0 \notag \\X_1 &= 1 \notag \\
            X_n+1 &= X_n+X_{n-1} \quad (n>0)  \notag
        \end{align}

        \item conway  folge
        \[ 1, 11 ,21 , 1211, 111217, 312211 \dots \]

        \item folge aller Primzahlen: \[ 2, 3 ,5 ,7 ,11, 13 , \dots \]

    \end{enumerate}
\end{example}

\section{Rechnen mit Folgen }
\begin{align*}
    ( M  = \mathbb{R} \quad & oder \quad M = \mathbb{C} ) \\
    (X_n)+(y_n) &:= (X_n+y_n)\\
    K(X_n)&:=(KX_n)\in \mathbb{R} \quad  oder \quad \in  \mathbb{C}
\end{align*}
\begin{remark}

    Die Folge bildet ein Vektorraum.
\end{remark}
\newpage

\begin{definition}$ \newline$
        \begin{enumerate}

        \item Eine reellwertige Funktion ist in der Mathematik eine Funktion, deren Funktionswerte reelle Zahlen sind.

        \item Eine reellwertige heißt beschränkt wenn gilt

        \[	\exists r \in \mathbb{R}_+ , \forall r \in \mathbb{N}: \underbrace{|X_n|}_{\mathclap{\text{Betrag einer reellen oder komplexer Zahl}}} \leqq r   \]

    \end{enumerate}
\end{definition}

\begin{example}
    \[(X_n)\quad mit \quad X_n = (-1)^n \times \frac{1}{n} \]
    \[-1 ,\quad \frac{1}{2}, \quad \frac{-1}{3}, \quad \frac{1}{4} ,\quad \frac{-1}{5},\dots \]
\end{example}



\begin{tikzpicture}
    \begin{axis}[
    ymin = -2,
    ymax = 2,
    xmin = -3,
    xmax = 3,
    axis x line=center,
    axis y line=center]
        \addplot[dashed, name path =A] coordinates {(-3,1) (3,1)};
        \addplot[samples at={-3,...,3},only marks,mark size=1]{(-1)^x*(1/x)};
    \end{axis}
\end{tikzpicture}
\begin{remark}

    $(X_n)$ ist beschränkt mit $r = 1$ denn $|(-1)^n \frac{1}{n}|=|\frac{1}{n}| \leqq 1 \hookleftarrow r $

\end{remark}

\newpage
\begin{example}
    \[  (X_n) \text{ mit  } X_n = (-1)^n \quad \frac{1}{n}+1 \quad \text{bechränkt r = 3/2}\]

    \[ -3/2 \quad \leq X_n \leq 3/2 \quad \forall n \in \mathbb{N} \]
    %
    \begin{tikzpicture}
        \begin{axis}[
        ymin = -2,
        ymax = 2,
        xmin = -4,
        xmax = 4,
        axis x line=center,
        axis y line=center]
            \addplot[dashed, name path =A] coordinates {(-4,3/2) (4,3/2)};
            \addplot[dashed,samples at={0,...,3},only marks,mark size=1][ domain=(0):(2)] {log10(3*x + 1)};
        \end{axis}
    \end{tikzpicture}
\end{example}

\begin{example}{Standard:}\\

{Die folge}
$ \bigg(\big(1 + \frac{1}{n} \big)^n \bigg)^\infty_{n=1}$
{ist beschränkt durch 3}\\

Zu zeigen: \quad $ -3 \leq X_n \leq 3 \quad \text{für alle} \quad n \in \mathbb{N} $


\[ { (a+b)^n = \sum_{k=0}^{n} \binom{n}{k} a^k . b^{n .k} = \sum_{k=0}^{n} \binom{n}{k} a^{n.k} b^k }  \]
\[  \binom{n}{k} = \frac{n!}{k!(n-k!)} =  \frac{n(n-1) -(n-k-1))}{k!} \]
\[  \sum_{K=1}^{n} \frac{1}{k} = 1+ \frac{1}{2} + \frac{1}{2.3} + \frac{1}{2.3.4} + \dots \]
\end{example}


\section{geometrische Summen Formel (Tafelwerk)}
\begin{definition}

    Die Folge $(X_n)$ heißt monoton $\Big\{ \text{wachsend fallend} \Big\}$
    \[ wenn \quad gilt: \forall n \in \mathbb{N}:
    \left\{
    \begin{array}{ll}
        X_n  & \leq X_n +1 \\
        X_n  & \geq X_n+1
    \end{array}
    \right. \]
    \text{man spricht von Streng monotonie}
    $ wenn \leqq durch > und \geqq durch < \dots  $
\end{definition}
\begin{remark}
    \[ X_n \leq X_{n+1} \ \Leftrightarrow  X_n - X_{n+1} \leq 0 \quad \Leftrightarrow  \frac{X_n}{X_{n+1}} \leq 1 \]
\end{remark}

\begin{example}


    \[	(X_n) \text{ mit }\quad X_0  := 1 \quad, X_{n+1}   := \sqrt{X_n +6} \]
    \text{ist Streng monoton wachsend Beweis mit Vollständiger Induktion}
    \paragraph{Standard Bsp:}
    $ \big( \big(1+ \frac{1}{n} \big)^n \big) $ ist streng monoton wachsend
\end{example}
\begin{remark}

    \[
        \begin{tabular}{|c| c c |}
            \hline
            monoton & ja & nein  \\
            \hline
            \rule{0pt}{3ex}
            Beschränkkeit & $(\frac{1}{n})$ & $(-1)^n$ \\
            nein & (n) & $(-1)^n$ \\
            \hline
        \end{tabular}
    \]

\end{remark}
\begin{definition}
    $(X_n)$ heißt $\boldsymbol{Konvergenz}$ wenn $(X_n)$ ein grenzwert hat.\\
    $(X_n)$ heißt $\boldsymbol{Divergenz}$ wenn sie keinen grenzwert hat.
\end{definition}
\begin{definition}[grenzwert]
 $ a \in$ $\mathbb{R}$ heißt grenzwert von $(X_n)$, wenn gilt:
\[ \underbrace{\forall \epsilon > 0 }_{beliebes \; klein} \quad \underbrace{\exists \mathbf{N} \in \mathbb{N}}_{beliebes \;  klein \; \underbrace{\Rightarrow |X_n -a|< \varepsilon}_{a- \varepsilon \leq X_n \leq a+\varepsilon } } , \forall n \in \mathbb{N} : m \geq \mathbb{N}  \]
\[ \text{Sei  } \varepsilon > 0 ; \varepsilon \text{  fest} \]
\[ \text{alle folglieder$ X_n$ mit n } \geq \mathbb{N} \curvearrowright \]
\begin{tikzpicture}[scale=1.1],

\begin{axis}[
height=8cm,
width=10cm,
ymax=2,
ymin=-2,
xmin=1,
xmax=5.5,
axis y line=left,
axis x line=bottom,
yticklabels={,, $ a - \varepsilon$, a , $a + \varepsilon$ },xtick={1,...,10},
xticklabels={,,N,n},
]

    \addplot[dashed, name path =A] coordinates {(0,1) (5.5,1)};


    \addplot[thick, samples=50, smooth,domain=0:6,magenta, name path=V] coordinates {(3,-2)(3,3)};
    \addplot[dashed, name path =B] coordinates {(0,-1) (5.5,-1)};
    \addplot[gray, pattern=north west lines] fill between[of=A and B, soft clip={domain=3:6}];
    \addplot[samples at={0,...,5},only marks,mark size=1] { 1/x };
    \addplot[samples at={0,...,5},only marks,mark size=1] { (1-x)/x^2 };
   



\end{axis}
\end{tikzpicture}%


\end{definition}

\section{vorlesung 2}
\begin{text}
    ist die folge beschränkt , monoton ?\\

    $(x_n)$ konvergierend : $\iff \exists a \in\mathbb{R} \quad \forall \epsilon > 0 \quad \exists n \quad \in N \quad \forall n \in N \quad \\
    n \geq N \Rightarrow |x_n - a |< \epsilon $
\end{text}


\begin{theorem}

    $(x_n)$ konvergierend : $\Rightarrow$ Der Grenzwert ist eindeutig beschränkt.

\end{theorem}



\begin{beweis}
    Sei a eine Grenzwert von $(x_n)$ , b eine Grenzwert von $(x_n)$ \\
    d.h sei $\epsilon > 0$,$\epsilon$ beliebig , $\epsilon$ fest \\


    \begin{equation}
        \exists  N_a \quad \forall n \geq N_a : |x_n-a|< \epsilon
    \end{equation}


    \begin{equation}
        \exists  N_b \quad \forall n \geq N_b : |x_n-b|< \epsilon
    \end{equation}

    Sei max $\{N_a,N_b\}=N$
    dann gilt : \\
    \begin{equation}
        n \geq N \Rightarrow |x_n - a| < \epsilon
    \end{equation}

    und \begin{equation}
            |x_n -b| < \epsilon \Rightarrow |x_n -a|+|x_n - b|< 2\epsilon
    \end{equation}\\
\newpage

    Annahme :- a $\neq$ b , d.h $|a-b|\neq 0 $
    \begin{gather*}
    |a-b|=|a+0-b|\\
    =|(a-x_n)+(x_n-b)| \leq |x_n - a|+|x_n-b|< 2 \epsilon \\
    also \quad |a - b|< 2 \epsilon
\end{gather*}


     wähle Z.B  \[\epsilon = \frac{|a-b|}{3}
        \quad
         \text{dann gilt}\ :|a-b|< \frac{2 \ |a-b|}{3}\]\\

        \[ \Rightarrow 1 < \frac{2}{3} \quad  \text{ falls  Aussage, Widerspruch  also  ist  die  Annahme  falsch  also  gilt }\ \quad a=b\]

\end{beweis}



\begin{example}

    $x_n$ mit $x_n = \frac{1}{n}$ (harmonische Folge)

\end{example}

\begin{beweis}
    Sei $\epsilon > 0 , \epsilon belibig , \epsilon fest$
    gesucht : N mit $n \geq$ N \\
    hat den Grenzwert 0

    \begin{gather}
        \Rightarrow |x_n-a|= |\frac{1}{n} =0|=\frac{1}{n}<\epsilon
    \end{gather}

    wähle N:= $\lceil \frac{1}{\epsilon} \rceil +1$

\end{beweis}

\begin{example}
    $\epsilon = \frac{1}{100}$ , gesucht N mit $n \geq N$
    $\Rightarrow \frac{1}{n} < \frac{1}{100}$ wähle $N=101$\\


    \begin{schreibweise}
    $x_n$ hat den Grenzwert a Limes
    $\lim\limits_{n \rightarrow \infty}{x_n}=a$
    $x_n$ geht gegen a für n gegen Unendlich.
    \end{schreibweise}
\end{example}

\begin{definition}[Nullfolge]
    $x_n$ heißt Nullfolge ,wenn $\lim\limits{x_n}=0$ gilt.
\end{definition}

\begin{remark}

    Es ist leichter, die konvergente einer Folge zu beweisen, als den Grenzwert auszurechnen.

\end{remark}

\begin{example}
    $x_n = \dfrac{1}{3} + \big(\dfrac{11-n}{9-n}\big)^9$\\
Behauptung: $\lim\limits_{n \rightarrow \infty}{x_n}=\dfrac{-2}{3}$

\newpage
    \begin{lemma}
        \begin{gather}
            \lim\limits_{n \rightarrow \infty}{x_n+y_n}=
            (\lim\limits_{n \rightarrow \infty}{x_n}) +
            (\lim\limits_{n \rightarrow \infty}{y_n})
        \end{gather}
    \end{lemma}

    \begin{gather}
        =\lim\limits_{n \rightarrow \infty}{\bigg(\big(\frac{1}{3}\big)+\bigg(\frac{11-n}{9+n}\bigg)^9\bigg)}
        = \lim\limits_{n \rightarrow \infty}{\frac{1}{3}+
        \lim\limits_{n \rightarrow \infty}{\bigg(\frac{11-n}{9+n}\bigg)^9}}
    \end{gather}

    \begin{gather}
        = \frac{1}{3} + \bigg(\lim\limits_{n \rightarrow \infty}{\frac{11-n}{9+n}}\bigg)^9
    \end{gather}


    \begin{gather}
        = \frac{1}{3} + \lim\limits_{n \rightarrow \infty}{\Bigg(\frac{n(\frac{1}{n}-1)}{n(\frac{9}{n}+1)}\Bigg)^9}
    \end{gather}

    \begin{gather}
        = \frac{1}{3}+\Bigg(\frac{\lim\limits_{n \rightarrow \infty}{(\frac{11}{n})}}{\lim\limits_{n \rightarrow \infty}{(\frac{9}{n}+1})}\Bigg)^9
    \end{gather}

    \begin{gather}
        = \frac{1}{3} + \Bigg(
        \frac{\lim\limits_{n \rightarrow \infty}{\frac{11}{n}} - \lim\limits_{n \rightarrow \infty}{1}}{\lim\limits_{n \rightarrow \infty}{\frac{9}{n}+\lim\limits_{n \rightarrow \infty}{1} } }   \Bigg)^9
    \end{gather}


    \begin{gather}
        =\Bigg(
        \frac{\lim\limits_{n \rightarrow \infty}{11} \times \lim\limits_{n \rightarrow \infty}{(\frac{1}{n})-1}}{\lim\limits_{n \rightarrow \infty}{9 \times \lim\limits_{n \rightarrow \infty}{(\frac{1}{n})+1} } }   \Bigg)^9
    \end{gather}

    \begin{gather}
        \frac{1}{3}+(-1)^9 = \frac{1}{3}-1 = \frac{-2}{3}
    \end{gather}
\end{example}

\begin{definition}[Unendliche Grenzwert]
    Eine Folge $(x_n)$ hat den unendliche Grenzwert $\infty$, wenn gilt : \\
    \[\forall r \in \mathbb{R} \quad \exists N \in N \quad \forall n \geq N : x_n > r \]

\begin{schreibweise}
$\lim\limits_{n \rightarrow \infty}{x_n}= \infty$
\end{schreibweise}
\end{definition}

\begin{remark}
    $\infty$ ist keine Grenzwerte und keine reelle Zahl.
\end{remark}

\begin{remark}
    Grenzwertsätze gelten nicht für uneigentliche Grenzwerte.
\end{remark}

\begin{remark}
    gilt $\lim\limits_{n \rightarrow \infty}{x_n}= \infty$ dann schreibt man $\lim\limits_{n \rightarrow \infty}{x_n}= -\infty$
\end{remark}

\begin{example}
    $x_n$ mit $x_n = q^n$ , $q \in \mathbb{R}$ , $q$ fest.\\

    $ \lim\limits_{n \rightarrow \infty}{q^n} = \begin{cases}
                                                    0 ,\quad |q|<1 \\
                                                    1 ,\quad |q|=1 \\
                                                    \infty ,\quad\quad q > 1  \\
                                                    ex. nicht ,\quad q\leq -1
    \end{cases}$
\end{example}

\vfil
\vfil


\section{Konvergenzkriterien}
(zum Beweis der Existenz eine Grenzwert, nicht zum berechnen von Grenzwert) \\


(1) $x_n$ konvergent $\Rightarrow$ $(x_n)$ beschränkt. \\

wenn $(x_n)$ nicht beschränkt $\Rightarrow$ $(x_n)$ nicht konvergent.\\


(2) Monotonie Kriterium:
wenn $(x_n)$ beschränkt ist können wir fragen ob $(x_n)$    konvergent.\\


$(x_n)$ beschränkt von Monotonie $\Rightarrow$ $(x_n)$ konvergent.

\begin{example} %2.17 

$\Big((-1)^n \times\frac{1}{n} \Big)$ konvergent (Nullfolge) diese Folge ist beschränkt aber nicht Monoton 
$$ \lim_{n \to \infty}{\Big( \big(1 + \frac{1}{n}\big)^n \Big)} $$  existiert. Diese ist beschränkt und monoton. \\
$$\Rightarrow \lim_{n \to \infty}{(1+\frac{1}{n})^n}$$ 
existiert. 
$$\lim_{n \to \infty}{(1+\frac{a}{n})=e^a}$$
 \end{example}
\newpage

\include{v3}
%%
%% Author: mahmoud
%% 19/04/18
%%


%fehlend ein Beispie mit g=A 
%2A=1^2+(1/2)^2

\section{Vorlesung 4 \href{https://tu-dresden.de/mn/math/algebra/das-institut/beschaeftigte/antje-noack/ressourcen/dateien/v120-1/MathMethInf04.pdf?lang=en}{(16.04.2019)} }
\section{Reihen}
\begin{example}{Zur geometrischen Reihen}\\
gesucht : A
\[2A = 1^2 + (\frac{1}{2})^2+(\frac{1}{4})^2 + \dots + (\frac{1}{k})^2 + ... \]\\
\[= (\frac{1}{4})^0 + (\frac{1}{4})^1+ (\frac{1}{4})^2+(\frac{1}{2^2})^3+(\frac{1}{2^2})^k+ \dots \]\\
\[9 = \frac{1}{4} = \frac{1}{1- \frac{1}{4}} = \frac{1}{\frac{3}{4}} = \frac{4}{3}= 2A \Rightarrow A = \frac{2}{3}\] 
\end{example}

\begin{example}
\begin{equation}
\begin{aligned}
0,4 \overline{3} = \frac{3}{4} + \frac{3}{100} + \frac{3}{10000}+ \dots \\
\frac{4}{10} + \frac{3}{100}(\frac{1}{10})^0 + \frac{1}{10} + \frac{3}{10^2} + \dots \\
=\frac{4}{10} + \frac{3}{100} \times \frac{1}{1-\frac{1}{10}}\\
= \frac{4}{10}+ \frac{1}{30} = \frac{12+1}{30}= \frac{13}{30}
\end{aligned}
\end{equation}
wenn $0,4 \overline{3}$ erlaubt wäre, dann,\\
\[\frac{4}{10} + \frac{9}{100} \times \frac{10}{9} = \frac{4}{10} + \frac{1}{10} = \frac{5}{10} = \frac{1}{2} = 0.5  \]
\end{example}

\newpage
\begin{example}
	\[\sum_{R=1}^{\infty} \frac{1}{K} \text{ ist divergent ,denn  } \lim\limits_{\infty} \sum_{K=1}^{n} \frac{1}{k} \text{ ex. nicht } \]
	
	\[\frac{1}{1}+\frac{1}{2}+\frac{1}{3}+\frac{1}{4}+\frac{1}{5}+\frac{1}{6}+\frac{1}{7}+\frac{1}{8}+\frac{1}{9}+\dots+\frac{1}{16}+\dots+\frac{1}{n}\]
	
	\[ > 1+\frac{1}{2}+  \underbrace{\frac{1}{4}+\frac{1}{4}}+\underbrace{\frac{1}{8}+\frac{1}{8}+\frac{1}{8}+\frac{1}{8}}+\frac{1}{10}+\frac{1}{16}+\dots+\frac{1}{n}\] 
	
	\[1+\frac{1}{2} + \qquad \frac{1}{2}+   \quad  \qquad \frac{1}{2}  + \dots + \frac{1}{n}  \rightarrow \lim\limits_{n\rightarrow \infty }s_n = \infty \]
	
\end{example}
\section{Allgemeine harmonische Reihe}

				\begin{equation*}
				\begin{split}
				 \sum_{K=1}^{\infty} \frac{1}{k^\alpha} \quad( \infty \quad \text{fest} ,\quad mit \quad \alpha \in \mathbb{R}) \qquad
				&\text{falls} \quad \alpha \geq 1 \rightarrow \text{konvergent}  \\
				&\text{falls} \quad \alpha \leq 1 \rightarrow \text{Divergent} 
				\end{split}
				\end{equation*}
 \begin{example}
 	\[ \sum_{k=1}^{\infty} \frac{1}{k^2} \qquad \text{ist konvergent}\]\\
 	\[ \sum_{k=1}^{\infty} \frac{1}{k^{\frac{1}{2}}} \qquad \text{ist Divergent}\]
 	
\begin{beweis}[Monotoniekriterium]
{mit Monotoniekriterium für Folge}  
		\[ \text{Reihe ist konvergent} \left\{ \begin{array}{ll}
	   (1) & \mbox{ $\sumIn \frac{1}{k^2} $ \quad ist monoton wachsend}.\\
	   (2) & \mbox{$\sumIn \fracKetwo $ \text{ist beschränkt}}.\end{array} \right. \] 
	   	
	   \[ \sumIin \fracKetwo = \frac{1}{1}+\frac{1}{2^2}+\frac{1}{2^2}+\frac{1}{4^2}+\frac{1}{5^2}+  \dots + \frac{1}{8^2}  \]
	   \[ <  1 + \frac{1}{4}+\underbrace{\frac{1}{2^2}+\frac{1}{2^2}}_{2. \frac{1}{4}}+\underbrace{\frac{1}{4^2}+\dots + \frac{1}{4^2}}_{4.\frac{1}{4^2}}+\]
	   
	   \[ 1+\frac{1}{4}+\frac{1}{2}.1+\underbrace{\frac{1}{4}}_{(\frac{1}{2})^2}+\underbrace{\frac{1}{8}}_{(\frac{1}{2})^3} =1+\frac{1}{4}+\frac{\frac{9}{4} }{1-\frac{1}{2}-1} \]
	   
 \end{beweis}
 \end{example}
%photo auf der Tafel
\section{Expotentiale Reihe}

\[ \sumOin \frac{1}{k!} = \limNin ( 1-\frac{1}{n})^n =: e \quad \text{ist konvergent} \]

 \pgfkeys{/forest,
	tria/.append style={ellipse, draw},
}
	\begin{forest}
	for tree={l+=0.5cm, s sep+=1cm}
	[Konvergentkreterium für Reihen 
	[	für Folge ]
	[Kreterien für absolute konvergenz [konvergent Reihen] ]
	[Hauptkriterium ]
	]
	\end{forest}

\section{Hauptkriterium}
$$ \star \quad \text{konvergent die Reihe} \quad \sum_{k=0}^{\infty}{a_k} \quad \text{dann ist} \quad (a_k) \quad\text{Nullfolge.}$$ \\
$$\lim\limits_{k \rightarrow \infty}{a_k} \neq 0 \Rightarrow  \sum_{k=0}^{\infty}{a_k} \quad \underbrace{null konvergent }_{divergend}$$\\
oder
			$$\lim\limits_{k \rightarrow -\infty}{ a_k} \quad ex.null$$

						\begin{example}
						
						\[ \sumIin \frac{3k^2+1}{4k^2-1} \quad divergend, \quad aber \quad \sumIin \fracKone \quad divergend \quad \text{und $\fracKone$ Null folge }  \]
						

						\end{example}
\begin{beweis}
\[  \sumOin a_k \quad (konvergent) \Rightarrow \underbrace{(a_k) \quad Nullfolge }_{\limKin a_k=0}  \] 
\[s_n =\sumOn a_k \quad ,\quad s_{n+1} =\sum_{k=0}^{n+1}{a_k} \quad ,\quad s_{n+1}= s_n + s_{n+1} \] \\
\[ s = \limNin s_n= \limNin s_{n+1}  \quad ,\quad \limNin s_{n+1} = \limNin s_{n+1} - \limNin s_n = s-s=0 \]
\end{beweis}

\newpage
\section{Kriterium für Alternierende  Reihe}
\begin{beweis}[Alternierende Reihe]
	$$\sumOin (-1)^k \fracKone \text{ist konvergent }$$
	$$ \sumOin (-1)^k a_k $$
	wobei $(a_k)$ einer Streng monoton fallend Nullfolge mit $a_k \geqq 0 $ \\
	$\Rightarrow$ Die Reihe ist konvergent. $$ Also \sumOin (-1)^k \fracKone \quad \text{ist konvergent.} $$ 

\end{beweis}
\begin{definition}[absolute Reihe]
	
$$ \text{Eine Reihe}  \sumOin  a_k \quad \text{heißt absolute konvergent wenn} \sumOin |a_k| \text{konvergent ist.
	}$$
\end{definition}
\begin{example}

\[	\sumIin (-1)^k \fracKone \text{ ist konvergent , aber nicht absolote konvergent } \]

\[ \sumIin (-1)^k \fracKetwo  \text{ ist kovergend und \textbf{absolute} konvergent } \]
\end{example}


\begin{theorem}
$$ Reihe \sumOin a_k \quad \text{absolut konvergent} \quad \Rightarrow \quad \text{Reihe} \sumOin a_k \quad \text{ist Konvergent} $$	
\end{theorem}

\begin{remark}	
absolute konvergente Reihe kann man multiplizieren wie endliche summen. (aber konvergente Reihen nicht !)
\end{remark}

\section{Quotienkriterium (QK):}
Für absolute  Konvergenz , wenn gilt:\\

\begin{align*}
\limKin \bigg|\frac{a_{k+1}}{a_k} \bigg|
\begin{cases}
< 1 \Rightarrow  \sum_{k=0}^{\infty} \quad \text{ist absolut konvergent}\\
> 1 \Rightarrow \quad \text{ist divergent)}\\
= 1 \Rightarrow \quad \text{Kriterium ist nicht anwendbar} 
\end{cases}
\end{align*}

\section{Wurzel Kriterium : WK }
Die Reihe $\sumOin a_k$ ist  \textbf{absolute} konvergent genau wenn $\Leftrightarrow$ :
\begin{align*}
\limKin \sqrt[k]{|a_k|}
\begin{cases}
< 1 \Rightarrow \sumOin a_k \quad \text{ist absolut konvergent}\\
> 1 \Rightarrow \quad \text{ist divergent}\\
= 1 \Rightarrow \quad \text{Kriterium ist nicht anwendbar} 
\end{cases}
\end{align*}

\begin{example}[QK]
	    \begin{gather*}
	    	\sumOin \dfrac{1}{k!}\\
	 	   \limKin {\bigg|\frac{\frac{1}{(k+1)!}}{\frac{1}{k!}}\bigg|}  \\
	 	   = \limKin \frac{k!}{(k+1)!} \\
													   	   =\limKin \frac{1}{k+1} = 0   \\
													   	  d.h \quad <1 \Rightarrow \quad \text{Die Reihe ist absoulte Konvergent.}  
		    \end{gather*}     
	
\end{example}
\begin{example}[WK]
	\begin{gather*}
	\limKin \sqrt[k]{\big|\frac{1}{k!}\big|} \\
=\limKin \frac{\sqrt[k]{1}}{\sqrt[k]{k!}}
=\\
	\frac{1}{\limKin \sqrt[k]{k!}}=0 <1 \\
	\text{Die Reihe is absolut konvergent.}
	\end{gather*}	 
	     
	
\end{example}



 
\listoftheorems[ ignoreall,show={definition}]
\listoftheorems[ignoreall,show={example}]
\end{document}
