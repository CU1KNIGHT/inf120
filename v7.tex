\section{Vorlesung 7}

\begin{align*}
&	f:D \rightarrow \mathbb{R} , D \subseteq \mathbb{R} , a \notin D &\\
&\underbrace{\lim_{x\to a}}_{x \neq a} f(x)= r \in \mathbb{R}  \Leftrightarrow \forall (x_n) \limNin x_n= a \text{  und  } x_n \in D \\
&\Rightarrow \limNin f(x_n)=r
\end{align*}
\begin{example}

GWS nicht anwendbar $\limXo \overbrace{ x \sin x }^{f(x)} = \limXo x  .\quad \limXo \sin x=0. 0=0$
\end{example}
\begin{remark}
GWS nicht anwendbar $\underbrace{\limXo(x \ \sin \frac{1}{x}}_{f(x)} \underbrace{\limXo}_{0},\limXo \sin \frac{1}{x}$
\end{remark}
\begin{definition}
	Sei $f: (a,b) \to \mathbb{R} , x_0 \in (a,b) $\\
	 $x_0 \in (a,b) \Leftrightarrow  x_0 \in \mathbb{R} $  und  $ a<s_0 < b ( \Leftrightarrow (skizze  not complate))$\\
	$f$ ist in $x_0$  differenzierbar : $\Leftrightarrow f'(x_0) := \ \underbrace{\lim\limits_{x \rightarrow f(x)}}_{f \neq x_0} \frac{f(x) -f(x_0)}{x-x_0}$ existiert  $(f'(x_0) \in \mathbb{R}) $\\
	Falls der Grenzwert ex., nennt man $f'(x_0)$ die erste Ableitung von $f$ in $x_0$.\\
	Existiert $f'(x_0)$ für alle $x_0 \in (a,b)$ , dann nennt man $ f':(a,b) \rightarrow \mathbb{R} \longmapsto f'(x_0)$ die erste Ableitung von $f$.

	

\end{definition}
\begin{example}
	$f(x) =\frac{1}{x}$ auf $(0,r)$ $r\in \mathbb{R}_{>0}$ ,$r$ fest und $x_0 \in (0,)r $, ges: $ f'(x_0)$
	\begin{align*}
			f'(x_0) =
		\underbrace{\lim\limits_{x \to x_0}}_{x \neq x_0} \frac{\frac{1}{x}-\frac{1}{x_0}}{x-x_0}= 
		\underbrace{\lim\limits_{x \to x_0}}_{x \neq x_0} \frac{\frac{x_0-x}{x.x_0}}{x-x_0}=
		\underbrace{\lim\limits_{x \to x_0}}_{x \neq x_0} \frac{(x_0-x)}{x-x_0(x-x_0)}=
		\underbrace{\lim\limits_{x \to x_0}}_{x \neq x_0} \underbrace{(-\frac{1}{x_0})}_{konst.}\frac{1}{x}=
		-\frac{1}{x_0} \underbrace{\lim\limits_{x \to x_0}}_{x \neq x_0} \frac{1}{x}\\
		\overset{\overset{\frac{1}{x} stetig F.}{\downarrow}}{=}
		-\frac{1}{x}. \frac{1}{x}= - \frac{1}{x^2}
	\end{align*}
	\begin{flalign*}
		f' : (0,r) \rightarrow \mathbb{R} : x \longmapsto -\frac{1}{x^2} \text{in die erste Abbildung von } f(x)= \frac{1}{x}
	\end{flalign*}
  

\end{example}
\subsection{tafelwerk}   
$f \qquad \qquad$					 $f'$\\ \\
$x^n \qquad \qquad$				   	$nx^{n-1}$\\
$\downarrow n=-1 \qquad $		$ \downarrow $ \\
\\
$\frac{1}{x} \qquad \qquad$					$-\frac{1}{x^2}$
\begin{theorem}
	$f$ in $x_0$ differenzierbar $\Rightarrow$ $f$ in $x_0$ stetig
\end{theorem}
\begin{proof}
Sei $f$ in $x_0$ d.b $\Rightarrow$ $f'(x_0) = \limXin \frac{f(x) - f(x_0)}{x-x_0 }$ ex.
...
%fehlt den Rest von proof
\end{proof}

\begin{tikzpicture} 
  \newcommand*\funktion[1]{2*sin(0.5*deg(#1)) + 1.5}% dargestellte Funktion
  \newcommand*\ableitung[1]{cos(0.5*deg(#1))}% Ableitung der Funktion
  \newcommand*\tangente[2]{\ableitung{#2}*(#1-#2)+\funktion{#2}}

  \begin{axis}[axis lines=middle,,enlargelimits,
    xlabel=$x$,xlabel style={anchor=north},xtick=\empty,
    ylabel=$y$,ylabel style={anchor=east},ytick=\empty
  ]
  \addplot[domain=-1:10,samples=200]{\funktion{\x}};
  \addplot[domain=4:8]{\tangente{\x}{6}};
  \coordinate (P) at (axis cs:6,{\funktion{6}}) ;
  \coordinate (Q) at (axis cs:0,{\funktion{6}}) ;
  \coordinate (R) at (axis cs:6,0) ;
  \node[coordinate,pin=30:{$(x_0,f(x_0))$}] at (P) {};
  \draw[red,dotted] (P) -- (Q) node[left] {$f(x_0)$} ;
  \draw[red,dotted] (P) -- (R) node[below] {$x_0$} ;
  \end{axis}
\end{tikzpicture}

Die Linie repräsentiert die Tangente ((T)) an den Grenzwert von $f(x_0)$ im Punkt $(x_0 , f(x_0))$

\begin{gather*}
(x) = \dfrac{t(x)-t(x_0)}{x - x_0} = \dfrac{f(x)-f(x_0)}{x - x_0}
\end{gather*}

\subsection{Tangente Gleichung}
\begin{align*}
T : t(x) = f(x_0) + f'(x_0)(x-x_0)
\end{align*}

\begin{remark}
$f(x)$ gibt die Ableitung der Tangente an den Grenzwert der Funktion $f$ im Punkt $x_0 , f(x_0)$ an.
\end{remark}

\section{Berechnen an $f'(x)$ Ableitungsregeln:- }

\subsection{Linearität:-}
Sei $\underbrace{f(x)$ und $f(g)}_{h'(x)}$ gegeben sind , dann wie sieht die Ableitung von $h'(x)$ ?

\[(f(x)+g(x))'=f'(x)+g'(x)\]
\[ \underbrace{r f(x)'}_{h(x)}=\underbrace{r}_{\in \mathbb{R}}f'(x)\]
\begin{align*}
f'(x_0)=\dfrac{f(x)-f(x_0)}{x-x_0} = \lim \dfrac{f(x)-f(x_0)+g(x)-g(x_0)}{x-x_0} = f'(x_0)+g'(x_0)
\end{align*}
 
\subsection{Produktregel:-}
\[ (f(x) . g(x))' = f'(x) . g(x) + f(x) . g'(x)  \]
\subsection{kettenregel:-}
\begin{align*}
\underbrace{(f \circ g )'(x)}_{f(g(x))'} = f'(g(x)). g'(x)
\end{align*}
\subsection{Quotientenregeln:-}
In Tafelwerk : \\
\[ (\dfrac{f}{g})' = \dfrac{f'.g - f.g'}{g^2}\]
Herleitung :
\begin{align*}
\bigg(\dfrac{f(x)}{g(x)}\bigg)' =
\bigg(f(x)\frac{1}{g(x)}\bigg)'\\
&=f'(x)\frac{1}{g(x)} + 
f(x)\bigg(\frac{1}{g(x)}\bigg)'\\
&=\dfrac{f'(x).g(x)-f(x).g'(x)}{g(x)^2}
 \end{align*}
 
\begin{remark}[Tafelwerk]
\[(sin(x))'=cos(x)\]
\[(cos(x))'= - sin(x)\]
\end{remark}
\begin{example}
\begin{align*}
(tan(x))' &= \bigg( \dfrac{sin(x)}{cos(x)}\bigg)'\\
 &=  \dfrac{cos(x)cos(x)-sin(x)(-sin(x))}{(cos(x))^2}\\
 &=\dfrac{(cos(x))^2+(sin(x))^2}{(cos(x))^2}\\
 &= \dfrac{1}{(cos(x))^2}\\
 &= 1 + (tan(x))^2
\end{align*}
\end{example} 
\subsection{Ableitung der Umkehrfunktion $f^-1$ zu $f$ }
\begin{definition}
Ist $ y = f(x) $ eine umkehrbare differenzierbare Funktion, dann ist die Umkehrfunktion $ x = g(y)$ differenzierbar und es gilt: 
$g'(y)= \frac{1}{f'(g(y)}$ oder $\frac{dx}{dy}= \frac{1}{\frac{dy}{dx}}$ für $f'(x)\neq 0$.
Überlicherweise verraucht man die Variablen $x , y$ and schreibt $y = g(x)$ und $y'=g'(x)$. 
\end{definition}

\begin{example}
$f(x)=e^x$\\
$f'(x)=e^x$\\

\begin{proof}
Der Beweis ist einfach.Man geht wider von der Definition der Ableitung aus:
\begin{align*}
f'(x)= \lim_{h \to 0}{\frac{f(x+h)-f(x)}{h}}= \lim_{h \to 0}{\frac{e^{x+h}-e^x}{h}}
\end{align*}
Nutzt man die Potenzregln $e^{x+h}=e^x.e^h$ so ergibt sich : 
\begin{align*}
f'(x)= \lim_{h \to 0}{\frac{e^x.e^h-e^x}{h}}=e^x \lim_{h \to 0}{\frac{e^h-1}{h}}=1
\end{align*}
und weil $\lim_{h \to 0}{\frac{e^h-1}{h}}=1$ dann Also $f'(e^x)=e^x$
\end{proof}
\end{example}

\begin{remark}
\begin{align*}
f \circ f^{-1} = f^{-1} \circ f &= \quad \text{identisch}\\
e^{ln(x)} &= x \quad | \text{Abb} \\
e^{ln(x)}.(ln(x))' &= 1\\
\Rightarrow ln(x)'&= \frac{1}{e^{lnx}}= \frac{1}{x}
\end{align*}
\end{remark}

\begin{example}
\begin{align*}
f(x)=e^x\\
f'(x)=e^x\\
f^{-1}(x)=lnx\\
(f^{-1}(x))'=(lnx)=\frac{1}{x}
\end{align*}
\end{example}

\begin{example}
\begin{gather*}
f(x) = tan(x) \Rightarrow f'(x) = 1+(tan(x))^2\\
f^{-1}(x)= arctan(x) = x  | \quad Abl.\\
\Rightarrow 1 + ( \underbrace{tan(arctan x)}_{x})^2(arctanx)' = 1 \Rightarrow (arctanx)'=\frac{1}{1+x^2}
\end{gather*}
\end{example}

