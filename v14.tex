\section{Vorlesung 14}
\centering
\begin{forest}
	for tree={l+=0.5cm, s sep+=1cm}
	[Differenzialgleichung (Dgl) 
	[ gewöhnlich [gesucht : $ \underbrace{y(x)}_{Funktion}$ ] ]
	[partielle [ gesucht :   $ y(x_1 \,\ x_2  \dots  x_n)$ ] ]
	]
\end{forest}

\begin{definition}
	$y'(x)= f (x , g(x))$ heißt gewöhnliche Differenzialgleichung erste Ordnung \textbf{in explizite Form }
	$y : \underbrace{I}_{\text{ Intervall I }\in \mathbb{R}(\subseteq \mathbb{C}) }$ heißt lösung , wenn sich beim Einsetzen von $y(x)$ und dem Ableitung der Funktion in der Differenzialgleichung eine wahre Aussage gibt.   
\end{definition}
\begin{example}
	$y' = x - y^2$ auffüllbar $y'(x) = x(y(x))^2 $ $x > 0$
	\textbf{gesucht : $y(x)$} bei Lösungen dieser DGL sind.
	$y_1 = \frac{-2}{x^2} $ auf $(0,\infty)$ , $y_2(x)\frac{2}{2-x^2}$ auf $(0 , \sqrt{2})$ , $y_3(x) = \frac{2}{2-x^2}$ auf $(\sqrt{2}, \infty)$ 
\end{example}
\begin{enumerate}
	\item  $\underbrace{(L.S)}_{linke Seite} y' = (-2)(-2)\frac{1}{x^3} , \underbrace{R.S}_{right Seite} x\frac{-2}{x^2} \times \frac{-2}{x^2}$ , $L.S = R.S$ 
	\item  L.S $y_2' = 2 (\frac{1}{2-x^2})= 2(-2) \frac{1}{(2-x^2)^2 \times (-2x)},$ R.S $x(\frac{2}{2-x^2})^2$ \dots Selbst Studium\\
	ende Ergebnis soll L.S = R.S 
	
	
	\item \textbf{grafische Lösung von} 
	$y'= f(x ,y)$ , $\exists_m Punkt (x , y)$  ist eine  Anstieg\\
	\textbf{Anstieg} der Tangente an den Graphen einer Lösungen , $ y(x)$ im Punkt $(x,y)$. 
\end{enumerate}
\textbf{Linearelement } $(x , y)$\\
\textbf{Richtungsfeld} : Menge aller Linearelement\\
\textbf{Isokline} : verbinden punkte gleichen Anstieg 


\begin{example}
	$y'= x $ \begin{enumerate}
		\item Isoklinen bestimmen 
		\item  Lösungsklaieren y(x) in das Richtungsfield eintragen
	\end{enumerate}
y'=c =const $\Rightarrow x=c$ z.B.\begin{align*}
c=0 , c=1 , c=3 ,c=-1,\\
 x=0, x=1 ,x=2, x =-1
\end{align*} 
$y_1,y_2, y_3$ sind Bsp. für Lösungen$\underbrace{ \text{Alle lösungen}}_{\text{Allgemeine Lösung } y(x,\underset{\underset{\scriptstyle\text{reeller Parameter } c \in \mathbb{R} }{\scriptstyle\downarrow}}{C} )}$ bilden die lösungskurvenschar 
Wählt man ein festes $C \in R$, dann erhält man eine singuläre Lösung.
(fehlt Skizze !)
\end{example}
\begin{example}
	$y'=\frac{-x}{y} mit y(x)ist nicht die Funktionen y \neq 0$\\
	(fehlt skizze!)
	\begin{align*}
	y'= c=const  -\frac{x}{y}=C \Rightarrow y= -\frac{x}{c}\text{ z.B.} &c=1 \ c=-1 \\ 
																   &y=\dots , y=\dots
	\end{align*}

	
\end{example}
\begin{definition}
$y^{n}(x)=f(x,y(x),y'(x),\dots, y^{n-1}(x) )$ glo. Dgl. n-te ordn, Lösung $ y(x):I\rightarrow \mathbb{R}$\\
Allg. Lösung :$ y(x,c_1,c_2,\dots,c_n) $ und $ c_1,c_2,\dots ,c_n \in \mathbb{R}$
für Konkrete werte für ; erhällt wenn spezielle Lösungen
\end{definition}
\section{Anfangswert-Aufgabe}
( Spezielle Bedieniegung  zur Bestimmung der Parameter )
$y(x_0)=r_1$\\
$y'(x_0)= r_1$\\
$y^{(n-1)}(x_0)=r_{n-1}$

\begin{align*}
\text{ n Bedingungen }
\begin{cases}
y(x_0)=r_1\\
y'(x_0)= r_1\\
y^{(n-1)}(x_0)=r_{n-1}
\end{cases}
\end{align*}

Methode:Trennung der ceränderlichen für Dgl.$ y'=h(x)g(y(x))$

\begin{example}
	\begin{align*}
 y'=y \Rightarrow \frac{dy}{dx} =y \Rightarrow dy=y.dx | :y \text{( für  }y \neq 0) 
	\end{align*}
\begin{enumerate}
	\item Fall :$ y \equiv 0$ \\	
	\begin{align*}
		\text{ l.s =r.s. }
	\begin{cases}
	l.s. y' \equiv 0 \\
	r.s. y \equiv 0
	\end{cases}	
	\Rightarrow y\equiv 0\text{ ist eine Lösung.}
\end{align*}
	\item Fall: 
	 \begin{flalign*}
 y \equiv 0 \dots \Rightarrow \frac{dy}{y}=dx \Rightarrow \int \frac{dy}{y}= \int dx &\Rightarrow \frac{dy}{y}=dx \Rightarrow \int \frac{dy}{y}= \int dx \\
 &\Rightarrow \ln |y| =x+k, k\in \mathbb{R}\\
 &\Rightarrow|y|=e^{x+k}=e^x.\underbrace{e^k}_{>0} \Rightarrow y =\pm\underbrace{ e^K}_{>0}\Rightarrow .e^k \\ &\Rightarrow y= C .e^x\text{ und }c \in \mathbb{R}\backslash \{ 0\}
\end{flalign*}
\end{enumerate}
$y'=y$ hat die allgemeine Lösung $y(x)=C.e^x (c\in \mathbb{R})$ $\underbrace{Probe}_{y'=C.e^x=y}$ erforderlich !

\end{example}