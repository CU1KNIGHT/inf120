\section{Vorlesung 14}
\centering
\begin{forest}
	for tree={l+=0.5cm, s sep+=1cm}
	[Differenzialgleichung (Dgl) 
	[ gewöhnlich [gesucht : $ \underbrace{y(x)}_{Funktion}$ ] ]
	[partielle [ gesucht :   $ y(x_1 \,\ x_2  \dots  x_n)$ ] ]
	]
\end{forest}

\begin{definition}
$y'(x)= f (x , g(x))$ heißt gewöhnliche Differenzialgleichung erste Ordnung \textbf{in explizite Form }
$y : \underbrace{I}_{\text{ Intervall I }\in \mathbb{R}(\subseteq \mathbb{C}) }$ heißt lösung , wenn sich beim Einsetzen von $y(x)$ und dem Ableitung der Funktion in der Differenzialgleichung eine wahre Aussage gibt.   
\end{definition}
\begin{example}
$y' = x - y^2$ auffüllbar $y'(x) = x(y(x))^2 $ $x > 0$
\textbf{gesucht : $y(x)$} bei Lösungen dieser DGL sind.
$y_1 = \frac{-2}{x^2} $ auf $(0,\infty)$ , $y_2(x)\frac{2}{2-x^2}$ auf $(0 , \sqrt{2})$ , $y_3(x) = \frac{2}{2-x^2}$ auf $(\sqrt{2}, \infty)$ 
\end{example}
\begin{enumerate}
\item  $\underbrace{(L.S)}_{linke Seite} y' = (-2)(-2)\frac{1}{x^3} , \underbrace{R.S}_{right Seite} x\frac{-2}{x^2} \times \frac{-2}{x^2}$ , $L.S = R.S$ 
\item  L.S $y_2' = 2 (\frac{1}{2-x^2})= 2(-2) \frac{1}{(2-x^2)^2 \times (-2x)},$ R.S $x(\frac{2}{2-x^2})^2$ \dots Selbst Studium\\
 ende Ergebnis soll L.S = R.S 
 
      
\item \textbf{grafische Lösung von} 
$y'= f(x ,y)$ , $\exists_m Punkt (x , y)$  ist eine  Anstieg\\
\textbf{Anstieg} der Tangente an den Graphen einer Lösungen , $ y(x)$ im Punkt $(x,y)$. 
\end{enumerate}
\textbf{Linearelement } $(x , y)$\\
\textbf{Richtungsfeld} : Menge aller Linearelement\\
\textbf{Isokline} : verbinden punkte gleichen Anstieg 

