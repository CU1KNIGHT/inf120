\section{Vorlesung 18}
\section{( Totale ) Differenzeierbarkit für Funktionen in zwei Veränderlichen}
$y = f(x_1 , \dots x_n)\rightsquigarrow \text{ partielle Ableitung }$\\
$f(x_i) = \dfrac{d \partial}{ \partial x_i }$ ( wo bei $x_j$ mit $ j \neq i $ konstant )\\
- existiert eine Partielle Ableitung $\Rightarrow f $ ist partielle  Differenzeierbar.\\
Def. $\Rightarrow$ f ist ( total ) differenzierbar.
lineare Approximierbarkeit $\Leftrightarrow f$ ist (total ) differenzierbar. 
\begin{definition}  
Sei $f : X \rightarrow \mathbb{R}$ , $X \subseteq R^n$ , $x_0 \in X$ : $( x_{01} = (x_0 , \dots , x_{0n}))$ Dann heißt f an der stelle $x_0$ total Differenzierbar wenn gilt :\\
\begin{enumerate}
\item $\exists a \in \mathbb{R^n}$ \begin{equation*}
f(\underline{x})= f(\underline{x_0}) + \underline{a} (x-x_0) + R(\underline{x}) \quad (x \in X) \text{ mit } \lim_{x \to x_0} \dfrac{|R_n(\underline{x})|}
{\lVert \underline{x} - x_0 \rVert} = 0
\end{equation*}
\[ \text{ für } n = 1 \Rightarrow f(x)= f(x_0) + f'(x_1)(x-x_0) + R(x) \]
\end{enumerate}
\end{definition}
\begin{remark}
Ist $f$ in $x_0$ (total) differenzierbar ? dann gilt\\
$f(\underline{x}) \underbrace{ \approx f(x_0) + \underline{a}( x - x_0)^1}_{\text{ mittels Aprroximation von f  }}$
\end{remark}
\begin{remark}
Ist f in $x_0$ linear approximierbar , dann kann $\underline{a}$ wie folgt berechnet werden.
\begin{equation}
a = \begin{pmatrix}
f_{x_1}(\underline{x_0})\\
f_{x_2}(\underline{x_0}\\
\vdots\\
f_{x_n}(\underline{x_0}
\end{pmatrix} = 
\begin{pmatrix}
\dfrac{\partial f}{\partial x_1}\\
\dfrac{\partial f}{\partial x_2}\\
\vdots\\
\dfrac{\partial f}{\partial x_n}
\end{pmatrix}_{|x_0}
\end{equation}
\end{remark}
\begin{remark}
$\begin{pmatrix}
f_{x_1}\\
\vdots\\
f_{x_n}
\end{pmatrix}_{x_0}$ heißt Gradient von f an der Stelle $\underline{x_0}$
\end{remark}
\begin{remark}
f ist ( total ) differenzierbar $\Rightarrow$ f ist partiell differenzierbar.\\
f partiell differenzierbar und hat stetige Ableitung  $\Rightarrow$ f (total) differenzierbar. ($\Leftarrow$ nicht ex.)
\end{remark}
\section{Verallgemeinerte Kettenregel
}
$z = f(x ,y)$ $ x = x(t)  $ $ y = y(t)  $\\
$ f(x(t) , y(t) ) = \varphi(t) $
\begin{example}
$z= f(x,y)= x^2y$ ,  $x(t) = t^2 $
 , $y(t) = t$\\
$\varphi(t)=(t^2)^2 t = t^5 $ Ableitung ?\\
$f(x(t)) , y(t) = \varphi(t) $ Ableitung nach t\\
$\dfrac{d\varphi}{dt}(= \varphi') = f_x(x(t) , y(t)) \dfrac{dx}{dt} +
f_y(x(t) , y(t)) \dfrac{dy}{dt} $\\
\[\textbf{Kurz: } f_x \dfrac{dx}{dt} + f_y \dfrac{dy}{dt} = \dfrac{d \varphi}{dt}   \]
\end{example}
\begin{remark}
Für die Gültigkeit der Formel begütigt man stetige Differenzeierbarkit (für f nach x und y , für $\varphi$ nach t)  
\end{remark}
\begin{example}
$f(x,y)=x^2 y : x(t)=t^2$ , $y(t)=t = $\\
ableiten nach t 
\begin{align*}
\frac{d \varphi}{dt} &= 2xy \times 2t + x^2 \times 1 \times 1 \\
&=2t^2 \times t \times 2t + (t^2) \times 1\\
&= 4t^2 + t^4 = st^4
\end{align*}
\end{example}
\section{Taylorentwicklung für 
$ z = f (x, y)$ an der Stelle $(x_0, y_0)$}
$ f(x,y) = f( x_0 , y_0) + \dots $
$\varphi (1)  = f( \underbrace{ x_0 + 1 (x - x_0)}_{ = x}  , \underbrace{ y_0 +1 (y - y_0)}_{ = y} )$\\
$ \varphi(t) = f (x_0 + t \overbrace{( x - x_0 )}^{h} 
y_0 + t \overbrace{( y - y_0 )}^{k}) $\\
$x(t)=x_0 + th $ , $y(t)= y_0 +t \times k$
$\varphi(t)$ ist eine Funktion einer reellen Veränderlichen (t) Taylorentwicklung ist bekannt. 

\subsection{Taylorentwicklung für $\phi (t)$ an der Stelle $t_0=0$ }
\begin{flushleft}
	$\varphi (t) =\varphi (0) +\varphi (0).t+\frac{1}{2}\varphi''(0) t^2+\dots +\frac{1}{n!}\varphi ^{(n)} (0)t^n +R_n(t)$\\
für t=1\\
$f(x,y)=\varphi (1)= \varphi (0) +\varphi'(0)+\frac{1}{2}\varphi''(0) +\dots +\frac{1}{n!}\varphi ^{(n)}(0)+R_n(1)$\\
Weitere Scritte   t=0\\
\begin{align*}
	&\varphi'(t) \qquad &\varphi'(0)\\  
		&\varphi''(t) \qquad &\varphi''(0)\\  
			&\varphi'''(t) \qquad &\varphi'''(0)\\  
			& \dots \qquad &\dots  
\end{align*}
$\varphi(t) =f(x(t),y(t))  $ nach ableiten und dabei $\alpha $ die verallgemeinerte kettenregel bemerken\\
$\varphi '(t)= f_x \underbrace{\frac{dx}{dt}}_{n}+f_y\underbrace{\frac{dy}{dt}}_{k}$\\
$x(t)=x_0+t.h$   $\frac{dx}{dt}=0+1.h=h$\\
$y(t)=y_0+t.k$    $\frac{dx}{dt}=0+1.k=k$\\
$\varphi'(t)=f_x .h+f_y k$\\
\begin{align*}
\varphi''(t)= (\phi'(t))=& (f_x h+f_y k)\\
=&(f_{xx})h+f_{yx}k) h+(f_{xy}h+f_{yy}k)k\\
=& f_{xx}h^2+2f_{xy}h'.k'+f_{yy}k^2
\end{align*}
$\varphi'''(t)=1.f_{xxx}h^3  +3f_{xxy}h^2k+3.f_{xyy}hk^2+^.f_{yyy}k^3$


 

\end{flushleft}

\subsection{Lineare Näherungsformel:}
für f(x,y) an der Stelle$ (x_0,y_0)$ 
\[ \varphi (0) +\varphi'(0) = \underbrace{f(x_0+0(x-x_0),y_0(y-y_0))}_{f(x_0,y_0)} + f_x(x_0,y_0)\underset{h}{(x-x_0)}+f_y(x_0,y_0)\underset{k}{(y-y_0)} \]
$= f(x_0,y_0)+f_x(x_0,y_0)(x-x_0)+f_y(x_0,y_0)(y-y_0)$\\
$:=$ tangentialebene
\subsection{Quaratische Nääherungsformel:}
tangentialebene+$\frac{1}{2}(f_{xx}(x_0,y_0)(x-x_0)^2+2f_{xy}(x_0,y_0)(y-y_0)(x-x_0)+f_{yy}(x_0,y_0)(y-y_0)^2)
$
\begin{example}
	$f(x,y)=\sin x ,\sin y,(x_0,y_0)=(0,0)$\\Gesucht Tangentialebene:\\
	$f_x=\cos x \sin y $   $f_x(0,0)=0$\\
	$f_y=\sin x \cos y $ $f_y(0,0)=0$\\
	$t(x,y)=f_1(x,y)$\\ 
	$\qquad \quad =\underbrace{f(0,0)}_{0}+\underbrace{f_x(0,0)(x-0)}_{0}+\underbrace{f_y(0,0)(y-0)}_{0}=0$
\end{example}