\section{Vorlesung 10}
\section{Spezielle Ableitungen }
\begin{gather*}
f(x) = x^x \quad x>0\\
ln(f(x)) = \underbrace{ln  x^x}_{x \: ln  x } \Rightarrow \frac{1}{p(x)}f'(x) = 1 \times ln \ x + \underbrace{x \ \frac{1}{x}}_{1}\\
\Rightarrow f'(x) = \underbrace{x^x }_{f(x)}(ln \ x + 1) \text{ logarithmisches Differenzieren }\\
f(x) = x^x = e^{ln  x^x} = e^{x \ ln x} \Rightarrow f'(x)= \underbrace{e^{x \ ln  x}}_{x^x}(x \ ln  x)' = x^x (ln  x + 1) 
\end{gather*}
\begin{align*}
(\sinh  x)'&= \cosh  x\\
(\cosh  x)'&= \sinh x\\
\cosh x:&= \frac{e^x + e^{-x}}{2} \text{ Kosinus Hyperbolicus}\\
\sinh x:&= \frac{e^x + e^{-x}}{2}\\
\end{align*}

\begin{tikzpicture}[scale=1],

\begin{axis}[xlabel=$x$,
ylabel= $y$,
height=12cm,
width=16cm,
ymax=5,
ymin=-5,
xmin=-5,
xmax=5,
axis y line=center,
axis x line=center,
]
\addplot [ very thick,mark options={solid},red]{sinh(x)}  node at (-1.5, -4.5) {$\sinh x$};
\addplot [ very thick,mark options={solid},blue]{cosh(x)} node at (1, 3.5) {$\cosh x$};
\addplot[dashed]{x}node at (3.1, 4) {$"kettenlinie"$};
\end{axis}
\end{tikzpicture}
gesucht: \textbf{Taylor-Reihe Entwicklung} für $\cosh x$\\
\begin{align*}
e^x &= \sum_{ k = 0 }^{ \infty }{ \frac{x^k}{k!} }\\
e^{-x} &= \sum_{ k = 0 }^{\infty}{\frac{(-x)^k}{k!}}\\
&= \sum_{ k = 0 }^{\infty}{(-1)^k\frac{x^k}{k!}}
\qquad (x \in \mathbb{R})
\end{align*}
Reihe ist absolut konvergent für alle $ x \in \mathbb{R}$ \\
\[\cosh x= \frac{e^x + e^{-x}}{2}\]
\begin{gather*}
\rightsquigarrow \cosh x= \frac{1}{2}(1+\frac{x}{1!}+ \frac{x^2}{2!} + \frac{x^3}{3!}+\frac{x^4}{4!} + \dots )\\
+(1-\frac{x}{1!}+ \frac{x^2}{2!} - \frac{x^3}{3!}+\frac{x^4}{4!} + \dots)\\
=1 + \frac{x^2}{2!}+  \frac{x^4}{4!} = \sum_{k=0}^{\infty}{\frac{x^{2k}}{(2k)!}}
\end{gather*}
\section{Spezielle Grenzwerte}
\subsection{Regeln von Bernoulli l'Hospital -}
Seien $ f(x) $ , $g(x)$ reelle , zweimal stetig differenzierbare Funktionen auf $(a,b)$ und $f(x_0) = g(x_0) = 0$\\
gesucht: $$\lim_{x \to x_0}{\frac{f(x)}{g(x)}}$$\\
\begin{align*}
\dfrac{f(x)}{g(x)} &=
\dfrac{f(x_0) + f'(x_0)(x - x_0)+ \frac{1}{2}  f''(z_f)(x-x_0)^2}
{g(x_0) + g'(x_0) + (x - x_0) + \frac{1}{2}  g''(z_g)(x - x_0)^2 }\\
&= \dfrac{x - x_0}{x - x_0} . 
{ \dfrac 
{f'(x_0) + \frac{1}{2} f''(z_f )(x-x_0)}
{g'(x_0) + \frac{1}{2} g''(z_g )(x-x_0)} 
} \\
\lim\limits_{x \rightarrow x_0} \frac{f(x)}{g(x)}&=1.\lim_{x \to x_0}\dfrac{f'(x_0)+\frac{1}{2} f''(z_f).0}{g'(x_0)+\dots .0}\\
&= \lim_{x \to x_0}{\frac{f'(x_0)}{g'(x_0)}} \qquad \text{falls dieser existiert}
\end{align*}
\[\Rightarrow \lim_{x \to x_0}{\frac{f(x)}{g(x)}} =
\lim_{x \to x_0}{\frac{f'(x)}{g'(x)}}   \qquad \text{falls der Grenzwert existiert }\]
\begin{example}
\begin{gather*}
\lim_{x \to 0}{\frac{sin x}{x}} =
 \lim_{x \to 0}{\frac{(sin x)'}{(x)'}}
= \lim_{x \to 0}{\frac{cos x}{1}} = cos(0) = 1
\end{gather*}
\end{example}
\begin{example}
\begin{align*}
\lim_{x \to 0}{\dfrac{e^x - sin(x) + cos(x)-2}{x^3 . cos(x)}} 
= \lim_{x \to 0 }{\frac{e^x - cos(x) - sin(x)}{3x^2 cos(x)-x^3(-sin(x))}} \dots \dots &= \frac{1}{3}
\end{align*}
\end{example}
\begin{remark}
Diese Methode kann man durch anwenden für $x \to + \infty$ , $x \to - \infty$ . und für $\frac{\infty}{\infty}$ , $\frac{-\infty}{-\infty}$ , $\frac{+\infty}{-\infty}$ , $\frac{-\infty}{+\infty}$\\
\[ \lim_{x \to x_0 \ \pm \infty}{\frac{f(x)}{g(x)}}= 
\lim_{x \to x_0 \ \pm \infty}{\frac{f'(x)}{g'(x)}} \text{falls der Grenzwert existiert. }\]
falls der Grenzwert existiert. 
\end{remark}
\begin{remark}
Man kann durch geeignetes Umformen auch Grenzwerte vom Typ $0.\infty $ berechnen , sowie $0^0 , 1^0 , 1^0$
\end{remark}
 \begin{example}
$\lim\limits_{x \rightarrow 0+} x \ lnx=$\\
1. Mögl. $\lim\limits_{x \rightarrow 0+} \dots = \lim\limits_{x \rightarrow 0+}\frac{x}{\frac{1}{lnx}}$  \\ \\
2.Mögl. $\lim\limits_{x \rightarrow 0+}=\lim\limits_{x \rightarrow 0+}\frac{lnx}{\frac{1}{x}}=\lim\limits_{x \rightarrow 0+}\frac{(lnx)'}{(\frac{1}{x})'}= \lim\limits_{x \rightarrow 0+} \frac{1.x^2}{x.1}(-1)=\lim\limits_{x \rightarrow 0+}(-x)=0$
 \end{example}
\begin{example}
$\lim\limits_{x \rightarrow 0}x^2=\lim\limits_{x \rightarrow 0}e \ lnx^x= \lim\limits_{x \rightarrow 0}e^{xlnx}=e^{\lim\limits_{x \rightarrow 0} xlnx}=e^0=1 \\ \\
\lim\limits_{x \rightarrow 0}x^{\frac{1}{lnx}}= \lim\limits_{x \rightarrow 0} e^{\frac{1}{lnx}lnx}= \lim\limits_{x \rightarrow 0} e^1=e$
\end{example}
\begin{example}
	\begin{align*} 
	\limXin (1+\frac{1}{x})^x &=\limXin e^{\ln(1+\frac{1}{x})^x} \\	&=\limXin e^{x \ln (1+\frac{1}{x})}\\
	&=e \limXin x \ln (1+\frac{1}{x})= \dots = e^1=e \\	
	\text{(Nebenrechnung) NR }\limXin x \ln (1+\frac{1}{x})=& \limXin\frac{\ln (1+\frac{1}{x})}{\frac{1}{x}} = \limXin \frac{\frac{1}{1+\frac{1}{x}}(1+\frac{1}{x})'}{(\frac{1}{x})'}= \limXin \frac{1}{1+\frac{1}{x}}=\frac{1}{1}=1\\
	\text{also auch } \limXin (1+\frac{1}{n})=e \\
	e^x = \sum_{k=0}^{\infty} \frac{x^k}{k!} \overbrace{\longrightarrow}^{x=1} e^1 =e= \sumOin \frac{1^k}{k!}= \sumOin \frac{1}{k!}
	\end{align*}
\end{example}
\section{Integral}
\begin{align*}
f(x)>0 \text{ auf } [a,b]\\
\underline{S_p}= \sumIin f_k (x_k - x_{k-n})und f_k =\min \{ f(x) | \in [x_{k-1},x_K ] \} \\
\overline{S_p}= \sumIin f_k (x_k - x_{k-n})und \overline{f_k} =\max \dots \\
\underbrace{ \lim\limits_{||p|| \rightarrow 0} \underline{S_p}}_{(1) ex.}\underbrace{=}_{(3)} \lim\limits_{||p|| \rightarrow 0 } \overline{S_p}= \underbrace{\int_a^b f(x) \mathrm{d}x}_{Integral von f(x) auf [a,b]}
\end{align*}
%skizze
\begin{example}
$ 	D(x)=\begin{cases} 
		0  \\    
		1 \\  
	\end{cases} auf[0,1]$
	%skizze
	 \t{skizze fehlt!}
\end{example}
\begin{remark}
In jeder reelle Intervall liegen rationale und irrationale Zahlen
	%skizze
\end{remark}
Riemann : $\lim \underline{S_p} \overset{\overset{ex. irrationale Zahl im Int.}{\downarrow}}{=} \lim \sum (x_k- x_{k-1}) 
= \lim 0 = 0 $ 
\[  \neq \lim \overline{S_p}= \lim \sum (x_k-x_{k-1}))> 0\]
Das Riemann - Integral von D(x) ex. nicht 

\subsection{Lebague-Integral }
 skizze fehlt!\\
 %skizze!
\[ \phi(x) \text{ treppenfunktion } \int_a^b \phi(x) = \sumIn c_k (x_k - x_{k-1} ) 
(\phi_k (x)) \]
\[ \text{ Folge von Treppenfunktion auf } [a,b] \backslash M \]
M := $ Nullmenge \textbf{ z.B } \phi \int_a^b D(x)\mathrm{d}x=0$ \\
\[\underbrace{\limKin \phi_k (x)}_{ex.}= f(x)\]
\[\underbrace{\limKin \int_a^b \phi(x)  }_{ex.}=\underbrace{\int_a^b f(x) \mathrm{d}x }_{\text{Lebague-Integral}} \]
