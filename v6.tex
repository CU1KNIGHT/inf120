\section{Vorlesung 6}
\[|x-a| < \delta \]
\begin{equation}
\\|x-a| = \begin{cases}
	x-a , \: x-a \geq 0  \\
	-(x-a),\:  x-a<0 \\
	\end{cases}
 = \begin{cases}
x-a , x \geq \\
a-x , x < a \\
\end{cases}	
 \begin{cases}
  x \leq a : x-a < \delta \Rightarrow x < a+\delta \\
  x<a : a-x < \delta \Rightarrow a-\delta < x	\end{cases} \Rightarrow 
\end{equation}


\begin{equation}
\\ \begin{cases}
a \leq x < a+\delta  \\
a+\delta < x < a \\
\end{cases}
\end{equation}
\subsection{Ergebnis}
$|x-a|<\delta \Leftrightarrow a-\delta < x < a+ \delta$ \\
$\Leftrightarrow x \in (a-\delta , a+d ) $ offenes intervall \\
$|x-a| < \delta$ \\

 %skizze++++++++++++++++++++++++++++++++++++++
 
 x liegt in der $\delta $-Umgebung von a
	
\[	|f(x) -f(a)| < \epsilon  \Leftrightarrow f(x)\text{ liegt in der }\epsilon\text{-umgebung con} f(a)
							 \] \[\Leftrightarrow f(x) \in \left( f(a)-\epsilon , f(a)+ \epsilon \right) \epsilon > o	 \]
 
 %skizze ++++++++++++++++++++++++
 
 \begin{equation}
 	I(\frac{1}{e})= I(e^-1)= 1.k   \quad \text{rell}
 		I(e^{-n})= I(\underbrace{e^{-1} \dots e^{-1}}_n )= I (e^{-1})+\dots+ I(e^{-1})= k.n 
 		\end{equation}
 		 \begin{equation}
 		\frac{n}{m} \in Q : I ( e^{-\frac{n}{n}})= k. \frac{n}{m}\text{, denn}
 		\end{equation}
 		\begin{equation}
 		kn=I(e^{-n}) = I(e^{-\frac{n}{m} . m})= \underbrace{I ( e^{-\frac{n}{m}} \dots  e^{-\frac{n}{m}} )}_{m}+\dots + I (e^{-\frac{n}{m}})= I(e{-\frac{n}{m}})+ \dots + I(e^{-\frac{n}{m}}) 
 		\end{equation}
 		\begin{equation}
 		r \in \mathbb{R}_+ : I(e^{-r})=?
  		\end{equation}

\begin{gather*}
	\limNin \underbrace{q_n}_{\in \mathbb{Q_+}}= r \\
	I(e^{-r})= I(e^{-\limNin (q_n)})
	= I (e^{\limNin (-\frac{q}{n})})
		\overset{\overset{e \: stetig}{\downarrow}}{=} I(\limNin e^{-q_n})	\overset{\overset{I \: stetig}{\downarrow}}{=} \limNin I(e^{-\frac{q}{n}})	
	= \limNin k.q_n
	=\overbrace{k.\underbrace{\limNin q_n}}_{r}^{k.r}\\
	I(\frac{1}{e})= I(e^{-1})= \frac{1}{k} \text{rell} \\
	I(p)=I(e^{\ln p}) = \underbrace{k}_{>0}(- \ln p) =\underbrace{-k}_{<0} \ln p
\end{gather*}

%skizze +++++++++++++++++++++++++++++++

\begin{example}
	D(x)=  
$	\begin{cases}
		1, x \in \mathbb{Q} \quad \quad (rational)\\		
		0, x \in \mathbb{R} \backslash \mathbb{Q} \quad (irrational) 	
	\end{cases}$\\ \\
\text{steteig für welche a?}\\
\text{1. Fall : a rational} \\ 
\text{2. Fall : a irrational }
\\ \\ a rational: a fest 
\\ sei $\varepsilon = \frac{1}{2}$ , beliebig  $\exists \delta > 0 \forall x \in D: |x-a| < \delta \Rightarrow |D(x) -D(a) | < \frac{1}{2}$ 
Sei $\delta$ beliebig, $\delta$ > 0, $x$ irrational , fest \\
$|x-a|<d \Rightarrow |0-1| =|11= 1 < \frac{1}{2}$, widerspruch \\
$\Rightarrow D$ ist nicht stetig, für jede $ a \in \mathbb{R}$  \\
Sei $\delta >0$ , beliebig, x rational , fest $|x-a| < \delta \Rightarrow |\underbrace{D(x)}_1 -\underbrace{D(a)}_0| < \frac{1}{2}= \varepsilon \Rightarrow 1 < \frac{1}{2} $ Widerspruch \\
$\Rightarrow$ D ist nicht stetig für jede a $\in \mathbb{R} \backslash \mathbb{Q}$ 
\end{example}
\begin{theorem}
Sei $ f: [a,b] \rightarrow \mathbb{R}$, stetig 
f besetzt in [a,b] ein globale Maximum und ein golbales Minimum
\end{theorem}
\begin{remark}
Beide (unklar!)veränderung sind wichtig
\end{remark}
\begin{remark}
	[a,k]= {x $\in$ $\mathbb{R}$ | a $\leq$ x $\leq$ b}
%skizze +++++++++++++++++++++
%kizze ++++++++++++++++++
\end{remark}
\begin{theorem}[ZWS]
	 Sei $f: [a,b] \rightarrow \mathbb{R}$ stetig , $\frac{x_m}{x_M}$ eine globale Minimale stelle \\ 
	 eine golbale Maximalestalle\\	
	 Sei  \^{y}
	 $ \in $[ $f(x_m) , f(x_M) :$ Dann ex.  \^{x}$ \in $ [a,b] mit  \^{y}$=f($ \^{x} 
\end{theorem}
\begin{remark}
	Jeder zwischenwert wird als Funktionswert angenommen
\end{remark}
\begin{theorem}[Nullstellen]
	Sei $f: $[a,b] $\rightarrow \mathbb{R}$ stetig, $f(a) . f(b) < 0$
	Dann beliebig f in [a,] eine Nullstell $x_0$ , d.h. $\exists x_0 \in $ [a,b] : $f(x_0)=0$
	%skizze +++++++++++++++++++  
\end{theorem}
\begin{proof}
%skizze ....
$f(a) < 0 , f(x) > 0$  (analog für $f(a) > 0 , f(b) <0$)\\
\\$(\frac{a_1+b_1}{2})=
\begin{cases}
 \; 0 , \frac{a_1+b_1}{2}\text{ ist die gesamte Nullstelle}\\
< 0 \: ,a_2= \frac{a1+a2}{2} , b2=b1\\
> 0 \:, a_2=a_1 , b_2=\frac{a_1+b_1}{2}
\end{cases}$


 \[\text{usw. } \frac{a_2+b_2}{2}  \text{berechnen}\]
\[ f(..)  \begin{cases} =0 \\ <0 \\ >0 \end{cases}\]
 
 
\begin{forest}
	for tree={l+=0.5cm, s sep+=1cm}
	[Betrachte $(a_n)$ 
	[	Stetigmax ]
	[beschränkt  ]
	]
\end{forest}
$\Rightarrow konvergent$
\\
sei $\underbrace{\limNin a_n=:c}_{ex.}$
\\
sei $\underbrace{\limNin a_n=:c}_{ex.}$\\
$a\leqq \dots \leqq b_2 \leqq b_1 \leqq b$
ex. $\limNin b_n=2$

\begin{align*}
\limNin | a_n - b_n| &= \limNin \frac{|a-b|}{2^{n-1}}\\
  				   &= |a-b| \limNin \frac{1}{2^{ n-1}}\\
  				   &= |a-b|.0 \\
  				   &= 0\end{align*}
 \[ \limNin b_n=c \]
\begin{forest}
	for tree={l+=0.5cm, s sep+=1cm}
	[Betrachte $(b_n)$ 
	[	Stetigmax ]
	[beschränkt  ]
	]
\end{forest}
$\Rightarrow konvergent$

Falls keine Nullstelle beim bilden von $a_n,b_n$ gefunden wurden\\ \\
\\


	$\begin{rcases}
	f(c) &=f(\limNin a_n) \overset{\overset{f stetig}{\downarrow}}{=}\limNin f(a_n) \geqq 0 \\
	\quad =\\
	f(c)&=f(\limNin b_n)\overset{\overset{f stetig}{\downarrow}}{=} \limNin f(b_n) \leqq 0 \\
	\end{rcases}$
$f(c)=0$
\end{proof}