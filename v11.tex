\section{Vorlesung 11}
\begin{remark}
Aussage $A(x)$ gilt für alle $x \in \mathbb{R}$ bzw. $x \in \{ a , b\}$\\
Aussage $A(x)$ gilt für alle $x \in \mathbb{R}$ ohne $ M$ bzw. $x \in \{ a , b \}$ ohne $\underbrace{M}_{Nullmenge} $\\
$A(x)$ gilt für fast alle $x \in \mathbb{R}$ bzw. $x \in \{ a , b\} $  
\end{remark}
\begin{definition}[Nullmenge]
Die Menge $M \subseteq \mathbb{R}$ heißt \textbf{Nullmenge}, wenn gilt :\\
für alle $\epsilon > 0$ existiert Intervalle $ ]_1 , ]_2 , \dots \subseteq \mathbb{R}$ sodass :\\
1) $M \subseteq \cup_{k=1}^{\infty} J_k = J_1 \cup J_2 \cup \dots$\\
2) $ \sum_{k=1}^{\infty}|J_k| \leq \epsilon$ wobei $|J_k|$ die Lage des Intervalls $J_k$ bezeichnet.
\end{definition}
\begin{remark}
ab zählbar viele Intervalle endlich viele
ab zählbar unendlich viele
\end{remark}
\begin{example}
Die Menge $ M = \{ x_1 , x_2 , x_3 \}, |M| = 3$ Die Behauptung : M ist eine Nullmenge .  
\end{example}
\begin{proof}
Sei $\epsilon > 0 $ beliebig und fest 
wähle $J_k$ = $[x_k - \frac{\epsilon}{6}, x_k + \frac{\epsilon}{6}]$
Dann gilt : $|\exists_k|=\frac{\epsilon}{3}$  $x_k \in J_k$ und (1) , (2).
\end{proof}  
\begin{remark}
Endliche Mengen sin Nullmenge.
\end{remark}
\begin{remark}
Abzählbar endliche Mengen sind Nullmengen.
\end{remark}
\begin{proof}
Sei $ M = \{ x_1 , x_2 , x_3 , \dots \}$ und sei $\epsilon > 0 $ beliebig und fest. Dann :
fehlende Skizze !!! 
\end{proof}