\section{Vorlesung 11}
\begin{remark}
Aussage $A(x)$ gilt für alle $x \in \mathbb{R}$ bzw. $x \in \{ a , b\}$\\
Aussage $A(x)$ gilt für alle $x \in \mathbb{R}$ ohne $ M$ bzw. $x \in \{ a , b \}$ ohne $\underbrace{M}_{Nullmenge} $\\
$A(x)$ gilt für fast alle $x \in \mathbb{R}$ bzw. $x \in \{ a , b\} $  
\end{remark}
\begin{definition}[Nullmenge]
Die Menge $M \subseteq \mathbb{R}$ heißt \textbf{Nullmenge}, wenn gilt :\\
für alle $\epsilon > 0$ existiert Intervalle $ ]_1 , ]_2 , \dots \subseteq \mathbb{R}$ sodass :\\
1) $$ M \subseteq \bigcup_{k=1}^{\infty} J_k = J_1 \cup J_2 \cup \dots $$\\
2) $ \sum_{k=1}^{\infty}|J_k| \leq \epsilon$ wobei $|J_k|$ die Lage des Intervalls $J_k$ bezeichnet.
\end{definition}
\begin{remark}
ab zählbar viele Intervalle endlich viele
ab zählbar unendlich viele
\end{remark}
\begin{example}[1] 
Die Menge $ M = \{ x_1 , x_2 , x_3 \}, |M| = 3$ Die Behauptung : M ist eine Nullmenge .  
\end{example}
\begin{proof}
Sei $\epsilon > 0 $ beliebig und fest 
wähle $J_k$ = $[x_k - \frac{\epsilon}{6}, x_k + \frac{\epsilon}{6}]$
Dann gilt : $|\exists_k|=\frac{\epsilon}{3}$  $x_k \in J_k$ und (1) , (2).
\end{proof}  
\begin{remark}
Endliche Mengen sin Nullmenge.
\end{remark}
\begin{remark}
Abzählbar endliche Mengen sind Nullmengen.
\end{remark}
\begin{proof}
Sei $ M = \{ x_1 , x_2 , x_3 , \dots \}$ und sei $\epsilon > 0 $ beliebig und fest. Dann :
fehlende Skizze !!! 
\end{proof}
\textbf{Gesamtlänge berechnen}
\begin{align*}
\sum_{k=1}^{\infty}{\frac{\epsilon}{2^k}} &= \epsilon \sum_{k=1}^{\infty}{\frac{1}{2^k}}\\
&= \epsilon (\dfrac{1}{1-\frac{1}{2}}-1)\\
&= \epsilon(2-1)= \epsilon
\end{align*}
Intervalle $J_k = [x_k - \dfrac{\epsilon}{2^{k+1}}, x_k + \dfrac{\epsilon}{2^{k+1}}](k=1,2,\dots)$ erfüllen (1)und (2).
\begin{remark}
Es gilt überabzählbar Mengen , die Nullmenge sind \textbf{Z.B} die \textbf{[Cantor-Menge]}.
\end{remark}
\begin{definition}[Cantor-Menge]
Unter der Cantor-Menge versteht man in der Mathematik eine bestimmte Teilmenge der Menge der reellen Zahlen.\\
\textbf{Schnitte von Intervallen}\\
Die Cantormenge lässt sich mittels folgender Iteration konstruieren:
Man beginnt mit dem abgeschlossenen Intervall $[0,1]$ der reellen Zahlen von 0 bis 1. 
\end{definition}
\[ \int_{a}^{b} f(x) dx = \int_{a}^{c} f(x) dx + \int_{c}^{b} f(x) dx \text{ für alle } x \in \mathbb{R} \]
\begin{definition}
$$ \int_{b}^{c} f(x) dx = - \int_{c}^{b} f(x) dx $$ 
\end{definition}
\subsection{Mittelwertsatz der Integralrechnung}
Sei $f : [a,b]$ stetige Funktion  dann existiert ein $ z \in [a,b]$ mit $\int_{a}^{b} f(x) dx = f(z) \times (a-b)$
fehlende Skizze !!
\subsection{Hauptsatz der Differenzial und Integralrechnung }
Sei $ f : [a , b] \rightarrow \mathbb{R} $ ein stetige Funktion , und sei $ \widetilde{F} : [a , b] \rightarrow \mathbb{R	}$ . $ x \longmapsto \int_{a}^{x} f(t) dt $
Dann ist $\widetilde{F}$ auf $(a,b)$ differenzierbar  und es gilt $\widetilde{F}'$ für alle $x \in (a , b)$
\begin{proof}
Sei $x_0 \in (a,b)$ beliebig und fest 
\begin{align*}
\widetilde{F}'(x_0) &= \lim_{x \to x_0 }{\frac{\widetilde{F}(x)-\widetilde{F}(x_0)}{x - x_0}}\\
&= \lim_{x \to x_0 }{\dfrac{\int_{a}^{x} f(t) dt - \int_{a}^{x_0} f(t) dt}{x - x_0}}\\
&= \lim_{x \to x_0 }{\dfrac{\int_{a}^{x_0} f(t) dt + \int_{x_0}^{a} f(t) dt - \int_{a}^{x_0} f(t) dt}{x - x_0}}\\
&= \lim_{x \to x_0 }{\dfrac{\int_{x_0}^{x} f(t) dt}{x -x_0}}
\end{align*}
laut \textbf{Mittelwertsatz} existiert $z \in (x_0,x)$ mit
\begin{align*} 
 \widetilde{F}'(x_0) &= \lim_{ z \to x_0}{\frac{f(z)( x - x_0 )}{ x - x_0}}\\
&= \lim_{z \to x_0}{f(z)} \underbrace{=}_{\text{f ist stetig}} f(x_0) 
\end{align*}
\end{proof}
\begin{remark}
$\widetilde{F}$ ist eine spezielle Stammfunktion.
\end{remark}
\begin{definition}[Stammfunktion]
eine Funktion heißt \textbf{Stammfunktion} zu $f(x)$ im Intervall $(a , b)$ , wenn gilt :
\[ F'(x) = f(x) \text{ für alle } x \in (a , b) \]
\end{definition}
\begin{remark}
\begin{align*}
F'_1(x) = F'_2(x) = f(x) &\Rightarrow 
F'_2(x)-F'_1(x) = 0\\
&\Rightarrow (F_2(x) - F_1(x))'=0 \text{ für alle x }
\end{align*}
\end{remark}