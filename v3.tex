\section{Vorlesung 3 \href{https://tu-dresden.de/mn/math/algebra/das-institut/beschaeftigte/antje-noack/ressourcen/dateien/v120-1/MathMethInf03.pdf?lang=en}{(12.04.2019)}}

\begin{example}

\end{example}
\begin{equation}
    \lim\limits_{n \rightarrow \infty}{\frac{11+n}{9-n}}\quad ? \\
    \\\quad x_n = \frac{11+n}{9-n}=\frac{n}{n} \frac{\frac{11}{n}+1}{\frac{9}{n}-1}
\end{equation}

\begin{equation}
    \lim\limits_{n \rightarrow \infty}{\bigg(\frac{11}{n}+1\bigg)}=1
\end{equation}

\begin{equation}
    \lim\limits_{n \rightarrow \infty}{\bigg(\frac{9}{n}-1\bigg)}=-1
\end{equation}

\begin{equation}
    \lim\limits_{n \rightarrow \infty}{(x_n)}= \frac{1}{-1}=-1
\end{equation}

\begin{lemma}[Quetschlemma]
    Seien $(x_n),(y_n)$ Folgen mit $\lim\limits_{n \rightarrow \infty}{(x_n)}= \lim\limits_{n \rightarrow \infty}{(y_n)}= a$ und es gelte
    $x_n \leq z_n \leq y_n$ für "fast alle"  $n \in \mathbb{N}$\\

    Dann gilt für die Folge $(Z_n) \lim\limits_{n \rightarrow \infty}{(z_n)}=a$
\end{lemma}

\begin{example}
    Ist die Folge $(-1)^n\frac{1}{n})$ konvergent ?\\

    \[ - \frac{1}{n} \leq(-1)^n(\frac{1}{n}) \leq 1 \frac{1}{n}\]

    \[ \lim\limits_{n \rightarrow \infty}{- \big(\frac{1}{n} \big)}= -1 \]
    \[ \lim\limits_{n \rightarrow \infty}{ \big(\frac{1}{n} \big)}= 0 \Rightarrow \lim\limits_{n \rightarrow \infty}{(-1)^n \frac{1}{n}}= 0
    \]
\end{example}

\newpage

\begin{example}
    \begin{equation}
        \begin{aligned}
            x_n \leq  = \frac{a^n}{n!} = \frac{a}{n} \times \frac{a^{a-1}}{n-1!} %
        \end{aligned}
    \end{equation}\\

    denn $ x_n = 0 \leq \frac{a_n}{n!} \leq y_n$
    , gesucht! $\underbrace{y_n}_{\lim\limits_{n \rightarrow \infty}{y_n}=0}$  für hinreichend großes n.

    \begin{equation}     
    \begin{aligned}   
            \frac{a^n}{n!} = \frac{a}{n} \times \frac{a^{n-1}}{(n-1)!} \\ \leq
            \frac{1}{2} \times
            \frac{a^{n-1}}{(n-1)!} \\ =
            \frac{1}{2} \times
            \frac{a}{(n-1)} \times
            \frac{a^{n-2}}{(n-2)!} \\ \leq
            \frac{1}{2} \times
            \frac{1}{2} \times
            \frac{a^{n-2}}{(n-2)!} \\ \leq
            \frac{1}{2} \times
            \frac{1}{2} \times
            \frac{1}{2} \times
            \frac{a^{n-3}}{(n-3)!}\\
            %
            y_n = (\frac{1}{2})^{n-k} \times \frac{a^k}{k!} \quad \text{k ist fest}
        \end{aligned}
    \end{equation}\\

    {Es gilt} $\frac{a^n}{n!} \leq y_n$  für hinreichend großes n und
    $\lim\limits_{n \rightarrow \infty}{(y_n)}$ \\

    \begin{equation}
        \begin{aligned}
            &=
            \lim\limits_{n \rightarrow \infty}{(\frac{1}{2})^{n-k}} \times
            \underbrace{\frac{a^k}{k!}}_{Konst} \\
            &=
            \lim\limits_{n \rightarrow \infty}{(\frac{1}{2})^{n}} \times
            \underbrace{\lim\limits_{n \rightarrow \infty}{(\frac{1}{2})^{-k}}}_{\in \mathbb{R}} \times
            \underbrace{\lim\limits_{n \rightarrow \infty}{(\frac{a^k}{k!})}}_{\in \mathbb{R}} \\
            &= 0 . (\frac{1}{2})^{-k} \times \frac{a^k}{k!}=0
            \\
        \end{aligned}
    \end{equation}
\end{example}



\newpage

\section{Grenzwerte rekursive definierte Folgen:}

man kann oft durch lösen "Fixpunktgleichung" berechnen.\\
$x_0 \quad , x_{n+1}= ln(x_n)$ \\
Folge, Falls $(x_n)$ hinreichend ist, was gelten 
\[ \limNin x_n= \limNin x_{n-1} = \limNin x_{n-2}=\dots = 4 \]
\begin{example}
\[(x_n) \quad x_0 = \frac{7}{5} \quad,\quad x_{n+1}= \frac{1}{3}(x_n^2+2)  \]

Ü $(x_n)$ ist monoton fallend , beschränkt , konvergent . 

\[\lim\limits_{n \rightarrow \infty}{x_n}=a \quad,\quad 
\lim\limits_{n \rightarrow \infty}{x_{n+1}}=a \]

\begin{equation*}
\begin{aligned}
\lim\limits_{n \rightarrow \infty}{x_{n+1}} 
= \linebreak  
\lim\limits_{n \rightarrow \infty}{\frac{1}{3}(x_n^2 + 2)=
\frac{1}{3}} \lim\limits_{n \rightarrow \infty}{(x_n^2 + 2)}
=
\frac{1}{3} (\lim\limits_{n \rightarrow \infty}{(x_n))^2 + 2)}
\end{aligned}
\end{equation*}
\end{example}

\subsubsection{Fixpunktgleichung }
 $a = \frac{1}{3}(a^2 + 2) $  , gesucht = a
 
\[ 3a = a^2 +2 \Leftrightarrow a^2 -3a+2 = 0 \] \\
\[ \Leftrightarrow a_{1/2} = \frac{3}{2} \pm \sqrt{\frac{9}{4}-\frac{8}{4}}= \frac{3}{2} \pm \frac{1}{2}\]
Lösung:  $a_1 = 2$ (keine Lösung),  $a_2 =1 $

\begin{example}{$(x_n)$ mit $(x_0) = c \in \mathbb{R} , c  $ fest $x_{n+1}= \frac{1}{2}(x_n+\frac{c}{x_n})$ }\\
(1) $(x_n)$ beschränkt \checkmark\\
(2) $(x_n)$ Monoton \checkmark\\
Also $(x_n)$ konvergent \\
Sei $\lim\limits_{n \rightarrow \infty}{x_n}= a $. 
Dann $\underbrace{\lim\limits_{n \rightarrow \infty}{x_{n-1}}= }_{a}$ $\lim\limits_{n \rightarrow \infty}{\frac{1}{2}}(x_n) + \frac{c}{x_n} = \frac{1}{2}(a + \frac{a}{c})= a \\
 \Leftrightarrow 2a = a + \frac{c}{a} \Leftrightarrow a = \frac{c}{a } \Leftrightarrow a^2 = c \Leftrightarrow a = \sqrt{c}$
\end{example}

\begin{remark}
Der Nachweis der konvergent der rekursiv definierte Folge darf nicht weggelassen werden, denn Z.B $x_0=2$ , $x_{n+1}=x_n^2$ \quad \quad \quad 2 , 4 ,16 ,256 , $\dots $ divergent gegen + $\infty$  \\

Annahme: $\lim\limits_{n \rightarrow \infty}{x_n}= a $ 
$\underbrace{\quad \lim\limits_{n \rightarrow \infty}{x_{n+1}}}_{a}$ = 
$\underbrace{\lim\limits_{n \rightarrow \infty}{x_n^2}}_{a^2} \Rightarrow a \in \{ 0,1 \}$
\end{remark}  

%new 
 
\newpage
\section{Reihen :}
\begin{definition}[Unendliche Reihen]
Sei $(a_n)$ eine reellefolge (komplexwertig) Folge\\
$$\sum_{k = 0}^{n} {a_k} = a_a , a_1, \dots , a_n , $$
 n-k heißt Partialsumme.
$(S_n)$ heißt unendliche Reihe.

schriebweise : $(S_n)^\infty =$ bsw 
$(S_n)$ $$ \bigg( \sum_{l=0}^{n} {a_l} \bigg)$$ bzw
 $$ \bigg( \sum_{l=0}^{\infty} {a_l} \bigg)$$  
\end{definition}

\begin{remark}
Reihen sind spezielle Folgen , alle konvergent oder divergent. 
\end{remark}

\begin{definition}[wert der Reihe]
Für eine konvergente Reihen wird der Grenzwert auch wert der Reihe genannt.\\

\begin{schreibweise}
 :  $\lim\limits_{n \rightarrow \infty}{S_n}= $
$$\lim\limits_{n \rightarrow \infty}{ \sum_{k=0}^{n} {a_k} }  $$ 
bzw 
$$ \sum_{k=0}^{\infty} {a_k}  $$
\end{schreibweise}

\end{definition}

\begin{example}{Teleskopreihe} 
\begin{gather*}
\sum_{k=1}^\infty(\frac{1}{k}-\frac{1}{k+1}) \text{in Grenzwert der Reihe ist}\\ 
\sum_{k=1}^\infty(\frac{1}{k}-\frac{1}{k-1})=1 \\
\lim_{n \to \infty}{S_n} = \lim_{n \to \infty}{\sum_{k=1}^n(\frac{1}{k}- \frac{1}{k-1}) }\\
= \limNin{(\frac{-1}{2})+\frac{1}{2}(\frac{1}{3} + \frac{1}{3})(-\frac{1}{4})+) \dots +(\frac{1}{n})-\frac{1}{n+1}}\\
= \limNin{(1- \frac{1}{n+1})}= 1-0 =1 
\end{gather*}
\end{example}

\begin{example}
geometrische Reihe $ \sum_{k=0}^{\infty} {q^k} $
ist für $$ {|q|} < 1 $$ konvergent . wert der Reihe für $ {|q|} < 1 $  $\sum_{k=0}^{\infty} {q^k}= \frac{1}{1-q} $ für 
$ {|q|} < 1 $  konvergent , werte der Reihe für 
$$ {|q|} <1 : \sum_{k=0}^n{q^k}= \dots $$

\begin{gather*}
S_n = q^0 + q^1 + \dots + q^n | *q \\
-q S_n = q^1 + q^2 + \dots + q^{n+1} \\
(1-q)S_n=q^0 - q^{n+1} \\
S_n = \frac{1-q^{n+1}}{1-q} = \frac{1}{1-q}(1-q)^{n+1}\\
\Rightarrow lim_{n \to \infty}{S_n} = \frac{1}{1-q} \times 
\lim_{n \to \infty}{((1-q)^{n+1})}\\
=\frac{1}{1-q}(1- \lim_{n \to \infty}{q^{n+1}})=\frac{1}{1-q}
\end{gather*}

\subsection{Rechnenregeln für Reihen}
konvergent Reihe kann man addieren oder subtrahieren mit einem Skalar multiplizieren
wie endliche Summen.
\underline{ABER:}
 das gilt im Allgemein nicht für das Multiplizieren  


\end{example}