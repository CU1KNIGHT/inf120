% Mathematische Methoden für Informatiker: Institut %Algebra Algebra
% TU Dresden SS19
% Prof Dr Ulrike Baumann

% Setup document
\documentclass[a4paper,12pt,leqno]{report}
\usepackage{inputenc,fontenc}
\usepackage{longtable,tabularx,tabulary,array,booktabs,multicol}
\usepackage{graphicx,caption,color,xcolor}
\usepackage{verbatim}
\usepackage{url}
\usepackage{amsmath,amsthm,amssymb,amsfonts}
\usepackage[a4paper,margin=3cm]{geometry}
\usepackage{pgfplots}
\usepackage[english,german]{babel}
\usepackage{hyperref}
\usepackage{pgfplots}
\usepackage{pgfplots}
\usepackage{MnSymbol} %sch\"{o}nere Mathe-fonts
\usepackage{eucal}    %sch\"{o}nere Skript-Buchstaben
\usepackage{dsfont}   %sch\"{o}nere "Doppelstrich"-Fonts

% Fontauswahl
\pgfplotsset{compat=1.12}
\usepgfplotslibrary{fillbetween}
\usetikzlibrary{patterns}


% Define oftenly used commands

\newcommand{\nameE}{Mohamed}
\newcommand{\vornameE}{Abdelshafi}
\newcommand{\emailE}{m.abdelshafi@mail.de}
\newcommand{\nameS}{kiki}
\newcommand{\vornameS}{Mahmoud}
\newcommand{\emailS}{mahmoud.kiki@tu-dresden.de}
\newcommand{\mathscr}{\mathcal}
\newcommand{\institut}{Institut Ihres Betreuers / Ihrer Betreuerin}
\newcommand{\thema}{Mathematische Methoden für Informatiker}
\newcommand{\datum}{\today}

%created theorem
\swapnumbers

\theoremstyle{plain} %Text ist Kursiv
\newtheorem{theorem}{Satz}[chapter]
\newtheorem{lemma}[theorem]{Lemma}
\newtheorem{proposition}[theorem]{Proposition}
\newtheorem{corollary}[theorem]{Korollar}
\theoremstyle{definition} %Text ist \"upright"
\newtheorem{remark}[theorem]{Bemerkung}
\newtheorem{definition}[theorem]{Definition}
\newtheorem{example}[theorem]{Beispiel}



% Vertical lines in matrices
\makeatletter
\renewcommand*\env@matrix[1][*\c@MaxMatrixCols c]{%
  \hskip -\arraycolsep
  \let\@ifnextchar\new@ifnextchar
  \array{#1}}
\makeatother

% Define new column types for direct math mode
\newcolumntype{L}{>{$}l<{$}}
\newcolumntype{R}{>{$}r<{$}}
\newcolumntype{C}{>{$}c<{$}}

\hypersetup{
	colorlinks=true,
	linkcolor=blue,
	filecolor=magenta,      
	urlcolor=cyan,
}



{\author{ \vornameE \nameE \vornameS \nameS}}
{\title{Mathematische Methoden für Informatiker}}



\begin{document}
	\selectlanguage{german}
	%\selectlanguage{english} %Entferne "%", wenn Sprache Englisch ist
	
	
	
	%% Titelseite
	
	\thispagestyle{empty}
	
	\includegraphics[scale=1]{../../../Downloads/TU_Dresden.eps} 

	\begin{center}
		{\Large
			 Fakult\"{a}t Informatik
		}
		
		\vfil
		
		{\bfseries\Huge\thema}
		
		\vfil
		
		{\LARGE
			Mitschrift zur Vorlesung Sommer Semester 2019 \\[\bigskipamount]

			\bfseries{\itshape Bachelor of Science  \textup{(}B.Sc.\textup{)}}\\[\bigskipamount]
		}% Ende Large
		
		\vfil\vfil\vfil
		Dozent: Prof. Dr. Ulrike Baumann \\
		vorgelegt von\\ 
		\item ... \\
		\item \textsc{\vornameE\ \nameE } \\ \texttt{\emailE} \\  \item 
		\textsc{\vornameS\ \nameS \qquad } \\ \texttt{\emailS}  \\ 
		\item ... \\
		Tag der Einreichung: \datum\\[\bigskipamount]

	\end{center}


\cleardoublepage



\tableofcontents

\thispagestyle{empty}

%\makeatletter
%\begin{titlepage}
	
%	The title is \@title
%	It was written by \@author\space on \@date
	
%\end{titlepage}
\setcounter{page}{0}
\chapter*{Einleitung}

\chapter{Folge und Reihen:}
\section{Folgen und Reihen}

\begin{definition}
 Ein folge ist eine Abbildung
 
\[ f: \mathbb{N} \rightarrow \underbrace{\mathbf{M}}_{Menge} : \mathrm{n} \mapsto \underbrace{X_n}_{folgenglied} \]

\end{definition}
\begin{remark}
	
	\begin{align}	\mathbb{M} &= \mathbb{R} \quad reelewert Folge  \notag \\
	M&= \mathbb{C} \quad	 Komplexewetig folge  \notag \\
	\mathbb{M}&= \mathbb{R}^n \quad vertical folge \notag
	\end{align}	
	

	
	
\end{remark}
\begin{description}
	
	\item[Bezeichnung]
	
	 \quad $(X_n)$ \space \text{mit} \space $ \left( X_n =  \right) \frac{n}{n+1} $
	\\ \\ Aufzählung der folglieder: 0 , $\frac{1}{2}$ ,$\frac{2}{3}$ , $\frac{3}{4}$ , \dots
	\begin{remark}
		zuweten wird $\mathbb{N}$ durch $\mathbb{N}$ {0,1 \dots} erstellt. 
		
		
	\end{remark}
	 \begin{tikzpicture}
	 	\begin{axis}[
	 	xlabel= $x$,
	 	ylabel= {$y$}]
	 	\addplot { x/(x+1)};
	 	\end{axis}
	 \end{tikzpicture}
\end{description} 
\begin{example}
\[\]
	\begin{enumerate}
		
		\item Konstante Folge $(X_n)$ mit \quad $X_n = a \in \mathbb{M },a \dots$ \\
		\[ X_n = a \in \mathbb{M} \]
		\item Harmonische Folge $(X_n)$  mit $X_n$ =  $\frac{1}{n+1}$ \quad$ n \geq 1$
		\item Geometrische folge $(X_n)$ mit $X_n = q^n \:. \: q \in \mathbb{R}, \dots $
		\item Fibonaccifolge $(X_n)$ mit 
		\[ X_n =\frac{1}{\sqrt{5}} \Big(  \big( \frac{1+ \sqrt{5}}{2} \big)^n - \big( \frac{1- \sqrt{5}}{2}\big)^n   \Big)    \]
		
		\item Fibonacci folgen $(X_n)$ 
		\begin{align}
			X_0 &=0 \notag \\X_1 &= 1 \notag \\
			X_n+1 &= X_n+X_n-1 (n>0)  \notag
		\end{align}
		
		\item conway  folge
		\[ 1, 11 ,21 , 1211, 111217, 312211 \dots \]
		
		\item folge aller Primzahlen:- \[ 2, 3 ,5 ,7 ,11, 13 , \dots \]  
		
	\end{enumerate}
\end{example}

\section{Rechnen mit Folgen }
\begin{align*}
			M  = \mathbb{R} \quad oder \quad M = \mathbb{C} \\
			(X_n)+(y_n) := (X_n+y_n)\\
			K(X_n):=(KX_n)\in \mathbb{R} \quad  oder \quad \in  \mathbb{C}  		 		 		 		 		 		 		 		 
		\end{align*}
\begin{remark}
\[\]
Die Folge bildet ein Vektorraum. 
\end{remark}


\begin{definition}


\begin{enumerate}

\mbox{}\item Eine reellwertige Funktion ist in der Mathematik eine Funktion, deren Funktionswerte reelle Zahlen sind.

\item Eine reellwertige heißt beschränkt wenn gilt 

 \[	\exists r \in \mathbb{R}_+ , \forall r \in \mathbb{N}: \underbrace{|x|}_{\text{ Betrag einer reellen oder komplexer Zahl}} \leqq r   \]

\end{enumerate}
\end{definition}

\begin{example}
\[(X_n)\quad mit \quad X_n = (-1)^n \times \frac{1}{n} \]
\[-1 ,\quad \frac{1}{2}, \quad \frac{-1}{3}, \quad \frac{1}{4} ,\quad \frac{-1}{5},\dots \]
\end{example}

\begin{remark}

$(X_n)$ ist beschränkt mit $r = 1$ denn $|(-1)^n \frac{1}{n}|=|\frac{1}{n}| \leqq 1  $  

\end{remark}

\begin{tikzpicture}
 \begin{axis}[
  ymin = -2,
  ymax = 2,
  xmin = -3,
  xmax = 3,
  axis x line=center,
  axis y line=center]
    \addplot[samples at={-3,...,3},only marks,mark size=1]{(-1)^x*(1/x)};
  \end{axis}
\end{tikzpicture}
\begin{example}
	\[  (X_n) mit X_n = (-1)^n \frac{1}{n}+1 \quad \text{bechränkt r = 3/2}\]
	
	\[-1/3 \quad \leq X_n \leq 3/2 \quad \forall n \in \mathbb{N} \]
	%
	\begin{tikzpicture}
	\begin{axis}[
	ymin = -2,
	ymax = 2,
	xmin = -4,
	xmax = 4,
	axis x line=center,
	axis y line=center]
	\addplot[samples at={-3,...,3},only marks,mark size=1][ domain=(-1/3):(3/2)] {((-1)^x)*(1/x)+1};
	\end{axis}
	\end{tikzpicture}
\end{example}
\vfil
\vfil
\begin{example}{Standard:}\\

{Die folge}
$ \bigg(\big(1 + \frac{1}{n} \big)^n \bigg)^\infty$	
	 {ist beschränkt durch 3}\\
	 
Zu Zeigen:
	
\[ -3 \leq X_n \leq 3 \quad \text{für alle} \quad n \in \mathbb{N} \]
\[ { (a+b)^n = \sum_{k=0}^{n} \binom{n}{k} a^k . b^{n .k} = \sum_{k=0}^{n} \binom{n}{k} a^{n.k} b^k }  \]  
	\[  \binom{n}{k} = \frac{n!}{k!(n-k!)} =  \frac{n(n-1) -(n-k-1))}{k!} \]
	\[  \sum_{K=0}^{n} \frac{1}{k} = 1+ \frac{1}{2} + \frac{1}{2.3} + \frac{1}{2.3.4} + \dots \]
\end{example}


\section{geometrische Summen Formel (Tafelwerk)}
\begin{definition}
	
	Die Folge $(X_n)$ heißt monoton $\Big\{ wachsend fallend \Big\}$
	\[ wenn \quad gilt: \forall n \in \mathbb{N}:
	\left\{
	\begin{array}{ll}
	X_n  & \leq X_n +1 \\
	X_n  & \geq X_n+1
	\end{array}
	\right. \]
	\text{man spricht von Streng monotonie}
	$ wenn \leq durch > und \geq durch < \dots  $
\end{definition}
\begin{remark}
\[ X_n \leq X_{n+1} \ \Leftrightarrow  X_n - X_{n+1} \geq 0 \quad \Leftrightarrow  \frac{x_n}{X_{n+1}} \leq 1 \]
\end{remark}

\begin{example}

	
	\[	(X_n) mit X_0 := 1 , X_{n+1} := \sqrt{X_n +6} \]
	\paragraph{Standard Bsp:}
	$ \big( \big(1+ \frac{1}{n} \big)^n \big) $ ist streng monoton wachsend
\end{example}
\begin{remark}
	
	\[
		\begin{tabular}{|c| c c |} 
			\hline 
			monoton & ja & nein  \\
			\hline 
			\rule{0pt}{3ex}    
			Beschränktkeit & $\frac{1}{n}$ & $(-1)^n$ \\ 
			nein & (n) & $(n)^n$ \\ 
			\hline
		\end{tabular}
	\]
	
\end{remark}
\begin{definition}
$(X_n)$ heißt $\boldsymbol{Konvergenz}$ wenn $(X_n)$ ein grenzwert hat.
$(X_n)$ heißt $\boldsymbol{Divergrnz}$ wenn sie keinen grenzwert hat.
\end{definition}
\begin{definition}(grenzwert)
	a $\in$ $\mathbb{R}$ heißt grenzwert von $(X_n)$, wenn gilt:
	\[ \underbrace{\forall > 0 }_{beliebes \; klein} \quad \underbrace{\exists \mathbb{N} \in \mathbb{N}}_{beliebes \;  klein \; \underbrace{\Rightarrow |X_n -a|< \varepsilon}_{a- \varepsilon \leq X_n \leq a+\varepsilon } } , \forall n \in \mathbb{N} : m \geq \mathbb{N}  \]
		\[ \text{Sei  } \varepsilon > 0 ; \varepsilon \text{  fest} \] 
		\[ \text{alle folglieder$ X_n$ mit n } \geq \mathbb{N} \curvearrowright \]
		\begin{tikzpicture}[scale=1.1],
		
		\begin{axis}[
		height=8cm,
		width=10cm,  
		ymax=2,
		ymin=-2,
		xmin=1,
		xmax=5.5,
		axis y line=left,
		axis x line=bottom,
		yticklabels={,, $ a - \varepsilon$, a , $a + \varepsilon$ },xtick={1,...,10},
	    xticklabels={,,N,n},
		]
		%\addplot {|x-a|<0};
		\addplot[dashed, name path =A] coordinates {(0,1) (5.5,1)};
		
		
    	\addplot[thick, samples=50, smooth,domain=0:6,magenta, name path=V] coordinates {(3,-2)(3,3)};
		\addplot[dashed, name path =B] coordinates {(0,-1) (5.5,-1)};
		\addplot[gray, pattern=north west lines] fill between[of=A and B, soft clip={domain=3:6}];
		\addplot[samples at={0,...,5},only marks,mark size=1] {sin(x)};
	
		\end{axis} 
		\end{tikzpicture}%
		

\end{definition}

\begin{text}
ist die folge beschränkt , monoton ?\\

$(X_n)$ konvergierend : $\iff \exists a \in\mathbb{R} \quad \forall \epsilon > 0 \quad \exists n \quad \in N \quad \forall n \in N \quad \\
 n \geq N \Rightarrow |X_n - a |< \epsilon $  
\end{text}

\begin{theorem}

$(X_n)$ konvergierend : $\Rightarrow$ Der Grenzwert ist eindeutig beschränkt.
 
\end{theorem}

\begin{proof}
Sei a eine Grenzwert von $(X_n)$ , b eine Grenzwert von $(X_n)$ \\
d.h sei $\epsilon > 0$,$\epsilon$ beliebig , $\epsilon$ fest \\

\begin{equation}
\exists  N_a \quad \forall n \geq N_a : |X_n-a|< \epsilon
\end{equation}

\begin{equation}
\exists  N_b \quad \forall n \geq N_b : |X_n-b|< \epsilon
\end{equation}

Sei max ${N_a,N_b}=N$
dann gilt : \\
\begin{equation}
n \geq N \Rightarrow |X_n - a| < \epsilon 
\end{equation}
und \begin{equation}
|X_n -b| < \epsilon \Rightarrow |X_n -a|+|X_n - b|< 2\epsilon
\end{equation}\\

Annahme :- a $\neq$ b , d.h $|a-b|\neq 0 $ 
\[|a-b|=|a+0-b| 
=|(a-X_n)+(X_n-b)| \leq |X_n - a|+|X_n-b|< 2 \epsilon \]
also $|a - b|< 2 \epsilon$


\begin{example}
\[\epsilon = \frac{|a-b|}\epsilon
\quad \text{dann gilt}\ :|a-b|<2 \frac{|a-b|}{3}\]\\

\[ \Rightarrow 1 < \frac{2}{3} \quad falls \quad Aussage, Widerspruch \quad also \quad ist \quad die \quad Annahme \quad falsch \quad also \quad gilt \quad a=b\] 

\end{example}
\end{proof}

\begin{example}

$X_n$ mit $X_n = \frac{1}{n}$ (harmonische Folge)  

\end{example}

\begin{proof}
Sei $\epsilon > 0 , \epsilon belibig , \epsilon fest$
gesucht : N mit $n \geq$ N 
   
\begin{gather}
\Rightarrow |X_n-a|= |\frac{1}{n} =0|=\frac{1}{n}<\epsilon
\end{gather}

wähle N:= $\lceil \frac{1}{\epsilon} \rceil +1$  

\end{proof}

\begin{example}
$\epsilon = \frac{1}{100}$ , gesucht N mit $n \geq N$
$\Rightarrow \frac{1}{n} < \frac{1}{100}$ wähle $N=101$\\


Schreibweise: $X_n$ hat den Grenzwert a Limes 
$\lim\limits_{n \rightarrow \infty}{x_n}=a$ 
$X_n$ geht gegen a für n gegen Unendlich.
\end{example}

\begin{definition}
$X_n$ heißt Nullfolge ,wenn $\lim\limits{X_n}=0$ gilt. 
\end{definition}

\begin{remark}
 
Es ist leichter, die konvergente einer Folge zu beweisen, als den Grenzwert auszurechnen.   

\end{remark}

\begin{example}
$X_n = \frac{1}{3} + \big(\frac{11-n}{9-n}\big)^9$\\


Behauptung: $\lim\limits_{n \rightarrow \infty}{x_n}=\frac{-2}{3}$

\begin{lemma}
\begin{gather}
\lim\limits_{n \rightarrow \infty}{x_n+y_n}= 
(\lim\limits_{n \rightarrow \infty}{x_n}) + 
(\lim\limits_{n \rightarrow \infty}{y_n})    
\end{gather}
\end{lemma}

\begin{gather}
=\lim\limits_{n \rightarrow \infty}{\bigg(\big(\frac{1}{3}\big)+\bigg(\frac{11-n}{9+n}\bigg)^9\bigg)} 
= \lim\limits_{n \rightarrow \infty}{\frac{1}{3}+
\lim\limits_{n \rightarrow \infty}{\bigg(\frac{11-n}{9+n}\bigg)^9}}
\end{gather}

\begin{gather}
= \frac{1}{3} + \bigg(\lim\limits_{n \rightarrow \infty}{\frac{11-n}{9+n}}\bigg)^9
\end{gather}


\begin{gather}
= \frac{1}{3} + \lim\limits_{n \rightarrow \infty}{\Bigg(\frac{n(\frac{1}{n}-1)}{n(\frac{9}{n}+1)}\Bigg)^9}
\end{gather}
 
\begin{gather}
= \frac{1}{3}+\Bigg(\frac{\lim\limits_{n \rightarrow \infty}{(\frac{11}{n})}}{\lim\limits_{n \rightarrow \infty}{(\frac{9}{n}+1})}\Bigg)^9
\end{gather}

\begin{gather}
= \frac{1}{3} + \Bigg( 
\frac{\lim\limits_{n \rightarrow \infty}{\frac{11}{n}} - \lim\limits_{n \rightarrow \infty}{1}}{\lim\limits_{n \rightarrow \infty}{\frac{9}{n}+\lim\limits_{n \rightarrow \infty}{1} } }   \Bigg)^9
\end{gather}


\begin{gather}
=\Bigg( 
\frac{\lim\limits_{n \rightarrow \infty}{11} \times \lim\limits_{n \rightarrow \infty}{(\frac{1}{n}-1)}}{\lim\limits_{n \rightarrow \infty}{9 \times \lim\limits_{n \rightarrow \infty}{(\frac{1}{n}+1)} } }   \Bigg)^9
\end{gather}

\begin{gather}
\frac{1}{3}+(-1)^9 = \frac{1}{3}-1 = \frac{-2}{3}
\end{gather}
\end{example}

\begin{definition}
Eine Folge $(X_n)$ hat den unendliche Grenzwert $\infty$, wenn gilt : \\
\[\forall r \in \mathbb{R} \quad \exists N \in N \quad \forall n \geq N : X_n > r \]

Schreibweise : $\lim\limits_{n \rightarrow \infty}{X_n}= \infty$  
\end{definition}

\begin{remark}
$\infty$ ist keine Grenzwerte und keine reelle Zahl.    
\end{remark}

\begin{remark}
Grenzwertsätze gelten nicht für uneigentliche Grenzwerte.     
\end{remark}

\begin{remark}
gilt $\lim\limits_{n \rightarrow \infty}{X_n}= \infty$ dann schreibt man $\lim\limits_{n \rightarrow \infty}{X_n}= -\infty$
\end{remark}

\begin{example}
$X_n$ mit $X_n = q^n$ , $q \in \mathbb{R}$ , $q$ fest.\\

$ \lim\limits_{n \rightarrow \infty}{q^n} = \begin{cases}
0 ,\quad |q|<1 \\ 
1 ,\quad |q|=1 \\
\infty ,\quad\quad q > 1  \\ 
ex. nicht ,\quad q\leq -1 
\end{cases}$
\end{example}


\section{Konvergenzkriterien}
(zum Beweis der Existenz eine Grenzwert, nicht zum berechnen von Grenzwert) \\


(1) $X_n$ konvergent $\Rightarrow$ $(X_n)$ beschränkt. \\

wenn $(X_n)$ nicht beschränkt $\Rightarrow$ $(X_n)$ nicht konvergent.\\


(2) Monotonie Kriterium:
wenn $(X_n)$ beschränkt ist können wir fragen ob $(X_n)$    konvergent.\\


$(X_n)$ beschränkt von Monotonie $\Rightarrow$ $(X_n)$ konvergent.  


\end{document}




