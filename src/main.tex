\documentclass[a4paper,12pt,leqno]{report}

\usepackage{inputenc,fontenc}
\usepackage{longtable,tabularx,tabulary,array,booktabs,multicol}
\usepackage{graphicx,caption,color,xcolor}
\usepackage{verbatim}
\usepackage{url}
\usepackage{amsmath,amsthm,amssymb,amsfonts}
\usepackage[a4paper,margin=3cm]{geometry}
\usepackage{pgfplots}
%\usetikzlibrary{calc}

\usepackage{hyperref}
\hypersetup{
colorlinks=true,
linkcolor=blue,
filecolor=magenta,
urlcolor=cyan,
}


% Fontauswahl
\usepackage{pgfplots}
\pgfplotsset{compat=1.12}

\usepgfplotslibrary{fillbetween}
\usetikzlibrary{patterns}

\usepackage{MnSymbol} %sch\"{o}nere Mathe-fonts
\usepackage{eucal}    %sch\"{o}nere Skript-Buchstaben
\usepackage{dsfont}   %sch\"{o}nere "Doppelstrich"-Fonts
\newcommand{\mathscr}{\mathcal}
%\usepackage{picinpar} %Einbinden von Bildern
%\usepackage{picins}   %Einbinden von Bildern
\usepackage[english,german]{babel}



%%
%% Theorem environments etc.
%% See also: ftp://ftp.ams.org/ams/doc/amscls/amsthdoc.pdf
%%

\swapnumbers

\theoremstyle{plain} %Text ist Kursiv
\newtheorem{theorem}{Satz}[chapter]
\newtheorem{lemma}[theorem]{Lemma}
\newtheorem{proposition}[theorem]{Proposition}
\newtheorem{corollary}[theorem]{Korollar}

\theoremstyle{definition} %Text ist \"upright"
\newtheorem{remark}[theorem]{Bemerkung}
\newtheorem{definition}[theorem]{Definition}
\newtheorem{example}[theorem]{Beispiel}




%%
%% Hier bitte eigene packages einbinden
%%





%%
%% Hier bitte eigene Befehle "\newcommand" einbinden
%%
%\newcommand{cmd}{def}



%%
%% Pflichtangaben
%%

\newcommand{\nameE}{Mohamed}
\newcommand{\vornameE}{Abdelshafi}
\newcommand{\emailE}{m.abdelshafi@mail.de}
\newcommand{\nameS}{kiki}
\newcommand{\vornameS}{Mahmoud}
\newcommand{\emailS}{mahmoud.kiki@tu-dresden.de}



{\author{ \vornameE \nameE \vornameS \nameS}}
{\title{Mathematische Methoden für Informatiker}}







\newcommand{\institut}{Institut Ihres Betreuers / Ihrer Betreuerin}

\newcommand{\thema}{Mathematische Methoden für Informatiker}

\newcommand{\datum}{\today}%Format tt.\ mm.\ jjjj




\begin{document}
    \selectlanguage{german}
    %\selectlanguage{english} %Entferne "%", wenn Sprache Englisch ist



    %% Titelseite

    \thispagestyle{empty}

    \begin{center}
    {\Large
    Technische Universit\"{a}t Dresden\  \ \textbullet\ \ Fakult\"{a}t Informatik
    }

        \vfil

        {\bfseries\Huge\thema}

        \vfil

        {\LARGE
        Mitschrift \\[\bigskipamount]

        \bfseries{\itshape Bachelor of Science  \textup{(}B.Sc.\textup{)}}\\[\bigskipamount]
        }% Ende Large

        \vfil\vfil\vfil

        vorgelegt von\\
        \item ... \\
        \item \textsc{\vornameE\ \nameE } \\ \texttt{\emailE} \\  \item
        \textsc{\vornameS\ \nameS \qquad } \\ \texttt{\emailS}  \\
        \item ... \\
        Tag der Einreichung: \datum\\[\bigskipamount]

    \end{center}
    %

    \cleardoublepage



    \tableofcontents

    \thispagestyle{empty}

    %\makeatletter
    %\begin{titlepage}

    %	The title is \@title
    %	It was written by \@author\space on \@date

    %\end{titlepage}
    \setcounter{page}{0}
    \chapter*{Einleitung}

    \chapter{Folge und Reihen:}
    \section{Folgen und Reihen}
    \begin{definition}
        Ein folge ist eine Abbildung

        \[ f: \mathbb{N} \rightarrow \underbrace{\mathbf{M}}_{Menge} : \mathrm{n} \mapsto \underbrace{X_n}_{folgenglied} \]

    \end{definition}
    \begin{remark}

        \begin{align}	\mathbb{M} &= \mathbb{R} \quad reelewert Folge  \notag \\
        M&= \mathbb{C} \quad	 Komplexewetig folge  \notag \\
        \mathbb{M}&= \mathbb{R}^n \quad vertical folge \notag
        \end{align}




    \end{remark}
    \begin{description}

        \item[Bezeichnung]

        \quad $(X_n)$ \space \text{mit} \space $ \left( X_n =  \right) \frac{n}{n+1} $
        \\ \\ Aufzählung der folglieder: 0 , $\frac{1}{2}$ ,$\frac{2}{3}$ , $\frac{3}{4}$ , \dots
        \begin{remark}
            zuweten wird $\mathbb{N}$ durch $\mathbb{N}$ {0,1 \dots} erstellt.


        \end{remark}
        \begin{tikzpicture}
            \begin{axis}[
            xlabel= $x$,
            ylabel= {$y$}]
                \addplot { x/(x+1)};
            \end{axis}
        \end{tikzpicture}
    \end{description}
    \begin{example}
        \[\]
        \begin{enumerate}

            \item Konstante Folge $(X_n)$ mit \quad $X_n = a \in \mathbb{M },a \dots$ \\
            \[ X_n = a \in \mathbb{M} \]
            \item Harmonische Folge $(X_n)$  mit $X_n$ =  $\frac{1}{n+1}$ \quad$ n \geq 1$
            \item Geometrische folge $(X_n)$ mit $X_n = q^n \:. \: q \in \mathbb{R}, \dots $
            \item Fibonaccifolge $(X_n)$ mit
            \[ X_n =\frac{1}{\sqrt{5}} \Big(  \big( \frac{1+ \sqrt{5}}{2} \big)^n - \big( \frac{1- \sqrt{5}}{2}\big)^n   \Big)    \]

            \item Fibonacci folgen $(X_n)$
            \begin{align}
                X_0 &=0 \notag \\X_1 &= 1 \notag \\
                X_n+1 &= X_n+X_n-1 (n>0)  \notag
            \end{align}

            \item conway  folge
            \[ 1, 11 ,21 , 1211, 111217, 312211 \dots \]

            \item folge aller Primzahlen:- \[ 2, 3 ,5 ,7 ,11, 13 , \dots \]

        \end{enumerate}
    \end{example}

    \section{Rechnen mit Folgen }
    \begin{align*}
        M  = \mathbb{R} \quad oder \quad M = \mathbb{C} \\
        (X_n)+(y_n) := (X_n+y_n)\\
        K(X_n):=(KX_n)\in \mathbb{R} \quad  oder \quad \in  \mathbb{C}
    \end{align*}
    \begin{remark}
        \[\]
        Die Folge bildet ein Vektorraum.
    \end{remark}


    \begin{definition}


        \begin{enumerate}

            \mbox{}\item Eine reellwertige Funktion ist in der Mathematik eine Funktion, deren Funktionswerte reelle Zahlen sind.

            \item Eine reellwertige heißt beschränkt wenn gilt

            \[	\exists r \in \mathbb{R}_+ , \forall r \in \mathbb{N}: \underbrace{|x|}_{\text{ Betrag einer reellen oder komplexer Zahl}} \leqq r   \]

        \end{enumerate}
    \end{definition}

    \begin{example}
        \[(X_n)\quad mit \quad X_n = (-1)^n \times \frac{1}{n} \]
        \[-1 ,\quad \frac{1}{2}, \quad \frac{-1}{3}, \quad \frac{1}{4} ,\quad \frac{-1}{5},\dots \]
    \end{example}

    \begin{remark}

        $(X_n)$ ist beschränkt mit $r = 1$ denn $|(-1)^n \frac{1}{n}|=|\frac{1}{n}| \leqq 1  $

    \end{remark}

    \begin{tikzpicture}
        \begin{axis}[
        ymin = -2,
        ymax = 2,
        xmin = -3,
        xmax = 3,
        axis x line=center,
        axis y line=center]
            \addplot[samples at={-3,...,3},only marks,mark size=1]{(-1)^x*(1/x)};
        \end{axis}
    \end{tikzpicture}
    \begin{example}
        \[  (X_n) mit X_n = (-1)^n \frac{1}{n}+1 \quad \text{bechränkt r = 3/2}\]

        \[-1/3 \quad \leq X_n \leq 3/2 \quad \forall n \in \mathbb{N} \]
        \text{(((i have to repaire it)))}
        \begin{tikzpicture}
            \begin{axis}[
            ymin = -2,
            ymax = 2,
            xmin = -4,
            xmax = 4,
            axis x line=center,
            axis y line=center]
                \addplot[samples at={-3,...,3},only marks,mark size=1][ domain=(-1/3):(3/2)] {((-1)^x)*(1/x)+1};
            \end{axis}
        \end{tikzpicture}
    \end{example}
    \begin{example}
        \text{Stander Bsp: \newline}
        \text{Die folge}
        $ ((1+1/n)^n)^\infty_{n=1} $ \text{ist beschränkt durch 3}
        \text{zu zeigen }
        \[ -3 \leq X_n \leq 3 \text{für alle} n \in \mathbb{N} \]
        \[ { (a+b)^n = \sum_{k=0}^{n} \binom{n}{k} a^k . b^{n^k} = \sum_{k=0}^{n} \binom{n}{k} a^{n^k} b^k }  \]
        \[  \binom{n}{k} = \frac{n!}{k!(n-k!)} =  \frac{n(n-1 -(n-k-1))}{K!} \]
        \[  \sum_{K=0}^{n} 1/k = 1+ \frac{1}{2} + \frac{1}{2.3} + \frac{1}{2.3.4} + \dots \]
    \end{example}
    \section{geometrische Summen formel (Tafelwerk)}
    \begin{definition}

        Die Folge $(X_n)$ heißt monoton $\Big\{ wachsend fallend \Big\}$
        \[ wenn \quad gilt: \forall n \in \mathbb{N}:
        \left\{
        \begin{array}{ll}
            X_n  & \leq X_n +1 \\
            X_n  & \geq X_n+1
        \end{array}
        \right. \]
        \text{man spricht von Streng monotonie}
        $ wenn \leq durch > und \geq durch < \dots  $
    \end{definition}
    \begin{remark}
        \[ X_n \leq X_{n+1} \ \Leftrightarrow  X_n - X_{n+1} \geq 0 \quad \Leftrightarrow  \frac{x_n}{X_{n+1}} \leq 1 \]
    \end{remark}

    \begin{example}


        \[	(X_n) mit X_0 := 1 , X_{n+1} := \sqrt{X_n +6} \]
        \paragraph{Standard Bsp:}
        $ \big( \big(1+ \frac{1}{n} \big)^n \big) $ ist streng monoton wachsend
    \end{example}
    \begin{remark}

        \[
            \begin{tabular}{|c| c c |}
                \hline
                monoton & ja & nein  \\
                \hline
                \rule{0pt}{3ex}
                Beschränktkeit & $\frac{1}{n}$ & $(-1)^n$ \\
                nein & (n) & $(n)^n$ \\
                \hline
            \end{tabular}
        \]

    \end{remark}
    \begin{definition}
        $(X_n)$ heißt $\boldsymbol{Konvergenz}$ wenn $(X_n)$ ein grenzwert hat.
        $(X_n)$ heißt $\boldsymbol{Divergrnz}$ wenn sie keinen grenzwert hat.
    \end{definition}
    \begin{definition}(grenzwert)
    a $\in$ $\mathbb{R}$ heißt grenzwert von $(X_n)$, wenn gilt:
    \[ \underbrace{\forall > 0 }_{beliebes \; klein} \quad \underbrace{\exists \mathbb{N} \in \mathbb{N}}_{beliebes \;  klein \; \underbrace{\Rightarrow |X_n -a|< \varepsilon}_{a- \varepsilon \leq X_n \leq a+\varepsilon } } , \forall n \in \mathbb{N} : m \geq \mathbb{N}  \]
    \[ \text{Sei  } \varepsilon > 0 ; \varepsilon \text{  fest} \]
    \[ \text{alle folglieder$ X_n$ mit n } \geq \mathbb{N} \curvearrowright \]
    \begin{tikzpicture}[scale=1.1],

    \begin{axis}[
    height=8cm,
    width=10cm,
    ymax=2,
    ymin=-2,
    xmin=1,
    xmax=5.5,
    axis y line=left,
    axis x line=bottom,
    yticklabels={,, $ a - \varepsilon$, a , $a + \varepsilon$ },xtick={1,...,10},
    xticklabels={,,N,n},
    ]
        %\addplot {|x-a|<0};
        \addplot[dashed, name path =A] coordinates {(0,1) (5.5,1)};


        \addplot[thick, samples=50, smooth,domain=0:6,magenta, name path=V] coordinates {(3,-2)(3,3)};
        \addplot[dashed, name path =B] coordinates {(0,-1) (5.5,-1)};
        \addplot[gray, pattern=north west lines] fill between[of=A and B, soft clip={domain=3:6}];
        \addplot[samples at={0,...,5},only marks,mark size=1] {sin(x)};

    \end{axis}
    \end{tikzpicture}%


    \end{definition}
\end{document}}