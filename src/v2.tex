\begin{example}
\begin{equation}
\begin{aligned}
0,4 \overline{3} = \frac{3}{4} + \frac{3}{100} + \frac{3}{10000}+ \dots \\
\frac{4}{10} + \frac{3}{100}(\frac{1}{10})^0 + \frac{1}{10} + \frac{3}{10^2} + \dots \\
=\frac{4}{10} + \frac{3}{100} \times \frac{1}{1-\frac{1}{10}}\\
= \frac{4}{10}+ \frac{1}{30} = \frac{12+1}{30}= \frac{13}{30}
\end{aligned}
\end{equation}
\end{example}

\begin{example}
\begin{equation}
\begin{aligned}
\sum_{K=1}^\infty{\frac{1}{k}} \text{ist divergent , denn }\\
\lim_{n \to \infty} \sum_{K=1}^n{\frac{1}{k}} \text{ex. nicht ! }\\
S_n = \frac{1}{1} + \frac{1}{2} + \big(\frac{1}{3} + \frac{1}{4} \big)+
\big(\frac{1}{5} + \frac{1}{6} + \frac{1}{7} + \frac{1}{8} \big) 
+ \frac{1}{9} \dots + \frac{1}{16} + \dots \frac{1}{n} \\
 > 1 + \frac{1}{2} \big(\frac{1}{4} + \frac{1}{4} \big) + 
 \big(\frac{1}{8}+ \frac{1}{8}+ \frac{1}{8}+ \frac{1}{8}+ \big)+ \\
 \big( \frac{1}{10} + \dots + \dots + \frac{1}{10} \big)+ \dots + \frac{1}{n}= 1+ \frac{1}{2} + \frac{1}{2} + \frac{1}{2} + \frac{1}{2}+ \dots + \frac{1}{n}\\
 \Rightarrow \lim_{n \to \infty}S_n=\infty
\end{aligned}
\end{equation}
\end{example}

\section{Alleemiene Reihen}
\begin{lemma}
\begin{equation}
\begin{aligned}
\sum_{k=1}^\infty {\frac{1}{k^x}} \text{x fest}\\
\text{falls:}\\
x > 1 \Rightarrow konvergent\\
x \leq q \Rightarrow Divergent\\
\end{aligned}
\end{equation}
\end{lemma}


\begin{proof}{mit Monotoniekriterium}
\begin{equation}
(1)\bigg(\sum_{k=1}^n {\frac{1}{k^2}} \bigg) \text{\quad Monoton (wachsend)} 
\end{equation}
%fellend
\begin{equation}
(2)(\bigg(\sum_{k=1}^n {\frac{1}{k^2}} \bigg)) \text{ \quad ist beschränkt}
\end{equation}  \\




\end{proof}
\section{Rechnenreglen für Regeln:}
Konvergenden Reige kann man addieren , subtrahieren, mit einem Skaler multiplizieren  wie endlichen Summen \\ \underline{ABER:} \\
Das gilt im Allgemein nicht für das multiplizieren