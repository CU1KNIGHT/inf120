\section{Vorlesung 15}
\textbf{Mäuse problem }\\
4 Mäuse = 4 Masse punkte 
Gleich......, gleich förmige Bewegung zu jedem Punkt bewegt sich $m_i$ auf $m_{i+1}$ zu.
( Bahnkmorn ??? ) der Mäuse ? Z.B für $m_0$\\
$y(t)= \begin{pmatrix}
x(t)\\
y(t)
\end{pmatrix}$
Sei $M_0$ zu Punkt t m Punkt $p_0(x(t),y(t))$\\
 $\Rightarrow m_1$ ist im Punkt $p_1 (-y(t),x(t))$\\
 Richtungsvektor : $p_1 - p_0 = \begin{pmatrix}
 -y(t) - x(t)\\
 x(t)-y(t)
 \end{pmatrix}$
 
 
