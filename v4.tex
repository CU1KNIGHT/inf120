%%
%% Author: mahmoud
%% 19/04/18
%%


%fehlend ein Beispie mit g=A 
%2A=1^2+(1/2)^2

\section{Vorlesung 4}

\section{Reihen}
\begin{example}{Zur geometrischen Reihen}\\
gesucht : A
\[2A = 1^2 + (\frac{1}{2})^2+(\frac{1}{4})^2 + \dots + (\frac{1}{k})^2 + ... \]\\
\[= (\frac{1}{4})^0 + (\frac{1}{4})^1+ (\frac{1}{4})^2+(\frac{1}{2^2})^3+(\frac{1}{2^2})^k+ \dots \]\\
\[9 = \frac{1}{4} = \frac{1}{1- \frac{1}{4}} = \frac{1}{\frac{3}{4}} = \frac{4}{3}= 2A \Rightarrow A = \frac{2}{3}\] 
\end{example}

\begin{example}
\begin{equation}
\begin{aligned}
0,4 \overline{3} = \frac{3}{4} + \frac{3}{100} + \frac{3}{10000}+ \dots \\
\frac{4}{10} + \frac{3}{100}(\frac{1}{10})^0 + \frac{1}{10} + \frac{3}{10^2} + \dots \\
=\frac{4}{10} + \frac{3}{100} \times \frac{1}{1-\frac{1}{10}}\\
= \frac{4}{10}+ \frac{1}{30} = \frac{12+1}{30}= \frac{13}{30}
\end{aligned}
\end{equation}
wenn $0,4 \overline{3}$ erlaubt wäre, dann,\\
\[\frac{4}{10} + \frac{9}{100} \times \frac{10}{9} = \frac{4}{10} + \frac{1}{10} = \frac{5}{10} = \frac{1}{2} = 0.5  \]
\end{example}

\newpage
\begin{example}
	\[\sum_{R=1}^{\infty} \frac{1}{K} \text{ ist divergent ,denn  } \lim\limits_{\infty} \sum_{K=1}^{n} \frac{1}{k} \text{ ex. nicht } \]
	
	\[\frac{1}{1}+\frac{1}{2}+\frac{1}{3}+\frac{1}{4}+\frac{1}{5}+\frac{1}{6}+\frac{1}{7}+\frac{1}{8}+\frac{1}{9}+\dots+\frac{1}{16}+\dots+\frac{1}{n}\]
	
	\[ > 1+\frac{1}{2}+  \underbrace{\frac{1}{4}+\frac{1}{4}}+\underbrace{\frac{1}{8}+\frac{1}{8}+\frac{1}{8}+\frac{1}{8}}+\frac{1}{10}+\frac{1}{16}+\dots+\frac{1}{n}\] 
	
	\[1+\frac{1}{2} + \qquad \frac{1}{2}+   \quad  \qquad \frac{1}{2}  + \dots + \frac{1}{n}  \rightarrow \lim\limits_{n\rightarrow \infty }s_n = \infty \]
	
\end{example}
\section{Allgemeine harmonische Reihe}

				\begin{equation*}
				\begin{split}
				 \sum_{K=1}^{\infty} \frac{1}{k^\alpha} \quad( \infty \text{fest}) \qquad
				& \alpha > 1 \rightarrow \mathbb{R}  \\
				& \alpha \leqq \rightarrow dev
				\end{split}
				\end{equation*}
 \begin{example}
 	\[ \sum_{k=1}^{\infty} \frac{1}{k^2} \qquad \text{ist konvergent}\]
 	
\begin{beweis}{mit Monotoniekriterium für Folge}  
		\[ \text{Reihe ist konvergent} \left\{ \begin{array}{ll}
	   (1) & \mbox{ $\sumIn \frac{1}{k^2} $ \quad ist monoton wachsend};\\
	   (2) & \mbox{$\sumIn \fracKetwo $ \text{ist beschränkt}}.\end{array} \right. \] 
	   	
	   \[ \sumIin \fracKetwo = \frac{1}{1}+\frac{1}{2^2}+\frac{1}{2^2}+\frac{1}{4^2}+\frac{1}{5^2}+  \dots + \frac{1}{8^2}  \]
	   \[ <  1 + \frac{1}{4}+\underbrace{\frac{1}{2^2}+\frac{1}{2^2}}_{2. \frac{1}{4}}+\underbrace{\frac{1}{4^2}+\dots + \frac{1}{4^2}}_{4.\frac{1}{4^2}}+\]
	   
	   \[ 1+\frac{1}{4}+\frac{1}{2}.1+\underbrace{\frac{1}{4}}_{(\frac{1}{2})^2}+\underbrace{\frac{1}{8}}_{(\frac{1}{2})^3} =1+\frac{1}{4}+\frac{\frac{9}{4} }{1-\frac{1}{2}-1} \]
	   
 \end{beweis}
 \end{example}
%photo auf der Tafel
\section{Expotentiale Reihe}

\[ \sumOin \frac{1}{k!} = \limNin ( 1-\frac{1}{n})^n =:e \text{ist konvergent} \]

 \pgfkeys{/forest,
	tria/.append style={ellipse, draw},
}
	\begin{forest}
	for tree={l+=0.5cm, s sep+=1cm}
	[Konvergentkreterium für Reihen 
	[	für Folge ]
	[Kreterien für absolute konvergenz [konvergent Reihen] ]
	[Hauptkriterium ]
	]
	\end{forest}

\section{Hauptkriterium}  
$$\sum_{k=0}^{\infty}{a_k} \quad \text{konvergent}
 \Rightarrow (a_k) \text{Nullfolge.}$$ \\
$$\lim\limits_{k \rightarrow \infty}{a_k} \neq 0 \Rightarrow  \sum_{k=0}^{\infty}{a_k} \quad \underbrace{null konvergent }_{divergend}$$\\
oder
			$$\lim\limits_{k \rightarrow -\infty}{ a_k} \quad ex.null$$

						\begin{example}
						
						\[ \sumIin \frac{3k^2+1}{4k^2-1} \quad divergend, \quad aber \quad \sumIin \fracKone \quad divergend \quad \text{und $\fracKone$ Null folge }  \]
						

						\end{example}
\begin{beweis}
\[  \sumOin a_k konv. \Rightarrow \underbrace{(a_k Nullfolge) }_{\limKin a_k=0}  \] 
\[s_n =\sumOn a_k , s_{n+1} =\sum_{k=0}^{n+1} \qquad s_{n+1}=s_n + a_{n+1} \]
\[s= \limNin s_n= \limNin s_{n+1} \qquad \limNin a_{n+1} = \limNin s_{n+1} - \limNin s_n = s-s=0 \]
\end{beweis}

\newpage
\section{Kriterium für Alternierende  Reihe}
\begin{beweis}{Alternierende }
	$\sumOin (-1)^k \fracKone$ ist konvergent\\
	$\sumOin (-1)^k a_k$\\
	wobei $(a_k)$ einer Streng monoton fallend Nullfolge mit $a_k \geqq 0 $ \\
	$\Rightarrow$ Die Reihe ist konvergent. Also $\sumOin (-1)^k \fracKone$ ist konvergent.

\end{beweis}
\begin{definition}[absolute Reihe]
	
	\text{Reihe} $\sumOin  a_k$  \text{heißt absolute konvergent wenn} $\sumOin $ $|a_k|$ \text{konvergent ist.
	}
\end{definition}
\begin{example}

\[	\sumIin (-1)^k \fracKone \text{ ist konvergent , aber nicht absolote konvergent } \]
\end{example}

\begin{example}
\[ \sumIin (-1)^k \fracKetwo  \text{ ist kovergend und abslote konvergent } \]
\end{example}

\begin{theorem}
Reihe $\sumOin$ $a_k$ abslot konvergent $\Rightarrow$ Reihe $\sumOin$ $a_k$ ist kovergend	
\end{theorem}

\begin{remark}	
Absolute konvergente Reihe kann man multiplizieren wie endliche summen d Reihen null
\end{remark}

\section{Quotionkriterium (QK):}
für endliche Konvergenz\\
\[\limKin |\frac{a_{k+1}}{a_k}|\] \\$< 1 \Rightarrow \sumOin a_k $ in absolut konvergent \\
$>1 \Rightarrow $ ist divergent\\
$=1 $ Kriterium ist nicht anwendbar 
\section{Wurzel kriterium : WK }
für (absolute) konvergent 
\\ $\limKin \sqrt[k]{|a_k|}$\\
$<1 \Rightarrow \sumOin a_k$ in (absolute) konvergent\\
$>1 \Rightarrow$ divergent\\
$=1 $ Kriterium ist nicht anwendbar

\begin{example}[QK]
	 \begin{equation*}
		 \begin{split}
	    	\sumOin \dfrac{1}{k!}
	 	   \limKin {|d\frac{\frac{1}{(k+1)!}}{\frac{1}{k!}}|} &= \limKin \frac{k!}{(k+1)!} \\
													   	   &=\limNin \frac{1}{k+1}   \\
													   	   &=0<1 \Rightarrow  Reihe \quad als \quad konv. 
		    \end{split}	 
		 \end{equation*}     
	
\end{example}
\begin{example}[WK]
	\begin{equation*}
	\begin{split}
	\limKin \sqrt[k]{\frac{1}{k!}}= \limKin \frac{\sqrt[k]{1}}{\sqrt[k]{k!}}=\frac{1}{\limKin \sqrt[k]{k!}}=0\\ <1 \\
	\Rightarrow Reihe \quad als \quad konv.
	\end{split}	 
	\end{equation*}     
	
\end{example}