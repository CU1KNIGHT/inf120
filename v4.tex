%%
%% Author: mahmoud
%% 19/04/18
%%


%fehlend ein Beispie mit g=A 
%2A=1^2+(1/2)^2

\section{Vorlesung 4 \href{https://tu-dresden.de/mn/math/algebra/das-institut/beschaeftigte/antje-noack/ressourcen/dateien/v120-1/MathMethInf04.pdf?lang=en}{(16.04.2019)} }
\section{Reihen}
\begin{example}{Zur geometrischen Reihen}\\
gesucht : A
\[2A = 1^2 + (\frac{1}{2})^2+(\frac{1}{4})^2 + \dots + (\frac{1}{k})^2 + ... \]\\
\[= (\frac{1}{4})^0 + (\frac{1}{4})^1+ (\frac{1}{4})^2+(\frac{1}{2^2})^3+(\frac{1}{2^2})^k+ \dots \]\\
\[9 = \frac{1}{4} = \frac{1}{1- \frac{1}{4}} = \frac{1}{\frac{3}{4}} = \frac{4}{3}= 2A \Rightarrow A = \frac{2}{3}\] 
\end{example}

\begin{example}
\begin{equation}
\begin{aligned}
0,4 \overline{3} = \frac{3}{4} + \frac{3}{100} + \frac{3}{10000}+ \dots \\
\frac{4}{10} + \frac{3}{100}(\frac{1}{10})^0 + \frac{1}{10} + \frac{3}{10^2} + \dots \\
=\frac{4}{10} + \frac{3}{100} \times \frac{1}{1-\frac{1}{10}}\\
= \frac{4}{10}+ \frac{1}{30} = \frac{12+1}{30}= \frac{13}{30}
\end{aligned}
\end{equation}
wenn $0,4 \overline{3}$ erlaubt wäre, dann,\\
\[\frac{4}{10} + \frac{9}{100} \times \frac{10}{9} = \frac{4}{10} + \frac{1}{10} = \frac{5}{10} = \frac{1}{2} = 0.5  \]
\end{example}

\newpage
\begin{example}
	\[\sum_{R=1}^{\infty} \frac{1}{K} \text{ ist divergent ,denn  } \lim\limits_{\infty} \sum_{K=1}^{n} \frac{1}{k} \text{ ex. nicht } \]
	
	\[\frac{1}{1}+\frac{1}{2}+\frac{1}{3}+\frac{1}{4}+\frac{1}{5}+\frac{1}{6}+\frac{1}{7}+\frac{1}{8}+\frac{1}{9}+\dots+\frac{1}{16}+\dots+\frac{1}{n}\]
	
	\[ > 1+\frac{1}{2}+  \underbrace{\frac{1}{4}+\frac{1}{4}}+\underbrace{\frac{1}{8}+\frac{1}{8}+\frac{1}{8}+\frac{1}{8}}+\frac{1}{10}+\frac{1}{16}+\dots+\frac{1}{n}\] 
	
	\[1+\frac{1}{2} + \qquad \frac{1}{2}+   \quad  \qquad \frac{1}{2}  + \dots + \frac{1}{n}  \rightarrow \lim\limits_{n\rightarrow \infty }s_n = \infty \]
	
\end{example}
\section{Allgemeine harmonische Reihe}

				\begin{equation*}
				\begin{split}
				 \sum_{K=1}^{\infty} \frac{1}{k^\alpha} \quad( \infty \quad \text{fest} ,\quad mit \quad \alpha \in \mathbb{R}) \qquad
				&\text{falls} \quad \alpha \geq 1 \rightarrow \text{konvergent}  \\
				&\text{falls} \quad \alpha \leq 1 \rightarrow \text{Divergent} 
				\end{split}
				\end{equation*}
 \begin{example}
 	\[ \sum_{k=1}^{\infty} \frac{1}{k^2} \qquad \text{ist konvergent}\]\\
 	\[ \sum_{k=1}^{\infty} \frac{1}{k^{\frac{1}{2}}} \qquad \text{ist Divergent}\]
 	
\begin{beweis}[Monotoniekriterium]
{mit Monotoniekriterium für Folge}  
		\[ \text{Reihe ist konvergent} \left\{ \begin{array}{ll}
	   (1) & \mbox{ $\sumIn \frac{1}{k^2} $ \quad ist monoton wachsend}.\\
	   (2) & \mbox{$\sumIn \fracKetwo $ \text{ist beschränkt}}.\end{array} \right. \] 
	   	
	   \[ \sumIin \fracKetwo = \frac{1}{1}+\frac{1}{2^2}+\frac{1}{2^2}+\frac{1}{4^2}+\frac{1}{5^2}+  \dots + \frac{1}{8^2}  \]
	   \[ <  1 + \frac{1}{4}+\underbrace{\frac{1}{2^2}+\frac{1}{2^2}}_{2. \frac{1}{4}}+\underbrace{\frac{1}{4^2}+\dots + \frac{1}{4^2}}_{4.\frac{1}{4^2}}+\]
	   
	   \[ 1+\frac{1}{4}+\frac{1}{2}.1+\underbrace{\frac{1}{4}}_{(\frac{1}{2})^2}+\underbrace{\frac{1}{8}}_{(\frac{1}{2})^3} =1+\frac{1}{4}+\frac{\frac{9}{4} }{1-\frac{1}{2}-1} \]
	   
 \end{beweis}
 \end{example}
%photo auf der Tafel
\section{Expotentiale Reihe}

\[ \sumOin \frac{1}{k!} = \limNin ( 1-\frac{1}{n})^n =: e \quad \text{ist konvergent} \]

 \pgfkeys{/forest,
	tria/.append style={ellipse, draw},
}
	\begin{forest}
	for tree={l+=0.5cm, s sep+=1cm}
	[Konvergentkreterium für Reihen 
	[	für Folge ]
	[Kreterien für absolute konvergenz [konvergent Reihen] ]
	[Hauptkriterium ]
	]
	\end{forest}

\section{Hauptkriterium}
$$ \star \quad \text{konvergent die Reihe} \quad \sum_{k=0}^{\infty}{a_k} \quad \text{dann ist} \quad (a_k) \quad\text{Nullfolge.}$$ \\
$$\lim\limits_{k \rightarrow \infty}{a_k} \neq 0 \Rightarrow  \sum_{k=0}^{\infty}{a_k} \quad \underbrace{null konvergent }_{divergend}$$\\
oder
			$$\lim\limits_{k \rightarrow -\infty}{ a_k} \quad ex.null$$

						\begin{example}
						
						\[ \sumIin \frac{3k^2+1}{4k^2-1} \quad divergend, \quad aber \quad \sumIin \fracKone \quad divergend \quad \text{und $\fracKone$ Null folge }  \]
						

						\end{example}
\begin{beweis}
\[  \sumOin a_k \quad (konvergent) \Rightarrow \underbrace{(a_k) \quad Nullfolge }_{\limKin a_k=0}  \] 
\[s_n =\sumOn a_k \quad ,\quad s_{n+1} =\sum_{k=0}^{n+1}{a_k} \quad ,\quad s_{n+1}= s_n + s_{n+1} \] \\
\[ s = \limNin s_n= \limNin s_{n+1}  \quad ,\quad \limNin s_{n+1} = \limNin s_{n+1} - \limNin s_n = s-s=0 \]
\end{beweis}

\newpage
\section{Kriterium für Alternierende  Reihe}
\begin{beweis}[Alternierende Reihe]
	$$\sumOin (-1)^k \fracKone \text{ist konvergent }$$
	$$ \sumOin (-1)^k a_k $$
	wobei $(a_k)$ einer Streng monoton fallend Nullfolge mit $a_k \geqq 0 $ \\
	$\Rightarrow$ Die Reihe ist konvergent. $$ Also \sumOin (-1)^k \fracKone \quad \text{ist konvergent.} $$ 

\end{beweis}
\begin{definition}[absolute Reihe]
	
$$ \text{Eine Reihe}  \sumOin  a_k \quad \text{heißt absolute konvergent wenn} \sumOin |a_k| \text{konvergent ist.
	}$$
\end{definition}
\begin{example}

\[	\sumIin (-1)^k \fracKone \text{ ist konvergent , aber nicht absolote konvergent } \]

\[ \sumIin (-1)^k \fracKetwo  \text{ ist kovergend und \textbf{absolute} konvergent } \]
\end{example}


\begin{theorem}
$$ Reihe \sumOin a_k \quad \text{absolut konvergent} \quad \Rightarrow \quad \text{Reihe} \sumOin a_k \quad \text{ist Konvergent} $$	
\end{theorem}

\begin{remark}	
absolute konvergente Reihe kann man multiplizieren wie endliche summen. (aber konvergente Reihen nicht !)
\end{remark}

\section{Quotienkriterium (QK):}
Für absolute  Konvergenz , wenn gilt:\\

\begin{align*}
\limKin \bigg|\frac{a_{k+1}}{a_k} \bigg|
\begin{cases}
< 1 \Rightarrow  \sum_{k=0}^{\infty} \quad \text{ist absolut konvergent}\\
> 1 \Rightarrow \quad \text{ist divergent)}\\
= 1 \Rightarrow \quad \text{Kriterium ist nicht anwendbar} 
\end{cases}
\end{align*}

\section{Wurzel Kriterium : WK }
Die Reihe $\sumOin a_k$ ist  \textbf{absolute} konvergent genau wenn $\Leftrightarrow$ :
\begin{align*}
\limKin \sqrt[k]{|a_k|}
\begin{cases}
< 1 \Rightarrow \sumOin a_k \quad \text{ist absolut konvergent}\\
> 1 \Rightarrow \quad \text{ist divergent}\\
= 1 \Rightarrow \quad \text{Kriterium ist nicht anwendbar} 
\end{cases}
\end{align*}

\begin{example}[QK]
	    \begin{gather*}
	    	\sumOin \dfrac{1}{k!}\\
	 	   \limKin {\bigg|\frac{\frac{1}{(k+1)!}}{\frac{1}{k!}}\bigg|}  \\
	 	   = \limKin \frac{k!}{(k+1)!} \\
													   	   =\limKin \frac{1}{k+1} = 0   \\
													   	  d.h \quad <1 \Rightarrow \quad \text{Die Reihe ist absoulte Konvergent.}  
		    \end{gather*}     
	
\end{example}
\begin{example}[WK]
	\begin{gather*}
	\limKin \sqrt[k]{\big|\frac{1}{k!}\big|} \\
=\limKin \frac{\sqrt[k]{1}}{\sqrt[k]{k!}}
=\\
	\frac{1}{\limKin \sqrt[k]{k!}}=0 <1 \\
	\text{Die Reihe is absolut konvergent.}
	\end{gather*}	 
	     
	
\end{example}



 