\section{vorlesung 2}
\begin{text}
	ist die folge beschränkt , monoton ?\\
	
	$(x_n)$ konvergierend : $\iff \exists a \in\mathbb{R} \quad \forall \epsilon > 0 \quad \exists n \quad \in N \quad \forall n \in N \quad \\
	n \geq N \Rightarrow |x_n - a |< \epsilon $
\end{text}


\begin{theorem}
	
	$(x_n)$ konvergierend : $\Rightarrow$ Der Grenzwert ist eindeutig beschränkt.
	
\end{theorem}



\begin{beweis}
	Sei a eine Grenzwert von $(x_n)$ , b eine Grenzwert von $(x_n)$ \\
	d.h sei $\epsilon > 0$,$\epsilon$ beliebig , $\epsilon$ fest \\
	
	
	\begin{equation}
	\exists  N_a \quad \forall n \geq N_a : |x_n-a|< \epsilon
	\end{equation}
	
	
	\begin{equation}
	\exists  N_b \quad \forall n \geq N_b : |x_n-b|< \epsilon
	\end{equation}
	
	Sei max $\{N_a,N_b\}=N$
	dann gilt : \\
	\begin{equation}
	n \geq N \Rightarrow |x_n - a| < \epsilon
	\end{equation}
	
	und \begin{equation}
	|x_n -b| < \epsilon \Rightarrow |x_n -a|+|x_n - b|< 2\epsilon
	\end{equation}\\
	Annahme :- a $\neq$ b , d.h $|a-b|\neq 0 $
	\begin{gather*}
	|a-b|=|a+0-b|\\
	=|(a-x_n)+(x_n-b)| \leq |x_n - a|+|x_n-b|< 2 \epsilon \\
	also \quad |a - b|< 2 \epsilon
	\end{gather*}
	wähle Z.B  \[\epsilon = \frac{|a-b|}{3}
	\quad
	\text{dann gilt}\ :|a-b|< \frac{2 \ |a-b|}{3}\]\\
	
	\[ \Rightarrow 1 < \frac{2}{3} \quad  \text{ falls  Aussage, Widerspruch  also  ist  die  Annahme  falsch  also  gilt }\ \quad a=b\]
\end{beweis}
\newpage


\begin{example}
	
	$x_n$ mit $x_n = \frac{1}{n}$ (harmonische Folge)
	
\end{example}

\begin{beweis}
	Sei $\epsilon > 0 , \epsilon belibig , \epsilon fest$
	gesucht : N mit $n \geq$ N \\
	hat den Grenzwert 0
	
	\begin{gather}
	\Rightarrow |x_n-a|= |\frac{1}{n} =0|=\frac{1}{n}<\epsilon
	\end{gather}
	
	wähle N:= $\lceil \frac{1}{\epsilon} \rceil +1$
	
\end{beweis}

\begin{example}
	$\epsilon = \frac{1}{100}$ , gesucht N mit $n \geq N$
	$\Rightarrow \frac{1}{n} < \frac{1}{100}$ wähle $N=101$\\
	
	
	\begin{schreibweise}
		$x_n$ hat den Grenzwert a Limes
		$\lim\limits_{n \rightarrow \infty}{x_n}=a$
		$x_n$ geht gegen a für n gegen Unendlich.
	\end{schreibweise}
\end{example}

\begin{definition}[Nullfolge]
	$x_n$ heißt Nullfolge ,wenn $\lim\limits{x_n}=0$ gilt.
\end{definition}

\begin{remark}
	
	Es ist leichter, die konvergente einer Folge zu beweisen, als den Grenzwert auszurechnen.
	
\end{remark}

\begin{example}
	$x_n = \dfrac{1}{3} + \big(\dfrac{11-n}{9-n}\big)^9$\\
	Behauptung: $\lim\limits_{n \rightarrow \infty}{x_n}=\dfrac{-2}{3}$
	
	\newpage
	\begin{lemma}
		\begin{gather}
		\lim\limits_{n \rightarrow \infty}{x_n+y_n}=
		(\lim\limits_{n \rightarrow \infty}{x_n}) +
		(\lim\limits_{n \rightarrow \infty}{y_n})
		\end{gather}
	\end{lemma}
	
	\begin{gather}
	=\lim\limits_{n \rightarrow \infty}{\bigg(\big(\frac{1}{3}\big)+\bigg(\frac{11-n}{9+n}\bigg)^9\bigg)}
	= \lim\limits_{n \rightarrow \infty}{\frac{1}{3}+
		\lim\limits_{n \rightarrow \infty}{\bigg(\frac{11-n}{9+n}\bigg)^9}}
	\end{gather}
	
	\begin{gather}
	= \frac{1}{3} + \bigg(\lim\limits_{n \rightarrow \infty}{\frac{11-n}{9+n}}\bigg)^9
	\end{gather}
	
	
	\begin{gather}
	= \frac{1}{3} + \lim\limits_{n \rightarrow \infty}{\Bigg(\frac{n(\frac{1}{n}-1)}{n(\frac{9}{n}+1)}\Bigg)^9}
	\end{gather}
	
	\begin{gather}
	= \frac{1}{3}+\Bigg(\frac{\lim\limits_{n \rightarrow \infty}{(\frac{11}{n})}}{\lim\limits_{n \rightarrow \infty}{(\frac{9}{n}+1})}\Bigg)^9
	\end{gather}
	
	\begin{gather}
	= \frac{1}{3} + \Bigg(
	\frac{\lim\limits_{n \rightarrow \infty}{\frac{11}{n}} - \lim\limits_{n \rightarrow \infty}{1}}{\lim\limits_{n \rightarrow \infty}{\frac{9}{n}+\lim\limits_{n \rightarrow \infty}{1} } }   \Bigg)^9
	\end{gather}
	
	
	\begin{gather}
	=\Bigg(
	\frac{\lim\limits_{n \rightarrow \infty}{11} \times \lim\limits_{n \rightarrow \infty}{(\frac{1}{n})-1}}{\lim\limits_{n \rightarrow \infty}{9 \times \lim\limits_{n \rightarrow \infty}{(\frac{1}{n})+1} } }   \Bigg)^9
	\end{gather}
	
	\begin{gather}
	\frac{1}{3}+(-1)^9 = \frac{1}{3}-1 = \frac{-2}{3}
	\end{gather}
\end{example}

\begin{definition}[Unendliche Grenzwert]
	Eine Folge $(x_n)$ hat den unendliche Grenzwert $\infty$, wenn gilt : \\
	\[\forall r \in \mathbb{R} \quad \exists N \in N \quad \forall n \geq N : x_n > r \]
	
	\begin{schreibweise}
		$\lim\limits_{n \rightarrow \infty}{x_n}= \infty$
	\end{schreibweise}
\end{definition}

\begin{remark}
	$\infty$ ist keine Grenzwerte und keine reelle Zahl.
\end{remark}

\begin{remark}
	Grenzwertsätze gelten nicht für uneigentliche Grenzwerte.
\end{remark}

\begin{remark}
	gilt $\lim\limits_{n \rightarrow \infty}{x_n}= \infty$ dann schreibt man $\lim\limits_{n \rightarrow \infty}{-x_n}= -\infty$
\end{remark}

\begin{example}
	$x_n$ mit $x_n = q^n$ , $q \in \mathbb{R}$ , $q$ fest.\\
	
	$ \lim\limits_{n \rightarrow \infty}{q^n} = \begin{cases}
	0 ,\quad |q|<1 \\
	1 ,\quad |q|=1 \\
	\infty ,\quad\quad q > 1  \\
	ex. nicht ,\quad q\leq -1
	\end{cases}$
\end{example}
\vfil

\section{Konvergenzkriterien}
(zum Beweis der Existenz eine Grenzwert, nicht zum berechnen von Grenzwert) \\


(1) $x_n$ konvergent $\Rightarrow$ $(x_n)$ beschränkt. \\

wenn $(x_n)$ nicht beschränkt $\Rightarrow$ $(x_n)$ nicht konvergent.\\


(2) Monotonie Kriterium:
wenn $(x_n)$ beschränkt ist können wir fragen ob $(x_n)$    konvergent.\\


$(x_n)$ beschränkt von Monotonie $\Rightarrow$ $(x_n)$ konvergent.

\begin{example} %2.17 
	
	$\Big((-1)^n \times\frac{1}{n} \Big)$ konvergent (Nullfolge) diese Folge ist beschränkt aber nicht Monoton 
	$$ \lim_{n \to \infty}{\Big( \big(1 + \frac{1}{n}\big)^n \Big)} $$  existiert. Diese ist beschränkt und monoton. \\
	$$\Rightarrow \lim_{n \to \infty}{(1+\frac{1}{n})^n}$$ 
	existiert. 
	$$\lim_{n \to \infty}{(1+\frac{a}{n})=e^a}$$
\end{example}
\newpage
